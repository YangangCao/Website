\documentclass[10pt,a4paper,oneside]{article}
\usepackage[utf8]{inputenc}
\usepackage[T1]{fontenc}
\usepackage{amsmath}
\usepackage{amsfonts}
\usepackage{amssymb}
\usepackage{graphicx}



\author{Baboo J. Cui}
\title{Topic 02 - Limits and Derivatives}
\begin{document}
\maketitle
\tableofcontents

\newpage

A series of problems lead to limits and derivative, here are $2$ examples:
\begin{itemize}
	\item \textbf{tangent line}: use \textbf{secant line} to approach tangent line
	\item \textbf{instantaneous velocity}: use \textbf{average velocity} in a shorter period time to estimate instantaneous velocity
\end{itemize}

\section{Limit of a Function}

\subsection {Definition of Limit}
Limit of a function is written as
\[
\lim _{x \rightarrow a} f(x) = L
\]
it means that the limit of $f$ as $x$ approaches $a$ equals $L$. Note that the limit has nothing to do with $f(a)$.

\subsection{Term Approach}
\begin{itemize}
	\item informally, approach means getting closer to a certain value
	\item formally definition will be introduced latter
\end{itemize}

\subsection{One-Sided Limits}
For left-hand limit:
\[
\lim _{x \rightarrow a^{-}} f(x) = L 
\]
For right-hand limit:
\[
\lim _{x \rightarrow a^{+}} f(x) = L 
\]
Note that
\[
\lim _{x \rightarrow a} f(x) = L 
\]
exists only when both left-hand and right-hand side limits exist.

\subsection{Infinite Limits}
Let $f(x)$ is defined on $(a-\delta, a) \cup (a,a+\delta)$,
\[
\lim _{x \rightarrow a} f(x) = \infty
\]
leads to infinite limits. Note that:
\begin{itemize}
	\item here $\infty$ can be either $+\infty$ or $-\infty$
	\item it often occurs when the function is not defined at $x = a$ or $a$ is pole of ration functions
\end{itemize}

\subsection{Vertical Asymptote}
The line $x = a$ is called a vertical asymptote of $y = f(x)$ if
\[
\lim_{x \rightarrow a, a^{+} \text{ or } a^{-}} = \infty
\]
natural log has a vertical asymptote $x = 0$ since
\[
\lim_{x \rightarrow 0^{+}} \ln x = - \infty
\]

\section{Limit Laws and Theorem}

\subsection{Limit Laws}
Suppose $f$ and $g$ are two functions and $c$ is a constant:
\begin{enumerate}
	\item \textbf{sum}: $\lim (f+g) = \lim  f + \lim g$
	\item \textbf{difference}: $\lim (f-g) = \lim  f - \lim g$
	\item \textbf{constant multiplication}: $\lim (cf) = c \lim  f$
	\item \textbf{product}: $\lim (fg) = (\lim  f)  (\lim g)$
	\item \textbf{quotient}: $\lim (f/g) = \lim  f / \lim g$, if $\lim g \neq 0$
\end{enumerate}
From the laws above, the following laws can be obtained:
\begin{itemize}
	\item \textbf{power}: $\lim f^n = (\lim  f)^n$
	\item \textbf{limit of constant}: $\lim c = c$
\end{itemize}

\subsection{Direct Substitution Property}
If $f$ is a polynomial or rational function and $a$ is in the domain of $f$, then
\[
\lim_{x \rightarrow a} f(x) = f(a)
\]
In fact this property is true if the function is \textbf{continuous} at $x=a$.

\subsection{A Useful Fact}
If $f(x) = g(x)$ when $x \neq a$, then
\[
\lim_{x \rightarrow a} f(x) = \lim_{x \rightarrow a} g(x)
\]
provide the limit exists, this is very useful to get the limit of a ration function whose pole that can be canceled by its numerator factor.

\subsection{Limit Theorems}
\begin{itemize}
	\item \textbf{two sides theorem}: $\lim_{x \rightarrow a} f(x) = L$ if and only if
	\[
	\lim_{x \rightarrow a^{+}} f(x) = \lim_{x \rightarrow a^{-}} f(x) = L
	\]
	\item \textbf{comparison theorem}: if $f(x) \leq g(x)$ and both function have limits at $x = a$, then
	\[
	\lim_{x \rightarrow a} f(x) \leq \lim_{x \rightarrow a} g(x) 
	\]
	\item \textbf{squeeze theorem}: if $f(x) \leq g(x) \leq h(x)$ and
	\[
	\lim_{x \rightarrow a} f(x) = \lim_{x \rightarrow a} h(x) = L
	\]
	then
	\[
	\lim_{x \rightarrow a} g(x) = L
	\]
\end{itemize}

\section{Precise Definition of Limit}
Let $f$ be a function defined on some open interval that contains the number $a$ except possibly at $a$ itself. Then we say the limit of $f$ as $x$ approaches $a$ is $L$:
\[
\lim_{x \rightarrow a} f(x) = L
\]
if for \textbf{every} number $\epsilon>0$ there is a number $\delta>0$ such that
 \[
 |x-a| < \delta \Longrightarrow |f(x) - L| < \epsilon
 \]
 Comment: range constrain leads to domain constrain.

\subsection{Precise Definition of Left Hand and Right Hand Limit}
Similar way can be applied to these $2$ definitions:
\begin{itemize}
	\item \textbf{Left-Hand Limit}:
	\[
	\lim_{x \rightarrow a^{-}} f(x) = L 
	\]
	if for every number $\epsilon>0$ there is a number $\delta>0$ such that
	\[
    a- \delta < x < a \Longrightarrow |f(x) - L| < \epsilon
	\]
	\item \textbf{Right-Hand Limit}:
	\[
	\lim_{x \rightarrow a^{+}} f(x) = L 
	\]
	if for every number $\epsilon>0$ there is a number $\delta>0$ such that
	\[
	a < x < a + \delta  \Longrightarrow |f(x) - L| < \epsilon
	\]
\end{itemize}

\subsection{Precise Definition of Infinite Limit}
Let $f$ be a function defined on some open interval that contains the number $a$ except possibly at $a$ itself. Then
\[
\lim_{x \rightarrow a} f(x) = +\infty
\]
if for \textbf{every} positive number $M$ there is $\delta>0$ such that
\[
|x-a| < \delta \Longrightarrow f(x)>M
\]
Similar definition can be applied to negative infinite limit.

\section{Continuity of Functions }





















\section{Extra}

\subsection{Greatest Integer Functions}
It is defined as:
\[
y = f(x) =\lfloor x \rfloor
\]
\begin{itemize}
	\item piece-wise function
	\item continuous from right side  
\end{itemize}

\subsection{Triangle Inequality}
It states that:
\[
|a+b| \leq |a| + |b|
\]
\end{document}