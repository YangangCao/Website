\documentclass[10pt,a4paper,oneside]{article}
\usepackage[utf8]{inputenc}
\usepackage[T1]{fontenc}
\usepackage{amsmath}
\usepackage{amsfonts}
\usepackage{amssymb}
\usepackage{graphicx}


\usepackage{tikz}
\usepackage{xcolor}
\usepackage{eso-pic}
\newcommand{\watermark}[3]{\AddToShipoutPictureBG{
		\parbox[b][\paperheight]{\paperwidth}{
			\vfill%
			\centering%
			\tikz[remember picture, overlay]%
			\node [rotate = #1, scale = #2] at (current page.center)%
			{\textcolor{gray!70!cyan!40!red!20}{#3}};
			\vfill}}}

\author{Baboo J. Cui}
\title{Limits and Derivatives}
\begin{document}
\watermark{50}{9}{www.baboocui.club}
\maketitle
\tableofcontents

\newpage

A series of problems lead to limits and derivative, here are $2$ examples:
\begin{itemize}
	\item \textbf{tangent line}: use \textbf{secant line} to approach tangent line
	\item \textbf{instantaneous velocity}: use \textbf{average velocity} in a shorter period time to estimate instantaneous velocity
\end{itemize}

\section{Limit of a Function}

\subsection {Definition of Limit}
Limit of a function is written as
\[
\lim _{x \rightarrow a} f(x) = L
\]
it means that the limit of $f$ as $x$ approaches $a$ equals $L$. Note that the limit has nothing to do with $f(a)$.

\subsection{Term Approach}
\begin{itemize}
	\item informally, approach means getting closer to a certain value
	\item formally definition will be introduced latter
\end{itemize}

\subsection{One-Sided Limits}
For left-hand limit:
\[
\lim _{x \rightarrow a^{-}} f(x) = L 
\]
For right-hand limit:
\[
\lim _{x \rightarrow a^{+}} f(x) = L 
\]
Note that
\[
\lim _{x \rightarrow a} f(x) = L 
\]
exists only when both left-hand and right-hand side limits exist.

\subsection{Infinite Limits}
Let $f(x)$ is defined on $(a-\delta, a) \cup (a,a+\delta)$,
\[
\lim _{x \rightarrow a} f(x) = \infty
\]
leads to infinite limits. Note that:
\begin{itemize}
	\item here $\infty$ can be either $+\infty$ or $-\infty$
	\item it often occurs when the function is not defined at $x = a$ or $a$ is pole of ration functions
\end{itemize}

\subsection{Vertical Asymptote}
The line $x = a$ is called a vertical asymptote of $y = f(x)$ if
\[
\lim_{x \rightarrow a, a^{+} \text{ or } a^{-}} = \infty
\]
natural log has a vertical asymptote $x = 0$ since
\[
\lim_{x \rightarrow 0^{+}} \ln x = - \infty
\]

\section{Limit Laws and Theorem}

\subsection{Limit Laws}
Suppose $f$ and $g$ are two functions and $c$ is a constant:
\begin{enumerate}
	\item \textbf{sum}: $\lim (f+g) = \lim  f + \lim g$
	\item \textbf{difference}: $\lim (f-g) = \lim  f - \lim g$
	\item \textbf{constant multiplication}: $\lim (cf) = c \lim  f$
	\item \textbf{product}: $\lim (fg) = (\lim  f)  (\lim g)$
	\item \textbf{quotient}: $\lim (f/g) = \lim  f / \lim g$, if $\lim g \neq 0$
\end{enumerate}
From the laws above, the following laws can be obtained:
\begin{itemize}
	\item \textbf{power}: $\lim f^n = (\lim  f)^n$
	\item \textbf{limit of constant}: $\lim c = c$
\end{itemize}

\subsection{Direct Substitution Property}
If $f$ is a polynomial or rational function and $a$ is in the domain of $f$, then
\[
\lim_{x \rightarrow a} f(x) = f(a)
\]
In fact this property is true if the function is \textbf{continuous} at $x=a$.

\subsection{A Useful Fact}
If $f(x) = g(x)$ when $x \neq a$, then
\[
\lim_{x \rightarrow a} f(x) = \lim_{x \rightarrow a} g(x)
\]
provide the limit exists, this is very useful to get the limit of a ration function whose pole that can be canceled by its numerator factor.

\subsection{Limit Theorems}
\begin{itemize}
	\item \textbf{two sides theorem}: $\lim_{x \rightarrow a} f(x) = L$ if and only if
	\[
	\lim_{x \rightarrow a^{+}} f(x) = \lim_{x \rightarrow a^{-}} f(x) = L
	\]
	\item \textbf{comparison theorem}: if $f(x) \leq g(x)$ and both function have limits at $x = a$, then
	\[
	\lim_{x \rightarrow a} f(x) \leq \lim_{x \rightarrow a} g(x) 
	\]
	\item \textbf{squeeze theorem}: if $f(x) \leq g(x) \leq h(x)$ and
	\[
	\lim_{x \rightarrow a} f(x) = \lim_{x \rightarrow a} h(x) = L
	\]
	then
	\[
	\lim_{x \rightarrow a} g(x) = L
	\]
\end{itemize}

\section{Precise Definition of Limit}
Let $f$ be a function defined on some open interval that contains the number $a$ except possibly at $a$ itself. Then we say the limit of $f$ as $x$ approaches $a$ is $L$:
\[
\lim_{x \rightarrow a} f(x) = L
\]
if for \textbf{every} number $\epsilon>0$ there is a number $\delta>0$ such that
 \[
 |x-a| < \delta \Longrightarrow |f(x) - L| < \epsilon
 \]
 Comment: range constrain leads to domain constrain.

\subsection{Precise Definition of Left Hand and Right Hand Limit}
Similar way can be applied to these $2$ definitions:
\begin{itemize}
	\item \textbf{Left-Hand Limit}:
	\[
	\lim_{x \rightarrow a^{-}} f(x) = L 
	\]
	if for every number $\epsilon>0$ there is a number $\delta>0$ such that
	\[
    a- \delta < x < a \Longrightarrow |f(x) - L| < \epsilon
	\]
	\item \textbf{Right-Hand Limit}:
	\[
	\lim_{x \rightarrow a^{+}} f(x) = L 
	\]
	if for every number $\epsilon>0$ there is a number $\delta>0$ such that
	\[
	a < x < a + \delta  \Longrightarrow |f(x) - L| < \epsilon
	\]
\end{itemize}

\subsection{Precise Definition of Infinite Limit}
Let $f$ be a function defined on some open interval that contains the number $a$ except possibly at $a$ itself. Then
\[
\lim_{x \rightarrow a} f(x) = +\infty
\]
if for \textbf{every} positive number $M$ there is $\delta>0$ such that
\[
|x-a| < \delta \Longrightarrow f(x)>M
\]
Similar definition can be applied to negative infinite limit.

\section{Continuity of Functions }

\subsection{Definition}
A function $f$ is continuous at number $a$ if
\[
\lim_{x \rightarrow a} f(x) = a
\]
this indicate $3$ conditions:
\begin{itemize}
	\item $f(a)$ is defined at $a$
	\item limit of $f$ at $a$ exists
	\item limit equals to $f(a)$
\end{itemize}
A function is continuous on an interval if it is continuous at every number in the interval. Also a function can be either continuous from left or right.

\subsection{Types of Discontinuity}
There are $3$ types:
\begin{itemize}
	\item \textbf{removable}: one point problem
	\item \textbf{infinite}: reach infinity at certain point
	\item\textbf{jump}: left limit doesn't equal to right limit
\end{itemize}

\subsection{Theorems of Continuity}
\begin{itemize}
	\item if $f$ and $g$ are continuous at $a$ and $c$ is a constant, then $f \pm g$, $cf$, $fg$, $f/g \text{, where } g\neq 0$ are also continuous at $a$
	\item polynomials are continuous on $\mathbb{R}$
	\item any rational function is continuous wherever it is defined(denominator is not zero)
	\item the following function are continuous everywhere in their domains: polynomials, rational function, root, trig, inverse trig, exponential, logarithmic functions.
	\item if $f$ is continuous at $b$ and
	\[
	\lim_{x \rightarrow a} g(x) = b
	\]
	then
	\[
	\lim_{x \rightarrow a} f(g(x)) = f \left(\lim_{x \rightarrow a} g(x)\right) = f(b)
	\]
	\item if $g$ is continuous at $a$ and $f$ is continuous at $g(a)$, then $f\circ g$ is continuous at $a$
	\item \textbf{intermediate value theorem}: suppose $f$ is continuous on interval $[a, b]$ and let $N$ be any number between $f(a)$ and $f(b)$ where $f(a) \neq f(b)$, then there exist a number $c \in (a, b)$ such that $f(c)=N$, this is the foundation for finding root of polynomial $P(x)$ in $(a, b)$ if $P(a)pP(b)<0$
\end{itemize}

\section{Limits at Infinity and Horizontal Asymptote}
The line $y = L$ is called a \textbf{horizontal asymptote} of the curve $y = f(x)$ if either
\[
\lim_{x \rightarrow \infty} = L \quad \text{or} \quad  \lim_{x \rightarrow -\infty} = L
\]
Example:
\[
\lim_{x \rightarrow -\infty} \tan^{-1}x = -\frac{\pi}{2} \quad \text{and } \quad \lim_{x \rightarrow \infty} \tan^{-1} x = \frac{\pi}{2}
\]
Here is one useful theorem: if $r>0$ is a rational number, then
\[
\lim_{x \rightarrow \infty} \frac{1}{x^r} = 0
\]
If $x^r$ is defined for all $x$, then
\[
\lim_{x \rightarrow -\infty} \frac{1}{x^r} = 0
\]
And here is the precise definition for \textbf{bound limit} as $x$ reach infinity: let $f$ be a function defined on some interval $(a, +\infty)$, then
\[
\lim_{x \rightarrow \infty} f(x) = L
\]
means that for every $\epsilon>0$, there is a corresponding $N$ such that if $x>N$, then
\[
| f(x) - L | < \epsilon
\]
similar definition works for $x$ approaches $-\infty$.  For \textbf{unbound} limit as $x$ reach infinity: let $f$ be a function defined on some interval $(a, +\infty)$, then
\[
\lim_{x \rightarrow \infty} f(x) = \infty
\]
means that for every positive number $M$ there is a corresponding number $N$ such that if $x>N$, then $f(x)>M$.

\section{Derivatives and Rate of Change}
\subsection{Tangent Line}
The tangent line to the curve $y = f(x)$ at point $P(a, f(a))$ is the line through $P$ with \textbf{slope}
\[
m = \lim_{x \rightarrow a} \frac{f(x) - f(a)}{x-a}
\]
provided that this limit exists.

\subsection{Velocity}
\begin{itemize}
	\item velocity can be considered as slope of position function
	\item acceleration can be considered as slope of velocity function
\end{itemize}

\subsection{Derivative}
The derivative of a function $f$ at a number $a$, denoted by $f'(a)$, is
\[
f'(a) = \lim_{h \rightarrow 0} \frac{f(a+h) - f(a)}{h} = \lim_{x \rightarrow a} \frac{f(x)-f(a)}{x-a}
\]
the tangent line to $y=f(x)$ at $(a, f(a))$ is the line through $(a, f(a))$ whose slope equals to $f'(a)$. And derivative can also be considered as instantaneous rate of change of a function at certain value.

\section{Derivative as Functions}
The derivative of function $f$ is defined by:
\[
f'(x) = \lim_{h \rightarrow 0} \frac{f(x+h) - f(x)}{h}
\] 
here is some common alternative notation:
\[
f'(x) = y' = \frac{dy}{dx} = \frac{df}{dx} = \frac{d}{dx}f(x) = D f(x) = D_x f(x)
\]
A function is differentiable at $a$ if $f'(a)$ exists and it is differentiable on an open interval $I$ if it is differentiable at every number in the interval.

\subsection{Differentiable vs Continuous}
If $f$ is differentiable at $a$ then it is continuous at $a$, the reverse may not be true. Meaning differentiability requires a \textbf{higher} condition than continuity.

\subsection{Function Fail to be Differentiable}
\begin{itemize}
	\item a sharp corner, $y = |x|$
	\item discontinuity, $y = 1/x$
	\item vertical tangent, $y = \sqrt[3]{x}$
\end{itemize}

\subsection{Higher Derivative}
The derivative of a function is a function, which may have its own derivative.
\begin{itemize}
	\item second derivative: $f''$
	\item third derivative: $f'''$ 
\end{itemize}
Example:
\begin{itemize}
	\item \textbf{velocity} is the (first) derivative of position function
	\item \textbf{acceleration} is the second derivative of position function
	\item \textbf{jerk} is the third derivative of position function
\end{itemize}

\section{Extra}

\subsection{Greatest Integer Functions}
It is defined as:
\[
y = f(x) =\lfloor x \rfloor
\]
\begin{itemize}
	\item piece-wise function
	\item continuous from right side  
\end{itemize}

\subsection{Triangle Inequality}
It states that:
\[
|a+b| \leq |a| + |b|
\]
\end{document}