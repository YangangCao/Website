\documentclass[10pt,a4paper,oneside]{article}
\usepackage[utf8]{inputenc}
\usepackage[T1]{fontenc}
\usepackage{amsmath}
\usepackage{amsfonts}
\usepackage{amssymb}
\usepackage{graphicx}


\author{Baboo J. Cui}
\title{Sequences and Series}
\begin{document}
\maketitle
\tableofcontents

\newpage

\section{Sequence}
A sequence is a list of numbers written in a definite order:
\[
a_1, a_2, \cdots, a_n
\]
\begin{itemize}
	\item  $n$ could be either \textbf{finite} or \textbf{infinite}
	\item $a_i$ is the $i$-th \textbf{term} of the sequence
	\item  can be defined as a function whose domain is $\mathbb{Z^+}$
	\item can be written as $\{a_n\}$ or $\{a\}_{n=1}^{+\infty}$
\end{itemize}

\subsection{Limit and Convergence}
A sequence $\{a_n\}$ has the \textbf{limit} $L$, and it is denoted as
\[
\lim_{n \rightarrow \infty} a_n = L
\]
if we can make $a_n$ as close to $L$ as taking $n$ sufficiently large. If $L$ exists, the sequence is said to be \textbf{convergent}, otherwise \textbf{divergent}.

\subsection{Precise Definition of Limit of Sequence}
A sequence $\{a_n\}$ has the \textbf{limit} $L$, and it is denoted as
\[
\lim_{n \rightarrow \infty} a_n = L
\]
if for every $\epsilon>0$, there is a corresponding integer $N$ such that if $n>N$
\[
|a_n - L| < \epsilon
\]
And 
\[
\lim_{n \rightarrow \infty} a_n = \infty
\]
indicates that for every positive number $M$, there is an integer $N$ such that if $n>N$, then $a_n > M$

\subsection{Limit of a Sequence by Function}
If
\[
\lim_{x \rightarrow \infty} f(x) = L
\]
and
\[
f(n) = a_n, \forall n \in \mathbb{Z^+}
\]
then
\[
\lim_{n \rightarrow \infty} a_n = L
\]

\subsection{Limit Laws for Sequences}
If $\{a_n\}$ and $\{b_n\}$ are convergent and $c$ is a constant:
\begin{itemize}
	\item $\lim c = c$
	\item $\lim (a \pm b) = \lim a \pm \lim b$
	\item $\lim ca = c \lim a$
	\item $\lim (ab) = \lim a \lim b$
	\item $\lim (a/b) = \lim a / \lim b$
	\item $\lim a^p = (\lim a)^p$
\end{itemize}

\subsection{Squeeze Theorem}
If $a_n \leq b_n \leq c_n$, for $n>n_0$ and
\[
\lim_{n \rightarrow \infty} a_n = \lim_{n \rightarrow \infty} c_n = L 
\]
then
\[
\lim_{n \rightarrow \infty} b_n = L
\]

\subsection{Absolute Convergence}
If
\[
\lim_{n \rightarrow \infty} |a_n| = 0
\]
then
\[
\lim_{n \rightarrow \infty} a_n = 0
\]

\subsection{Convergence of a Sequence Applied to Function}
If
\[
\lim_{n \rightarrow \infty} a_n = L
\]
and a function $f$ is continuous at $L$, then
\[
\lim_{n \rightarrow \infty} f(a_n) = f(L)
\]

\subsection{title}
\subsection{title}
\subsection{title}


\section{Extra}
\subsection{Fibonacci Sequence}
It is defined recursively by the conditions:
\[
a_1 = 1 \qquad a_2 =1 \qquad a_n = a_{n-1} + a_{n-2}, n\geq3
\]
\end{document}