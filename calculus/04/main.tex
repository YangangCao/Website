\documentclass[10pt,a4paper,oneside]{article}
\usepackage[utf8]{inputenc}
\usepackage[T1]{fontenc}
\usepackage{amsmath}
\usepackage{amsfonts}
\usepackage{amssymb}
\usepackage{graphicx}

\usepackage{tikz}
\usepackage{xcolor}
\usepackage{eso-pic}
\newcommand{\watermark}[3]{\AddToShipoutPictureBG{
		\parbox[b][\paperheight]{\paperwidth}{
			\vfill%
			\centering%
			\tikz[remember picture, overlay]%
			\node [rotate = #1, scale = #2] at (current page.center)%
			{\textcolor{gray!70!cyan!40!red!20}{#3}};
			\vfill}}}

\author{Yangang Cao}
\date{August 2, 2019}
\title{Integrals}
\begin{document}
%\watermark{50}{9}{www.baboocui.club}
\maketitle
\tableofcontents

\newpage
\section{The Definite Integral}
When we compute an area, a limit of form is arised
\[
\lim\limits_{n\rightarrow\infty}\sum_{i=1}^{n}f(x^*_i)\Delta x=\lim\limits_{n\rightarrow\infty}[f(x^*_1)\Delta x+f(x^*_2)\Delta x+\cdots+f(x^*_i)\Delta x]
\]
If $f$ is a function defined for $a\geqslant x\geqslant b$, we divide the interval $[a,b]$ into $n$ subintervals of equal width $\Delta x = \frac{b-a}{n}$. We let $x_0(=a),x_1,x_2,...,x_n(=b)$ be the endpoints of thses subintervals and we let $x_1^*,x_2^*,...,x_n^*$ be any {\bfseries sample points} in these subintervals, so $x_i^*$ lies in the $i$th subinterval $[x_{i-1},x_i]$. Then the {\bfseries definite integral of $f$ from $a$ to $b$} is
\[
\int_{a}^{b}f(x)dx=\lim\limits_{n\rightarrow\infty}\sum_{i=1}^{n}f(x^*_i)\Delta x
\]
provided that this limit exists and gives the same value for all possible choices of sample points. If it does exist, we say that $f$ is {\bfseries integrable} on $[a,b]$.\\
\\If $f$ is continuous on $[a,b]$, or if $f$ has only a finite number of jump discontinuities, then $f$ is intergrable on $[a,b]$; that is, the definite integral $\int_{a}^{b}f(x)dx$ exists.
\section{Properties of the Definite Integral}
\subsection{Properties of the Integral}
\[
\int_{a}^{b}[f(x)+g(x)]dx=\int_{a}^{b}f(x)dx+\int_{a}^{b}g(x)dx
\]
\[
\int_{a}^{b}[f(x)-g(x)]dx=\int_{a}^{b}f(x)dx-\int_{a}^{b}g(x)dx
\]
If we reverse $a$ and $b$
\[
\int_{a}^{b}f(x)dx=-\int_{b}^{a}f(x)dx
\]
If $a=b$
\[
\int_{a}^{b}f(x)dx=\int_{a}^{a}f(x)dx=0
\]
If $c$ is any constant number
\[
\int_{a}^{b}cdx=c(b-a)
\]
\[
\int_{a}^{b}cf(x)dx=c\int_{a}^{b}f(x)dx
\]
\[
\int_{a}^{c}f(x)dx+\int_{c}^{b}f(x)dx=\int_{a}^{b}f(x)dx
\]
\subsection{Comparison Properties of the Integral}
If $f(x)\geqslant0$ for $a\leqslant x\leqslant b$
\[
\int_{a}^{b}f(x)dx\geqslant 0
\]
If $f(x)\geqslant g(x)$ for $a\leqslant x\leqslant b$
\[
\int_{a}^{b}f(x)dx\geqslant\int_{a}^{b}g(x)dx
\]
If $m\leqslant f(x)\leqslant M$ for $a\leqslant x\leqslant b$
\[
m(b-a)\leqslant\int_{a}^{b}f(x)dx\leqslant M(b-a)
\]
\section{The Fundamental Theorem of Calculus}
If $f$ is continuous on $[a,b]$, then the function $g$ defined by
\[
g(x)=\int_{a}^{x}f(t)dt\quad a\leqslant x\leqslant b
\]
is continuous on $[a,b]$ and differentiable on $(a,b)$, and $g'(x)=f(x)$.\\
\\If $f$ is continuous on $[a,b]$, then 
\[
\int_{a}^{b}f(x)dx=F(b)-F(a)
\]
where $F$ is any antiderivative of $f$, that is, a function such that $F'=f$.
\section{Indefinite Integrals and the Net Change Theorem}
\subsection{Indefinite Integrals}
We need a convenient notation for antiderivative that makes them easy to work with, the notation $\int f(x)dx$ is traditionally used for an antiderivative of $f$ and is called an {\bfseries indefinite Integral}. Thus
\[
\int f(x)dx=F(x)\quad \text{means}\quad F'(x)=f(x)
\]
and
\[
\int_{a}^{b} f(x) d x=\int f(x) d x ]_{a}^{b}
\]
\subsection{Table of Indefinite Integrals}
\[
\int c f(x) d x=c \int f(x) d x \quad \int[f(x)+g(x)] d x=\int f(x) d x+\int g(x) d x
\]
\[
\int k d x=k x+C
\]
\[
\int x^{n} d x=\frac{x^{n+1}}{n+1}+C \quad(n \neq-1) \quad \int \frac{1}{x} d x=\ln |x|+C
\]
\[
\int e^{x} d x=e^{x}+C \quad \int a^{x} d x=\frac{a^{x}}{\ln a}+C
\]
\[
\int \sin x d x=-\cos x+C \quad \int \cos x d x=\sin x+C
\]
\[
\int \sec ^{2} x d x=\tan x+C \quad \int \csc ^{2} x d x=-\cot x+C
\]
\[
\int \sec x \tan x d x=\sec x+C \quad \int \csc x \cot x d x=-\csc x+C
\]
\[
\int \frac{1}{x^{2}+1} d x=\tan ^{-1} x+C \quad \int \frac{1}{\sqrt{1-x^{2}}} d x=\sin ^{-1} x+C
\]
\[
\int \sinh x d x=\cosh x+C \quad \int \cosh x d x=\sinh x+C
\]
\[
\int \tan x d x=\ln|\sec x|+C
\]

\section{The Substitution Rule}
\subsection{The Substitution Rule for Definite Integrals}
If $g'$ is continuous on $[a,b]$ and $f$ is continuous on the range of $u=g(x)$, then
\[
\int_{a}^{b}f(g(x))g'(x)dx=\int_{g(a)}^{g(b)}f(u)du
\]
\subsection{Integrals of Symmetry Functions}
Suppose $f$ is continuous on $[-a,a]$.\\
If $f$ is even $[f(-x)=f(x)]$, then 
\[
\int_{-a}^{a}f(x)dx=2\int_{0}^{a}f(x)dx
\]
If $f$ is odd $[f(-x)=-f(x)]$, then
\[
\int_{-a}^{a}f(x)dx=0
\]

\end{document}