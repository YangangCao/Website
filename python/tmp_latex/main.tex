\documentclass[10pt,a4paper,oneside]{article}
\usepackage[utf8]{inputenc}
\usepackage[T1]{fontenc}
\usepackage{amsmath}
\usepackage{amsfonts}
\usepackage{amssymb}
\usepackage{graphicx}
\usepackage{enumerate}
\date{June 20, 2019}
\author{Baboo J. Cui, Yangang Cao}
\title{Python Q\&A}
\begin{document}
\maketitle
\tableofcontents

\newpage

\section{Python Itself}
\begin{enumerate}[1.]
\item How to install python?\\
Use conda directly or use brew etc
\item How to run a python code in cosole?\\
\$ python file\_name.py
\item Types of python execution mode?\\
Interactive mode ($>>>$) and command line mode (directly run python file by interpretor)
\item Strong type vs weak typrs?\\
If type needs to be specify when declaring the var
\item Python3 default coding?\\
UTF-8 for code and unicode in memory
\item Legal variable naming?\\
Letter, number and underscore, can't started with a number
\item How to declare a constant?\\
Use all capital letters, still fake constant
\item Camel-case vs underscore naming?\\
Just choose the one you like
\item Basic I/O control?\\
2 func, input() to get str and print() for output
\item What is duck typing?\\
Way of object inference, it is a type if it looks like it
\end{enumerate}
\section{Pythob Basics}
\begin{enumerate}[1.]
\item``//''  operator?\\
For getting the whole part of the division
\item ``\%'' operator?\\
For getting the residue
\item How to calculate a to the power of b?\\
2 ways: a**b or pow(a, b), tip pow() is built-in
\item Multiplication between int and str?\\
Just repeat the str for int times
\item Type coercion?\\
Type(o), change o to Type, may cause error
\item Type of value from input()?\\
Must be str, even number from this way will be str
\item How to get the type of an object?\\
Function type() will return the type
\item How to get keywords of python?\\
$>>>$ from keyword import kwlist, and print it
\item How many kw in python?\\
33
\item While else structure?\\
Else part will be execute if break statement in while is not executed
\item Which is faster “while 1” or “while True”?\\
“while 1” because bool is a subclass of int, similarly “if x” is faster than “if x=True”
\item How bool can be coerced to int?\\
True to 1 and False to 0
\item How number can be coerced to bool?\\
0 to False and others are True
\item Logic AND OR and XOR?\\
Corresponds to \&, | and \^\ , in terms of bit
\item Python print() format control?\\
Placeholder: \%, string: \%s, number: \%d, \%f, \%x(hex), string canonical: \%r, tip r is for repr. Then follow \%(var1, var2…)
\item Python print() format control with format()?\\
Use \{\} as place holder and follow .format(var1, var2…)
\item Common coding?\\
UTF-8, ASCII, GBK, unicode
\item Relationship between bit and Byte\\
1 Byte = 8 bit, bit is the smallest unit in computer
\item x or y?\\
If x is true, return x, otherwise return y
\item x and y?\\
If x is true, return y, otherwise return x, easy for bool var, may be complicate for numbers
\item Priority of NOT, AND, OR?\\
NOT > AND > OR
\item How many bits are there if decimal -> binary?\\
Use int.bit\_length() to get
\item How str can be coerced to bool?\\
Empty str to false and others to True
\item How bool can be coerced to str?\\
True to `True' and False to `False'
\item Does comma has effects in print()?\\
There will be a white space, print(``a'', ``b'') -> ``a b'', there is a comma
\item Usual input and output function of python?\\
I: input(), O: print()
\item Is python case-sensitive?\\
Yes
\item How indent is achieved in python?\\
4 white space, not TAB! And don't mix them together
\item How to comment code in python?\\
Multiline: ``````xxx'''''', single line: \# xxx
\item How to build up a block?\\
 Start with colon (:) and block code with indent
 \item Data type that python can directly deal with?\\
 int, float, str, bool, None(not 0), list, dict, etc
 \item Notion of number in binary, oct and hex?\\
 0b, 0o and 0x
 \item Essential of assignment to a var?\\
 Let the var point to a certain value(be clear about the mechanism)
 \item Coding of str in python?\\
 Unicode in memory, note that python code is in utf-8 for storage
 \item Size of ASCII?\\
 I Byte = 8bit, 256 cases
 \item Size of Unicode?\\
 2 Bytes = 16bit, 65536 cases
 \item Convert char to unicode number?\\
 ord()
 \item Convert unicode number to char?\\
 chr()
 \item How to declare str in byte?\\
 Prefix:  b`xxx', each one occupy 1 byte
 \item How to get size of an object in memory?\\
 Use sys.getsizeof() function, note this will return the size of an object, which may be quite big! Str is very complicate
 \item How to use python to run bash command?\\
 Use os.system(“COMMAND”)
 \item How to decoding a binary to string?\\
 Use str.decode() method, arg could be ``ascii'', ``utf-8'', ``gb2312'', to ignore errors, add errors=`ignore'
 \item How to encode a string to binary?\\
 Use str.encode() method
 \item How to format a decimal?\\
 Use \{num:a.b f\} where num is the position, a is num of integer bit, b is for decimal, f represent float
 \item How to find length of a list?\\
 Use len() function, len(LIST) will return
 \item How to find the last element of a list?\\
 Use LIST[-1], index of -1 represent the last element, -2 for the second last one, etc
 \item Difference between list and tuple?
 List is mutable, declared by [ ] and tuple is immutable and declared by ()
 \item List.append(o)?\\
 Append obj o to the end of list
 \item List.insert(idx, o)?\\
 Insert obj o to index idx
 \item List.pop() or list.pop(i)?\\
 Pop out the last element, and pop(i) pop the element with index i
 \item Acquire the idx element of list?\\
 List[idx]
\item How to init a tuple with one element?\\
T = (obj1, ), the key is to add a comma after obj1
\item How to understand immutability of tuple?\\
Each element that points to won't be mutable, but the content that each points may be mutable
\item Condition control structure?\\
If else, or if elif else
\item How to loop an iterable object by for?\\
Use for loop, for item in iterable\_obj
\item How to loop by while?\\
Use while structure: while condition: block\_exe
\item How to generate a range?\\
Use range() function, range(a) will generate [0\dots a-1] and range(a, b) will generate [a, a+1\dots b-1]
\item Difference between break and continue?\\
Break will stop the loop and continue will simply jump to next loop, usually both are related to if condition
\item How to terminate a python program directly?\\
Use ctrl + c to kill the process
\item What is dict?\\
A key-value mechanism, also known as map in other languages
\item How to declare a dict?\\
Format: \{key1: val1, key2: val2, \dots\}, curly braces, comma separation, colon separation, key must be immutable
\item Why dict can find a key value so fast?\\
Use binary tree mechanism, which lead to log(n) time scale
\item How to determine if a dict has a key?\\
Use code: if key\_val in dict, or: dict.get(key\_val)
\item How to delete a key-value in dict?\\
Use code: dict.pop(key)
\item Trade off between list and dict?\\
List is slower but occupy smaller memory, dict is faster but requires more memory
\item How to create a set?\\
Use curly braces:\{item1, item2, \dots\} or use code set(list), which will turn a list to set
\item Properties of set?\\
It has no order and there is no repeats
\item Add and remove key of a set?\\
Corresponding to set.add(key) and set.remove(key)
\item Common operation on sets?\\
Union ``|'' and intersection ``\&''
\item Difference between dict and set?\\
Share the same mechanism but set doesn’t have value, can't put mutable obj in
\item Is str mutable? What about list?\\
Str is immutable and list is mutable
\item Immutable objects?\\
Number, string and tuple. Contents in certain add can't be changed
\item How to find the memory address of an object?\\
Use function id(), id(o) will return the memory add
\item How to get console args?\\
By import sys, and then use sys.argv[idx]
\item What is configurative programming?\\
A framework is created such that coding is like setting configuration
\item Mechanism of importing a module? \\
Python interpreter just go through line by line(interview question)
\end{enumerate}
\section{Functions}
\begin{enumerate}[1.]
\item How to define a function?\\
Easy\dots
\item How to specify the return type of a function?\\
Use syntax def FUN\_NAME () -> RETURN\_TYPE:
\item Disadvantages of declaration like def func(a=[1,2])?\\
Args are mutable, may cause unpredicted errors!
\end{enumerate}
\section{OOP-Object Oriented Programming}
\begin{enumerate}[1.]
\item What is OOP?\\
Take object as the basic unit for programming, OOP vs procedure-oriented-programming(set of instruction)
\item Why OOP?\\
Everything are objects(a set of objects) and execution becomes the interaction between instances
\item What does an object include?\\
2 parts, properties(data) and methods(functions)
\item Class vs instance?\\
Class is template, instance is the specified class, instance is based on class
\item 3 characteristics of OOP?\\
Encapsulation, inheritance, polymorphism
\item How to add properties to an instance dynamically?\\
Use form instance.var\_name = val to dynamically add a property, this won't work for other instances
\item How to add method to class dynamically?\\
Use form class.method\_name = func\_name, this may arise warning from the IDE
\item What is \_\_slots\_\_ for?\\
A var in class which control the possible property names
\item What is the range of \_\_slots\_\_?\\
Only work for current class, won't work for subclass
\item What is the type of \_\_slots\_\_?\\
It is a tuple
\item What if both a class and its base have \_\_slots\_\_?\\
The possible property will be the union of the two \_\_slots\_\_
\item What is @property for?\\
A decorator which may help us to take a method like property
\item How many ways are there to define class methods?\\
3 ways, regular definition(related to self), decorated by @classmethod(related to cls) and @staticmethod
\item Difference between self and cls?\\
Self is bound to instance of class and cls is bound to class
\item Call of a regular method in a class?\\
Can be called by object but not class, or by class with first arg to be an instance of that class
\item Call of a class method in a class?\\
Can be called by both instance and class directly
\item Call of a static method in a class?\\
Can be called by both instance and class directly
\item Why static method in class instead of an independent func?\\
To indicate that the method belongs to the class and by inheritance, code can be managed better
\item What is MRO?\\
Method resolve order, a mechanism for inheritance
\end{enumerate}
\section{Metaclass}
\begin{enumerate}[1.]
\item Biggest difference between static and dynamic language?\\
Static: definition is done during compiling process. Dynamic: definition are created during runtime
\item How class can be defined?\\
Two ways, by general declaration and by type() method
\item How to define a class by type()?\\
Use form CLASS\_NAME = type(`NAME', (BASE\_CLASS,), dic(METHOD1=FUNC1,\dots)), dynamic way to define a class
\item What does type() do?\\
It can show the type of an object. It also can be used to define a class
\item Difference between \_\_new\_\_() and \_\_init\_\_()?\\
\_\_new\_\_() create the obj and \_\_init\_\_() initialize the obj. Initialization comes after creation
\item How class is created(not asking how defined)?\\
During running time, essentially by function type(), tip: not from declaration!!!
\item Difference between general class definition and type()?\\
First one is in static way, the second one can be used in dynamic process, essentially, both share the same purpose
\item Dynamically create a class by static or dynamic language?\\
Easier for dynamic language(itself support), static language requires constructing the source code in the beginning.
\item What is the type of a class?\\
All class name itself has type: ``type''
\item Relationship between type and object?\\
Type is subclass of object, metaclass of object is type. Both are created during the execution of interpreter
\item What is metaclass?\\
An class that controls how another class is defined, can be considered as template for other classes
\item What is the parent class of metaclass?\\
Must be type, cannot be object
\item How class derived from metaclass is created?\\
By calling type.\_\_new\_\_(mcs, name, bases, attrs), it is the return of \_\_new\_\_() function in the metaclass!
\item How metaclass is defined?\\
Name end with Metaclass by convention(not necessary), inherit from type, define \_\_new\_\_() to control how other classes are created
\item What type is name in \_\_new\_\_(mcs, name, bases, attrs)?\\
It is str, who has value of the name of the class that take it as template
\item What type is bases in \_\_new\_\_(mcs, name, bases, attrs)?\\
It is tuple, a tuple that contains the parent classes in the target class
\item What type is attrs in \_\_new\_\_(mcs, name, bases, attrs)?\\
It is dict, has form var\_name: value, var could be either function or properties
\item How are args in \_\_new\_\_(mcs, name, bases, attrs) passed?\\
When interpreter reading a class, it will use type() to create a class, and args are passed in the conventional way
\item Does any class has a corresponding metaclass?\\
Yes, and the metaclass is usually implicitly inherited
\item What is abstract class?(Not that important in python)\\
Class abstraction from many classes with certain similarities, it has a higher abstraction, a template for other classes
\item How to declare an abstract class in python?\\
First from abc import abstractmethod, ABCMeta, then declare a class with arg metaclass=ABCMeta and decorate method with @abstractmethod
\item Characteristics of abstract class?\\
Methods only have declaration, no implementation. Cannot be instantiated. Must have abstract method and must be overridden latter.
\item What is interface class?\\
Like header, can’t be instantiated, only contains method declaration, contains methods, properties, event etc\dots Doesn’t contain constants etc\dots
\end{enumerate}
\section{Enum}
\begin{enumerate}[1.]
\item What is enum class?\\
A enumeration, just list everything, like a key value system linked by equality, usually for constants
\item How to create a enum class?\\
First from enum import Enum, and then create a class inherit from Enum
\item How to get key of enum?\\
Directly use dot ``.'' or by enum\_name[`KEY\_VAL']
\item How to get key value in a enum?\\
By enum.KEY\_VAL.value
\item What is @unique for?\\
Make sure that both key and value won't repeat! (bi-jective)
\item How to import unique?\\
Use statement from enum import unique
\item What is enum generally for?\\
For finding key by value
\end{enumerate}
\section{Error, Debug and Test}
\begin{enumerate}[1.]
\item What is bug?\\
Any unexpected thing, bug must be repaired
\item What causes bug?\\
Programming error, wrong input, unexpected condition(disk is full\dots)
\item What is python pdb?\\
A way of debug, python debug
\item What is error code(value)?\\
When something go wrong, there will be a specific return value like return -1
\item Disadvantages of error code?\\
Mix the error code and general return value together
\item Try, except, else, finally?\\
Try to execute what’s in try, if any error, jump to except and else part will be execute if no error in try, then goes to finally(optional)
\item What if there might be more than a type of error?\\
One try can contain more than one expect block
\item How to write except part?\\
Write form except: or except ERROR\_TYPE as e:
\item Except ZeroDivisionError as e:, what type are they?\\
ZeroDivisionError has type: type since it is a class, e has type:<class `ZeroDivisionError'> since it is an instance of the error class
\item What is the base class of all error classes?\\
All are inherited from BaseException
\item Range of try\dots except works?\\
Function contains it and any outer part that contains them, catch the error at nice position will be okay, don't need to put it everywhere
\item What if error does't caught by any except?\\
It will be thrown upper until caught by Python interpreter to print error and stop the program
\item How to read Traceback (most recent call last):\\
It shows that there are error and goes from top to bottom, the last line show the real reason
\item How to record error?\\
Import logging and add logging.exception(e) to output error and finish the program
\item What essentially an error is?\\
It is an instance of a class
\item How to throw an instance of error?\\
Use raise, need to raise an instance of class that is well designed that inherit from some error class
\item What could raise do if put into the block of except?\\
Convert one error to other type, should be logically reasonable
\item Easiest way of debugging?\\
Use print(), have to delete it after debugging\dots sad
\item Syntax of assert?\\
Have form assert CONDITION, ERROR\_MESSAGE,  condition should be true, otherwise will be error
\item How to stop assert statement during execution?\\
Execute .py file with form \$ python -O xxx.py
\item How to log info?\\
Use form logging.info(STRING)
\item How to config the level of logging?\\
Add code logging.basicConfig(level=logging.INFO) after import logging
\item Level of logging?\\
There are 4: debug, info, warning, error, the higher level you set, the lower cases will be ignored
\item How to start a program by pdb?\\
Use form \$ python -m pdb xxx.py
\item How to see code in pdb?\\
Use command: ``1''
\item How to execute code one line by another?\\
Use command: ``n''
\item How to get value of a var in pdb?\\
Use command: ``p VAR\_NAME''
\item How to quit pdb?\\
Use command: ``q''
\item How to use pdb.set\_trace()?\\
Put it at position where might have error, it will pause the program there , use ``p Var'' to debug and press ``c'' to continue
\item What is the best way of debugging?\\
Ultimately\dots logging
\item What is TDD?\\
Short for Test-Driven-Development
\item What is unit test?\\
Check if a module, function or class work correctly, put all test conditions in a module, after revision, check if all conditions could pass the test
\item Advantage of unit test?\\
It can almost guarantee that the behavior of code is correct
\item What does a test unit class inherit from?\\
From unites.TestCase
\item Purpose of methods in unit test?\\
By convention, methods start with ``test'' are test methods, otherwise not test methods which won't be executed during test
\item What does assertEqual() do?\\
Has form self.assertEqual(abs(-1), 1), check if expected output equals to target output
\item What does assertRaises() do?
Has form with self.assertRaises(ERROR\_TYPE): BLOCK, if do anything in BLOCK, there will be ERROR\_TYPE will be thrown
\item How to run unit test by coding?\\
Directly use statement: unites.main()
\item How to run unit test in console?\\
By command: python -m unittest xxx.py
\item What are setUp() and tearDown() functionality?\\
They will be executed before and after a test
\item Which module is used for doc test?\\
Use import doctest
\item How to run doc test?\\
Use statement doctest.testmod()
\item Where the doc test code should be?\\
Within the triple comments: ``````xxx''''''
\item How to write test doc?\\
All statement start with $>>>$ STATEMENT(could be more than one) and then the next line follow the output
\end{enumerate}
\section{Others}
\subsection{Virtual Environment}
\begin{enumerate}[1.]
\item What is virtual environment for?\\
To build isolated environment for different programs. Different programs may depend on different python version and packages
\item What command is used to build virtual env?\\
Use virtualenv
\item How to install virtualenv?\\
Use command \$ pip3 install virtualenv by default,  or \$ conda install virtualenv(not recommend for conda, use create)
\item How to build up a virtual env?\\
Use command \$ virtualenv —no-site-packages ENV\_NAME, where the ENV\_NAME will generate a new folder that contains everything
\item What is no-site-packages for?\\
Avoid the copy of the third party package, make the env very clean
\item How to activate virenv?\\
Use command \$ source ENV\_PATH/bin/activate to activate
\item Where the packages will be installed in virenv?\\
At path: ./lib/pythonx.x/site-packages/
\item How to deactivate a virtual env?\\
Use command \$ deactivate
\item How to show all installed packages?\\
Use command \$ pip list
\item How to check packages installed under conda?\\
Use command \$ conda list
\item How to check existing env?\\
Use command \$ conda env list or \$ conda info -e
\item How to check info of conda?\\
Use command \$ conda info
\item How to create virtual env by conda?\\
Use command \$ conda create -n ENV\_NAME python=3.6, python version must be added!
\item How to activate virenv by conda?\\
Use command \$ conda activate ENV\_NAME
\item How to quit virenv by conda?\\
Use command \$ conda deactivate
\item How to delete virenv by conda?\\
Use command \$ conda remove -n ENV\_NAME —all (2 - before all)
\item How to install a package?\\
Use command \$ conda install PACK\_NAME
\item How to delete a package?\\
Use command \$ conda remove PACK\_NAME, this is intent to delete package under that environment 
\item How to update conda?\\
Use command \$ conda update -n base -c defaults conda
\item How to install requirement.txt?\\
Use command \$ pip install -r requirements.txt or by \$ conda install -file requirements.txt
\item How to generate requirement.txt for virenv?\\
Use command \$ pip freeze > requirement.txt
\item How to add Tsinghua source?\\
Use command \$ conda config - -add channels\\ https://mirrors.tuna.tsinghua.edu.cn/anaconda/pkgs/free/\\
\$ conda config - -add channels \\https://mirrors.tuna.tsinghua.edu.cn/anaconda/pkgs/main/\\
\$ conda config - -set show\_channel\_urls yes
\item How to reset source?\\
Use command \$ conda config - -remove-key channels
\end{enumerate}
\subsection{GUI}
\begin{enumerate}[1.]
\item Any packages for GUI?\\
Tk, wxWidgets, QT, GTK\dots
\item Advantage of TK?\\
It can be used directly
\item Which language is TK based on?\\
Based on TCL
\item How to import TK?\\
Add statement from tkinter import *, MUST be *
\item Which class is Application inherit from?\\
Class Frame
\item What args should \_\_init\_\_() of Application contains?\\
Two args: self and master=None
\item What is done in \_\_init\_\_()?\\
3 things: Frame.\_\_init\_\_(self, master), self.pack(), self.createWidgets(), (declaration and pack)
\item What is widget?\\
Any GUI object in TK is known as widget
\item What is pack() for?\\
Add widget into GUI container and achieve layout, after declare a component, pack() function is mandatory
\item Functions for layout?\\
2 functions, pack() for easy layout and grid() for more complex ones
\item What is label API?\\
Use statement Label(self, text=``TEXT''), and then pack()
\item What is button API?\\
Use statement Button(self, text=``TEXT'', command=self.COMMAND), and then pack()
\item What is entry API?\\
Use statement Entry(self), and then pack()
\item How to use messagebox?\\
First import tkinter.messagebox as messagebox, and follow messagebox.showinfo(`TITLE', `MESS\_DISP')
\item How to instantiate an application?\\
3 steps, create instance, set instance.master.title(``TEXT'') and start instance.mainloop()
\end{enumerate}
\section{PyCharm}
\begin{enumerate}[1.]
\item How to search?\\
Use shortcut cmd+O, then easy
\item Meaning of set a folder as resources root?\\
For searching file, will add the path to search path
\end{enumerate}
\end{document}