\documentclass[10pt,a4paper,oneside]{article}
\usepackage[utf8]{inputenc}
\usepackage[T1]{fontenc}
\usepackage{amsmath}
\usepackage{amsfonts}
\usepackage{amssymb}
\usepackage{graphicx}
\usepackage{pythonhighlight}


\author{Baboo J. Cui}
\title{Builtin}

\begin{document}
\maketitle
\tableofcontents

\newpage

\section{Basic Properties of Python}
\begin{itemize}
	\item code \textbf{block} is identified by $4$ white space(not tab) following a line end up with "$:$"
	\item \textbf{comments} are start with "$\#$", for multi-line case, use triple comma
	\item python is \textbf{case sensitive}
\end{itemize}
Here is a list of size issue:
\begin{itemize}
	\item 1 Byte = 8 bit, bit is the smallest unit in computer
	\item size of ASCII is 1 Byte = 8bit, 256 cases
	\item size of Unicode is 2 Bytes = 16bit, 65536 cases
	\item to get size of an object in memory, use \pyth{sys.getsizeof(o)} function, note this will return the size of an object, which may be quite big
\end{itemize}

\section{Keywords}
There are $33$ keywords in Python.
\begin{python}
	from keyword import kwlist
	print(len(kwlist))  # 33
	print(kwlist)       
\end{python}
keyword list:
\begin{itemize}
	\item True, False, not, or, and (5)
	\item def, pass, return, yield, lambda (5)
	\item if, else, elif, while, for, in, finally, continue, break (9)
	\item from, import, as (3)
	\item try, except, raise, assert (4)
	\item global, nonlocal (2)
	\item class, None (2)
	\item with, del, is (3)
\end{itemize}

\section{Variable}
Variables in python may be any type, naming a var should follow that:
\begin{itemize}
	\item combination of letter, number and "\_"
	\item cannot start with a number
\end{itemize}
Value can be assigned to a variable by "=" operator. Assignment do two things:
\begin{itemize}
	\item create value in memory
	\item create variable name in memory and point it to corresponding value
\end{itemize}
To declare a constant, name the var with all capital letters(fake constant, just for reading). There are two types of methods for naming variables:
\begin{itemize}
	\item camel-case
	\item underscore naming
\end{itemize}  
just choose the one you like.


\section{Operations}
Python offers lots of basic operations:
\begin{itemize}
	\item basic ones: "+, -, *, /"
	\item exponential: "**", or use \textbf{pow()} function
	\item floor division: "//", also known as whole part division
	\item reminder division: "\%", to get the residue
\end{itemize}

\section{Type}
A type is a kind of data structure in python. To find the type of an object, simply use \pyth{type(o)}.

\subsection{Duck Type}
Python is weak type language. If an object looks like a type, then methods for that type can be applied to it. This is one of the most important feather in Python.

\subsection{Type Coercion}
Force to convert any type to a certain type, this may cause error and some complicate problem.
\begin{python}
	print(int("123"))  # 123
	print(int(12.34))  # round off error 12
	# print(int("abc123"))  # error
\end{python}
here is a list of important type coercion
\begin{itemize}
	\item Boolean to int: True to 1 and False to 0
	\item int to Boolean: 0 to False and the others to True
	\item str to Boolean: empty str to False and non-empty str to True
	\item Boolean to str: True to "True" and False to "False"
\end{itemize}

\subsection{Integer}
it has type \pyth{int}:
\begin{itemize}
	\item \textbf{infinite} range(different from others), should avoid too \textbf{big} number
	\item \textbf{hex} notation: add \pyth{0x} prefix, or use \pyth{hex()} to convert
	\item \textbf{binary} notation: add \pyth{0b} prefix, or use \pyth{bin()} to convert
	\item \textbf{oct} notation: add \pyth{0o} prefix, or use \pyth{oct()} to convert
	\item has method \pyth{bit_length()} that return the bit length of an int
\end{itemize}

\subsection{Float}
it has type \pyth{float}:
\begin{itemize}
	\item \textbf{round off error} may occur
	\item \textbf{direct} notation: simply write the value
	\item \textbf{scientific} notation: \textbf{mantissa} exp \textbf{order of magnitude}
\end{itemize}

\subsection{String}
it has type \pyth{str}:
\begin{itemize}
	\item declare by putting string literal in single or double quotes
	\item escape character: begin with $\backslash$(backslash)
	\begin{python}
		# commonly used
		"\n"  # return
		"\t"  # table
	\end{python}
	\item coding: python code is in \textbf{UTF-8}, data in memory is in \textbf{unicode}
	\item direct string(simply what's inside): \pyth{r"..."} 
	\item \textit{binary string}(has type \textbf{bytes}, it's ASCII code,NOT str): \pyth{b"..."}, every letter is 1 byte
\end{itemize}

\subsubsection{Coding Functions}
Unicode is used for coding in python 3:
\begin{itemize}
	\item \pyth{ord()}: convert string to Unicode
	\item \pyth{chr()}: convert Unicode to string
\end{itemize}

\subsubsection{Encode and Decode}
To covert a string  literal to a certain code is called encoding, and the reverse process is call decoding. The common encoding methods are:
\begin{itemize}
	\item "ascii"
	\item "utf-8"
	\item "gb2312"
	\item tip: for Unicode, use \textbf{ord()}
\end{itemize}
\begin{python}
	str.encoding()  # encoding a string to binary
	bytes.decoding()  # decode binary to string, has parameter errors="ignore" to avoid error
\end{python}

\subsubsection{Multiplication Between int and str}
This will repeat the \textit{str} for \textit{int} times:
\begin{python}
	print("abc"*3)  # abcabcabc
\end{python}


\subsection{Boolean}
It has type \pyth{bool}, it has only two value: \textbf{True} and \textbf{False}(case sensitive). Operator on Boolean: 
\begin{itemize}
	\item \textbf{and}(\&)
	\item \textbf{or}(|)
	\item \textbf{not}(\^)
\end{itemize}
operation properties:
\begin{itemize}
	\item priority order: not > and > or
	\item x or y: if x is true, return x, otherwise return y
	\item x and y: If x is true, return y, otherwise return x
\end{itemize}

\subsection{None}
None is a special value in Python to indicate empty.

\subsection{List}

\subsection{Tuple}

\section{Control Flow}

\subsection{If}

\subsection{While}
in while-else structure, else part will be execute if break statement in while is not executed.
"while 1" is faster than "while True" because bool is a subclass of int, similarly "if x" is faster than "if x=True"

%---------
\begin{enumerate}

	\item How to use python to run bash command?\\
	Use os.system(“COMMAND”)


	\item How to format a decimal?\\
	Use \{num:a.b f\} where num is the position, a is num of integer bit, b is for decimal, f represent float
	\item How to find length of a list?\\
	Use len() function, len(LIST) will return
	\item How to find the last element of a list?\\
	Use LIST[-1], index of -1 represent the last element, -2 for the second last one, etc
	\item Difference between list and tuple?
	List is mutable, declared by [ ] and tuple is immutable and declared by ()
	\item List.append(o)?\\
	Append obj o to the end of list
	\item List.insert(idx, o)?\\
	Insert obj o to index idx
	\item List.pop() or list.pop(i)?\\
	Pop out the last element, and pop(i) pop the element with index i
	\item Acquire the idx element of list?\\
	List[idx]
	\item How to init a tuple with one element?\\
	T = (obj1, ), the key is to add a comma after obj1
	\item How to understand immutability of tuple?\\
	Each element that points to won't be mutable, but the content that each points may be mutable
	\item Condition control structure?\\
	If else, or if elif else
	\item How to loop an iterable object by for?\\
	Use for loop, for item in iterable\_obj
	\item How to loop by while?\\
	Use while structure: while condition: block\_exe
	\item How to generate a range?\\
	Use range() function, range(a) will generate [0\dots a-1] and range(a, b) will generate [a, a+1\dots b-1]
	\item Difference between break and continue?\\
	Break will stop the loop and continue will simply jump to next loop, usually both are related to if condition
	\item How to terminate a python program directly?\\
	Use ctrl + c to kill the process
	\item What is dict?\\
	A key-value mechanism, also known as map in other languages
	\item How to declare a dict?\\
	Format: \{key1: val1, key2: val2, \dots\}, curly braces, comma separation, colon separation, key must be immutable
	\item Why dict can find a key value so fast?\\
	Use binary tree mechanism, which lead to log(n) time scale
	\item How to determine if a dict has a key?\\
	Use code: if key\_val in dict, or: dict.get(key\_val)
	\item How to delete a key-value in dict?\\
	Use code: dict.pop(key)
	\item Trade off between list and dict?\\
	List is slower but occupy smaller memory, dict is faster but requires more memory
	\item How to create a set?\\
	Use curly braces:\{item1, item2, \dots\} or use code set(list), which will turn a list to set
	\item Properties of set?\\
	It has no order and there is no repeats
	\item Add and remove key of a set?\\
	Corresponding to set.add(key) and set.remove(key)
	\item Common operation on sets?\\
	Union ``|'' and intersection ``\&''
	\item Difference between dict and set?\\
	Share the same mechanism but set doesn’t have value, can't put mutable obj in
	\item Is str mutable? What about list?\\
	Str is immutable and list is mutable
	\item Immutable objects?\\
	Number, string and tuple. Contents in certain add can't be changed
	\item How to find the memory address of an object?\\
	Use function id(), id(o) will return the memory add
	\item How to get console args?\\
	By import sys, and then use sys.argv[idx]
	\item What is configurative programming?\\
	A framework is created such that coding is like setting configuration
	\item Mechanism of importing a module? \\
	Python interpreter just go through line by line(interview question)
\end{enumerate}

\end{document}