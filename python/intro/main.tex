\documentclass[10pt,a4paper,oneside]{article}
\usepackage[utf8]{inputenc}
\usepackage[T1]{fontenc}
\usepackage{amsmath}
\usepackage{amsfonts}
\usepackage{amssymb}
\usepackage{graphicx}
\usepackage{pythonhighlight}


\author{Baboo J. Cui}
\title{Introduction to Python}

\begin{document}
\maketitle
\tableofcontents

\newpage

\section{Quick Start}
This is an easy part, let's go through these problems!

\begin{enumerate}
	\item What is Python?
	
	a programming language
	
	\item Who create it?
	
	Guido van Rossum
	
	\item Any other programming language?
	
	like C, C++, java, javascript(almost nothing to do with java), php, matlab, etc. There are about $600$ languages, their popularity can be checked on \textbf{TIOBE}(The Importance Of Being Earnest, emphasize sincere and professional attitude) programming community index
	
	\item What kind of programming language Python is?
	\begin{itemize}
		\item advance language: more abstract,  less efficient, easier(relative)
		\item dynamic(vs static) language: all are determined during execution, structure and type are mutable
		\item interpreting(vs compiling) language: execute by interpreter during run-time 
		\item weak(vs strong) type language: no need to specify type to declare a variable
	\end{itemize}
	
	\item Which version to use?
	
	Just use Python 3, it has clear standard and repeated codes are removed
	
	\item Advantages of Python?
	\begin{itemize}
		\item elegant, clear, simple
		\item wildly used in companies: Douban, Google, YouTube, etc
		\item cover many scenarios: backup, back-end, web service, widgets, script, game
		\item relatively easy for developer(less code)
	\end{itemize}
	
	\item Disadvantages of Python?
	\begin{itemize}
		\item relatively slow: cannot be used for OS level, real-time processing(some cases)
		\item restriction for some areas: iOS, Android, etc
		\item code exposition(may be advantage)
	\end{itemize}

	\item What makes Python easy and fast for development?
	
	mainly because the built-in and third-part library
	
	\item What "run" Python code?
	
	Python interpreter. It's a software written by other language(official default: CPython, may be other: Jython)
	
	\item What is operating system?
	
	software that connects the hardware and application
\end{enumerate}

\section{Installation}
Recommend to use \textbf{Anaconda} or \textbf{Miniconda}. It will automatically configure everything for you. Better check before use it. In MacOS, \textbf{brew} can be used.

\section{Two Types of Modes}
\begin{itemize}
	\item \textbf{interactive} mode: execute input code line by line
	\item \textbf{command line} mode: directly run a python file. Note that the current path must include the python file, otherwise cause error
	\item \textbf{file execution} mode: add \pyth{#!/usr/bin/env python3} at first line and make the file executable by \textbf{\$ chmod +x }, this may cause error due to \textbf{environment} problem
\end{itemize}

\section{Fundamental IO}
\begin{itemize}
	\item input: \pyth{input()}, note that it always return a \textbf{string}
	\item output: \pyth{print()}, it can print almost everything
\end{itemize}

\section{Format}
Placeholder for format in print can be summarized as following:
\begin{itemize}
	\item \%d: int
	\item \%f: float
	\item \%x: hex
	\item \%r: string canonical, tip r is for repr
	\item \%s: string, this always works
\end{itemize}
\begin{python}
	print("%2d-%02d" % (3, 1))  #3-01
	print("%.2f" % 3.1415926)  #3.14
	print("Hi, %s, you have $%d." % ("abc", 1000000))
\end{python}
format function \pyth{format()} can also be used:
\begin{python}
	print("{} is {}".format(a, b))  # comma will be translated to white space
\end{python}



\section{Who to Take?}
Hopefully everybody!

\end{document}