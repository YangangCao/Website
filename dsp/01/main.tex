\documentclass[10pt,a4paper,oneside]{article}
\usepackage[utf8]{inputenc}
\usepackage{amsmath}
\usepackage{amsfonts}
\usepackage{amssymb}
\usepackage{graphicx}
\usepackage{breqn}
\usepackage{tikz} % system block diagram
\usepackage{textcomp}
\usetikzlibrary{datavisualization}
\usetikzlibrary{shapes,arrows} % system block diagram
\usepackage{booktabs}
\usepackage[framed,numbered,autolinebreaks,useliterate]{mcode} % matlab code block
\author{Yangang Cao}
\date{May 28, 2019}
\newcommand{\degree}{^\circ}
\tikzset{
	delay/.style    = {draw, thick, rectangle, minimum height = 3em,
		minimum width = 3em},
	sum/.style      = {draw, circle, node distance = 2cm}, 
	prod/.style     = {draw, circle, node distance = 2cm},
	input/.style    = {coordinate}, % Input
	output/.style  = {coordinate} % Output
}
% Defining string as labels of certain blocks.
\newcommand{\product}{$\displaystyle \times$}
\newcommand{\delay}{\large$z^{-1}$}
\begin{document}

\title{Part 01 - Introduction}
\maketitle 
\tableofcontents
\newpage
\section{Basics}
\subsection{Signals, Systems, and Signal Processing}
\begin{itemize}
\item  Signal: Physical quantity that varies independent variable or variables
\item  System: Device that performs an operation on a signal
\item Algrithm: Method or set of rules for implementing the system
\item Signal Processing: Passing a signal through a system
\item A/D Converter: Converting the signal from analog to digital
\item D/A Converter: Converting the signal from digital to analog
\end{itemize}
\subsection{Classification of Signals}
\begin{itemize}

\item Multichannel and Multidimensional Signals
\begin{itemize}
	\item  Multichannel Signals: Signal with one independent variable
	\item  Multidimensional Signals: Signal with multiple independent variables
\end{itemize}
\item Continuous-Time Versus Discrete-Time Signals
\begin{itemize}
	\item  Continuous-time Signals: Signal with every value of time
	\item  Discrete-time Signals: Signal with specific values of time
\end{itemize}
\item Continuous-Valued Versus Discrete -Valued Signals
\begin{itemize}
	\item  Continuous-valued Signals: Signal with continuous values
	\item  Discrete-valued Signals: Signal with discrete values
\end{itemize}
\item Deterministic Versus Random Signals
\begin{itemize}
	\item  Deterministic  Signals: Signals that can be uniquely described 
	\item  Random Signals: Signals evolve in an unpredictable manner
\end{itemize}
\end{itemize}

\section{The Concept of Frequency in Continuous-Time and Discrete-Time Signals}
We expect that the nature of time (continuous or discrete) would affect the nature of the frequency accordingly.
\subsection{Continuous-Time Sinusoidal Signals}
A simple harmonic oscillation:
\[
x_a(t) = A\cos(\Omega t+\theta), -\infty<t<\infty
\]
$A$ is the amplitude of sinuoid, $\Omega$ is the frequency, and $\theta$ is the phase in radians. Instead of $\Omega$, we often use the frequency $F$ in cycles per second or hertz(Hz), where
\[
\Omega = 2\pi F
\]
The analog sinusoidal signal is characterized by the following properties:
\begin{itemize}
\item  $x_a(t)$ is periodic if 
$
x_a(t+T_p)=x_a(t)
$, 
where $T_p=1/F$ is the fundamental period of the sinusoidal signal.
\item Continuous-time sinusoidal signals with distinct frequencies are themselves distinct.
\item Increasing the frequency $F$ results in an increase in the rate of oscillation of the signal.
\end{itemize}
Corresponding complex exponential form
\[
x_a(t) = Ae^{j(\Omega t+\theta)}
\]
This can easily be seen by expressing these signals in terms of sinusoids using the Euler indentity
\[
e^{\pm j\phi}=\cos \phi\pm j\sin \phi
\]
\subsection{Discrete-Time Sinusoidal Signals}
A discrete-time sinusoidal signal:
\[
x(n)=A\cos (\omega n+\theta), -\infty<n<\infty
\]
where $n$ is an integer variable, called the sample number, $A$ is the amplitude of the sinusoid, $\omega$ is the frequency in radians per sample, and $\theta$ is the phase in radians.

The discrete-time sinusoids are characterized by following properties:
\begin{itemize}
	\item  A discrete-time sinusoid is periodic only if its frequency $f$ is a rational number.
	\item Discrete-time sinusoids whose frequencies are separated by an integer multiple of $2\pi$ are identical.
	\item The highest rate of oscillation in a discrete-time sinusoid is attained when $\omega=\pi$(or $\omega=-\pi$) or, equivalently, $f=\frac{1}{2}$ (or $f=-\frac{1}{2}$).
\end{itemize}
Usually, we choose the range $0\leqslant\omega\leqslant2\pi$ or $-\pi\leqslant\omega\leqslant\pi$ ($0\leqslant f\leqslant1, -\frac{1}{2}\leqslant f\leqslant\frac{1}{2}$), which we call the $fundamental\ range$.
\subsection{Harmonically Related Complex Exponentials}
These are sets of periodic complex exponentials with fundamental frequencies that are multiples of a single positive frequency. We can construct a linear combination of harmonically related complex exponentials by adding continuous-time, harmonically related exponentials, which is called $Fourier\ series\ expansion$. For discrete-time, $f_0=1/N$, they are only $N$ distinct periodic complex exponentials, this is called $Fourier\ series$.
\section{Analog-to-Digital and Digital-to-Analog Conversion}
We view A/D conversion as a three-step process: sampling, quantization and coding. The accuracy of D/A conversion depends on the quality of D/A conversion, a simple form of D/A conversion is called zero-order hold or staircase approximation. The sampling rate is sufficiently high to avoid the problem commonly called $aliasing$.
\subsection{Sampling of Analog Signals}
We limit our discussion to $periodic$ or $uniform\ sampling$, $F_s=1/T$ is called the $sampling\ rate$(hertz). $F$ is frequency of analog signal, the frequency variables $F$ and $f$ are linearly related as
\[
f=\frac{F}{F_s}
\]
and $f$ is also called $relative$ or $normlized\ frequency$.

We observe that the fundamental difference between continuous-time and discrete-time signals is in their range of values of the frequency variable $F$ and $f$, or $\Omega$ and $\omega$.
\[
F_{max}=\frac{F_s}{2}=\frac{1}{2T}
\]
\[
\Omega_{max} = \pi F_s = \frac{\pi}{T}
\]
For example, $F_2=10Hz$ is an alias of the frequency $F_1=50Hz$ when $F_s=40Hz$. The relationship is
\[
F_k = F_0+kF_s,\ k=\pm1,\pm2,...
\]
$F_s/2$ is called $folding\ frequency$.
\subsection{The Sampling Theorem}
$F_s$ is selected that 
\[
F_s>2F_{max}
\]
 where $F_{max}$ is the largest frequency component in the analog signal. The sampling rate $F_N=2B=2F_{max}$ is called the $Nyquist\ rate$. 
\subsection{Quantization of Continuous-Amplitude Signals}
\begin{itemize}
\item Quantization: The process of converting a discrete-time continuous-amplitude signal into a digital signal by expressing each sample value as a finite number of digits
\item Quantization Error:  Difference between the quantized value and the actual sample value
\item Quantization Level: Values allowed in the digital signal
\item Quantization step \ size or resolution: Distance $\delta$ between two successive quantization levels
\item Two ways of quantization: Truncation and rounding
\end{itemize}
\subsection{Quantization of Sinusoidal Signals}
Sinusoids are used as test signals in A/D converters. If the sampling rate $F_s$ satisfies the sampling theorem, quantization is the only error in the A/D conversion process. The quality of the output of the A/D converter is usually measured by the $signal-to-quantization\ noise\ ratio(SQNR)$.
\[
\text{SQNR}=\frac{P_x}{P_q}=\frac{3}{2}\cdot2^{2b}
\]
Expressed in decibels(dB), the SQNR is
\[
\text{SQNR(dB)} = 10\log_{10}\text{SQNR}=1.76+6.02b
\]
\subsection{Coding of Quantized Samples}
The number of bits required in the coder is the smallest integer greater than or equal to $\log_2L$.
\end{document}
