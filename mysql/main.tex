\documentclass[10pt,a4paper,oneside]{article}
\usepackage[utf8]{inputenc}
\usepackage[T1]{fontenc}
\usepackage{amsmath}
\usepackage{amsfonts}
\usepackage{amssymb}
\usepackage{graphicx}
\usepackage{enumerate}
\date{June 21, 2019}
\author{Baboo J. Cui, Yangang Cao}
\title{MySQL Q\&A}
\begin{document}
\maketitle
\tableofcontents

\newpage

\section{MySQL Basic Concept}
\begin{enumerate}[1.]
\item What is database? \\
A container for data
\item What is SQL? \\
Short for structured query language, not a patented language
\item Is SQL case sensitive? \\
No!
\item  How to end a SQL statement?\\
Must be ended with `;', otherwise won’t be finished
\item What is DBMS? \\
Database manage system
\item What is table? \\
List of specific data type
\item What is schema? \\
Layout info of tables and databases
\item What is column? \\
A field in a table, all tables are composed by one or many columns
\item What is row? \\
A row essentially is a record
\item What is data type? \\
The type of data(straight forward)
\item  What is primary key?\\
A column that can uniquely determine a row(record)
\item Properties of primary key?\\
Must be unique for each row, cannot be NULL
\end{enumerate}
\section{MySQL Software}
\begin{enumerate}[1.]
\item What 2 parts does MySQL contain?\\
Server part and client part
\item How to install MySQL? \\
Use command \$ apt install mysql-sever, note, the program by default should be server!
\item  How to connect to MySQL in console? \\
Use command \$ mysql -u UNAME [-h HOST\_IP] [-P PORT\_NUM] -p to login into database
\item  How to connect to MySQL by workbench? \\
GUI, easy
\item  How to exit MySQL? \\
Use command > quit or > exit
\item  What is default port of MySQL? \\
Port 3306
\item  How to rename connection name in workbench? \\
Topbar Database -> Manage Connections -> Connection Name
\end{enumerate}
\section{Show Top-Level Info}
\begin{enumerate}[1.]
\item  How to show all databases?  \\
Use command > SHOW DATABASES;
\item How to choose a database to use?   \\
Use command > USE database\_name; Data can be used only after the database is opened!
\item  How to show all tables info in a database?  \\
Use command > SHOW TABLES; Tip: a schema usually has more than 1 table
\item  How to show column info of a table?  \\
Two ways: use command > SHOW COLUMNS FROM table\_name; or > DESCRIBE table\_name;
\item What will be show in column info?   \\
Field(name of col), Type(int, char…), NULL(control if could be empty), Key(primary key?), Default, Extra(auto\_incre?)
\item  How to see current accessibility?  \\
Use command > SHOW GRANTS;
\item   Use command > SHOW GRANTS; \\
Use command > SHOW STATUS;
\end{enumerate}
\section{Retrieving Data}
\begin{enumerate}[1.]
\item  How to show a table?  \\
Use command > SELECT * FROM table\_name; Tip: * is wildcard, it is not efficient, result orders may vary
\item   How to show a column? \\
Use command > SELECT col\_name FROM table\_name;
\item  How to show several columns?  \\
Use command > SELECT col\_1, col\_2 FROM table\_name; Tip: add ``,'' between several columns
\item Show distinct values in certain column?   \\
Use command > SELECT DISTINCT col\_name FROM table\_name; Tip: add ``,'' between several columns
\item  Show limited num of output?  \\
Use command > SELECT col\_name FROM table\_name LIMIT num; This will only output num rows, e.g. num=10
\item  Show limited num of output with offset?  \\
Use command > SELECT col\_name FROM table\_name LIMIT num1 OFFSET num2; start from num2 with limit=num1 
\item   What is about fully qualified table name? \\
Use ``.'' operator to specify column or table, e.g. > SELECT table\_name.col\_name FROM table\_name;
\end{enumerate}
\section{Sorting Retrieving Data}
\begin{enumerate}[1.]
\item   How to order the output? \\
Use keyword ORDER BY, e.g. > SELECT col\_1, col\_2, col\_3 FROM table ORDER BY col\_1, col\_2; Tip: first go through col\_1 and then col\_2
\item How to specifying order direction?   \\
By default, ASC(ascending) is used, we can set it to DESC: > SELECT col FROM table ORDER BY col DESC;
\item What is the range of DESC?   \\
DESC only works on its following column, apply to more columns if DESC is needed
\item  Relationship between ORDER BY and LIMIT?  \\
LIMIT must follow ORDER BY, namely, ORDER BY cannot be used before LIMIT
\end{enumerate}
\section{Data Filtering}
\begin{enumerate}[1.]
\item How to add search criteria?   \\
Use keyword WHERE, add a where clause
\item  Relationship between ORDER BY and WHERE?  \\
ORDER BY should follow WHERE, ORDER BY deals with output generated by WHERE
\item  How to search with equality constraint?  \\
Add equality condition after WHERE: > SELECT col FROM table WHERE col=num(like column condition)
\item  WHERE clause operators?  \\
=, !=,  <, <=, >, >=, BETWEEN, all of them are quite straightforward
\item  How to use keyword BETWEEN?  \\
Use clause BETWEEN a AND b, > SELECT col FROM table WHERE col BETWEEN a AND b;
\item How to check NULL value?   \\
Use clause WHERE col IS NULL, Note that NULL is a special value, must deal with it specifically
\item Logical operators for WHERE clause?   \\
Basically 3 operators: AND, OR and NOT
\item  Priority of logical operator?  \\
NOT>AND>OR. Don’t have to remember these, use ``()'' to make everything clear
\item  What is IN operator for?  \\
To specify a certain value in a set, > SELECT col FROM table WHERE col IN (a, b, c…);
\end{enumerate}
\section{Wildcard Filtering}
\begin{enumerate}[1.]
\item  What is wildcard?  \\
Special chars that is used to match certain pattern
\item   What is the keyword for using wildcard? \\
Keyword LIKE, > SELECT col FROM table WHERE col LIKE `1f\%';
\item What does ``\%'' match?   \\
``\%'' matches any char by anytime, sounds like whatever. ``\%'' doesn’t match NULL, matching may be case sensitive base on settings
\item  What does ``\_'' match?  \\
``\_'' matches any char by one time
\item   Techniques of using wildcard? \\
Don't use too much, resource consuming; Any matching start with wildcard is very slow 
\end{enumerate}
\section{Searching with Regular Expression}
\begin{enumerate}[1.]
\item  What is regular expression for?  \\
For matching special string in context, regular expression is a special language
\item Which keyword is related to regular expression?   \\
Use keyword REGEXP
\item  Similarity between wildcard and regexp?  \\
They both are used for matching a string(though in different way)
\item   Difference between wildcard and regexp? \\
Wildcard is offered in system level and regexp is a language. LIKE is for whole string, REGEXP is for substring
\item  What is the format for using regular expression?  \\
By clause REGEXP `regexp\_statement' > SELECT col FROM table WHERE col REGEXP `regexp';
\item   How to make REGEXP case-sensitive? \\
By adding a keyword BINARY after REGEXP
\item  What does ``.'' do in regexp?  \\
For matching any one char
\item  How to match more than one case?  \\
Connect more than one case by ``|'', e.g REGEXP `1000|2000', like OR in SELECT, or use ``[]'', e.g REGEXP `[123] ton''
\item   How to match a number? \\
By REGEXP `[0-9]'
\item   How to match a letter? \\
By REGEXP `[a-z]'
\item  How to match special content like ``.''  \\
Add `$ \verb|\\|$' before it, e.g REGEXP `$ \verb|\\|$.' Usually regexp requires one `$ \verb|\|$', but regexp in MySQL requires two
\item What is character class?   \\
Just remember some: [:alnum:]=letter+num, [:alpha:]=letter, [:digit:]=num
\item  What is ``*'' for?  \\
Match any times, include zero time
\item   What is ``+'' for? \\
Match one or many times
\item   What is ``?'' for? \\
Match zero time or one time
\item   What is ``\{n\}'' for? \\
For matching n times
\item  What is ``\{n,\}'' for?  \\
For matching at least n times
\item What is ``\{n, m\}'' for?   \\
For matching n to m times
\item What is `` \^\ '' for?   \\
Indicate the beginning of the text
\item  What is ``\$'' for?  \\
Indicate the ending of text
\item  What is ``[[:<:]]''?  \\
Indicate the beginning of a word
\item   What is ``[[:>:]]''? \\
Indicate the ending of a word
\end{enumerate}
\section{Create Calculated Field}
\begin{enumerate}[1.]
\item  What is field?  \\
Field == column (almost)
\item What is calculated field?   \\
Do something to field(s), it is not in database, it is generated during the run time
\item Concatenate field?   \\
Use keyword CONCAT(), e.g. > SELECT CONCAT(col\_1, ’+’, col\_2) FROM table;
\item  What does Trim(), LTrim() and RTrim() do?  \\
They all act on column(field) for deleting extra white-space
\item  How to use alias?  \\
Use keyword AS. e.g > SELECT whatever AS alias\_name FROM table; note that alias is also called derived column
\item What is the range of AS?   \\
AS only works on its previous column, namely give an alias name right before it
\item  Arithmetic operation in MySQL?  \\
Easy: +, -, *, /
\item  Example of arithmetic operation?  \\
> SELECT price, quantity, price*quantity AS total FROM table; 
\item   How to get current time? \\
> SELECT Now();
\end{enumerate}
\section{Using Data Manipulation Function}
\begin{enumerate}[1.]
\item  How to convert text to upper case?  \\
Use function UPPER(str), Tip on LOWER(str)
\item  How to get length of string?  \\
Use function LENGTH(str)
\item   How to trim a string? \\
Use function TRIM(str), Tip on LTRIM(str) and RTRIM(str)
\item  How to get current time info?  \\
SELECT NOW();
\item How to get current date?   \\
SELECT CURDATE();
\item  How to get current time?  \\
SELECT CURTIME();
\item How to get part of date?   \\
DAY(date), MONTH(date), YEAR(date), Tip: HOUR(time), MINUTE(time), SECOND(time)
\item   What is date format in SQL? \\
yyyy-mm-dd, Use =, BETWEEN in WHERE to make time control powerful
\item  How to avoid data format error in SQL?  \\
Always use DATE() to do type coercion to avoid that
\item  Any common numerical function?  \\
ABS(), EXP(), MOD(), PI(), RAND(), SQRT(), TAN(), SIN()…
\end{enumerate}
\section{Summarizing Data(Aggregate Function)}
\begin{enumerate}[1.]
\item What is aggregate function?   \\
A function that works on a column and return a scalar value
\item  What are all aggregate function offered in SQL?  \\
AVG() for average, COUNT(), MAX(), MIN(), SUM(), all 5 functions
\item   How AVG() acts on NULL value? \\
Just ignore the value, Tip: AVG() only works on one column
\item  Default of aggregate function?  \\
By default, ALL condition is used, if distinct values are required, use DISTINCT keyword, > SELECT AVG (DISTINCT col) 
\end{enumerate}
\section{Grouping Data}
\begin{enumerate}[1.]
\item How to create group?   \\
Use keyword GROUP BY, usually is used with aggregate function
\item  What keyword is used for setting condition?  \\
Use keyword HAVING, e.g > SELECT COUNT(*), price FROM juno.app\_order GROUP BY price HAVING price>10000 ORDER BY price;
\item  Keywords order with HAVING?  \\
SELECT, FROM, WHERE, GROUP BY, HAVING, ORDER BY, LIMIT, Tip: LIMIT is almost always in the end 
\item   Difference between WHERE and HAVING? \\
WHERE is used before grouping and HAVING is for grouped result
\end{enumerate}
\section{Subquery}
\begin{enumerate}[1.]
\item What is subquery?   \\
Nest one SQL statement into another
\item  Scope of variable?  \\
Usually, use table.var to specify the variable
\item  When will subquery be used?  \\
Usually in IN statement, other statements also use this
\item Comment on subquery?   \\
For convenience, subquery can be decomposed into several statements, also it can be used with calculated fields
\end{enumerate}
\section{Joining Tables(inner join)}
\begin{enumerate}[1.]
\item  What is primary key?  \\
A unique value that can determine a record(row)
\item  What is foreign key?  \\
A column(field) that contains the primary key of another table
\item  What is called scale well?  \\
System works well when magnitude of database increases significantly 
\item  How to join table?  \\
Just write more than 2 tables after select, this will create a Cartesian product, Usually WHERE statement is required to make it work
\item  What is inner join?(Most common one)  \\
It is equal-join, based on value equality 
\item   Syntax of inner join? \\
> SELECT a\_c1, a\_c2, b\_c1, b\_c2 FROM a INNER JOIN b ON a.a\_c1=b.b\_c;, note ON is used, not WHERE
\item Disadvantage of join?   \\
Don’t join too much if unnecessary, performance will be very poor
\end{enumerate}
\section{Advanced Join(self join and outer join)}
\begin{enumerate}[1.]
\item  How many joins are there?   \\
Inner join(equal-join), self join, natural join(trivial), outer join. 
\item   What is self join?  \\
Select part of table based on feature of itself. E.g > SELECT t1.c1, t1.c2 FROM tab AS t1,  tab AS t2 WHERE t1.c=t2.c AND t2.c1=val;
\item   Could self join be achieved by other methods?  \\
Yes, subquery can achieve the same result
\item   What is outer join?  \\
Join that contains irrelevant rows in table(will show everything of one table). Use key word LEFT OUTER JOIN or RIGHT OUTER JOIN, left and right must be specific
\item   Key words of outer join with aggregation?  \\
SELECT, COUNT, AS, ON, GROUP BY 
\end{enumerate}
\section{Combining Queries}
\begin{enumerate}[1.]
\item   What is combing queries?  \\
Execute more than one SELECT and return result as one
\item   How to combine queries?  \\
Use keyword UNION
\item   Regulation on UNION?  \\
Must be more than 2 SELECT, each SELECT result must be compatible, repetition is deleted by default
\item  How to show all result(repeated included)?   \\
Add keyword ALL after UNION, namely UNION ALL
\item  Where ORDER BY should be used w.r.t. UNION?    \\
In the end, ORDER BY works on the total result
\end{enumerate}
\section{Extra}
\begin{enumerate}[1.]
\item   Cannot install mysqlclient in centOS  \\
\$ yum install mysql-community-devel.x86\_64
\end{enumerate}
\end{document}