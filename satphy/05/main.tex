\documentclass[10pt,a4paper,oneside]{article}
\usepackage[utf8]{inputenc}
\usepackage[T1]{fontenc}
\usepackage{amsmath}
\usepackage{amsfonts}
\usepackage{amssymb}
\usepackage{graphicx}

\usepackage{tikz}
\usepackage{xcolor}
\usepackage{eso-pic}
\newcommand{\watermark}[3]{\AddToShipoutPictureBG{
		\parbox[b][\paperheight]{\paperwidth}{
			\vfill%
			\centering%
			\tikz[remember picture, overlay]%
			\node [rotate = #1, scale = #2] at (current page.center)%
			{\textcolor{gray!70!cyan!40!red!20}{#3}};
			\vfill}}}

\author{Baboo J. Cui}
\title{Modern Physics}
\begin{document}
\watermark{50}{9}{www.baboocui.club}
\maketitle
\tableofcontents

\newpage
Basic related concept:
\begin{itemize}
	\item \textbf{heat}: thermal energy transmitted from one body to another(energy in transit)
	\item \textbf{thermal energy}: due to random motion of molecules 
	\item \textbf{temperature}: measure of concentration of internal energy
\end{itemize}

\section{Temperature}
Two units for temperature:
\begin{itemize}
	\item degree Celsius($^\circ C$): water freeze at $0 ^\circ C$, and boils at $100 ^\circ C$ in standard condition
	\item kelvins($K$): absolute temperature scale 
	\item Fahrenheit is almost ignored in physics
\end{itemize}
The relationship between these two measures:
\[
T(K) = T(^\circ C) + 273
\]
\textbf{Triple point} is the temperature that $3$ phase(liquid, solid and vapor) of a substance that coexist.

\section{Heat Transfer}
\subsection{Heat Transfer Due to Temperature Change}
The relationship is:
\[
Q = mc \Delta T
\]
where
\begin{itemize}
	\item $Q$ is the heat transfered during the process, positive $Q$ represent heat coming and temperature increasing
	\item $m$ is the mass of sample
	\item $c$ is the specific heat(intrinsic property of an object)
	\item $\Delta T$ is the temperature change
\end{itemize}
Two contacted objects will finally reach thermal equilibrium, meaning they have the same temperature. 

\subsection{Heat Transfer Due to Phase Change}
The relationship is:
\[
Q = mL
\]
where
\begin{itemize}
	\item $Q$ is the heat transfered during phase change, note that temperature is constant in this process
	\item $m$ is the mass of sample
	\item $L$ is \textbf{latent heat of transformation}, could be latent heat of fusion, vaporization, etc
\end{itemize}
Summary of phase changes:
\begin{itemize}
	\item solid to liquid: \textbf{melting}, reverse: \textbf{freezing}
	\item liquid to gas: \textbf{vaporization}, reverse: \textbf{condensation}
	\item solid to gas: \textbf{sublimation}, reverse: \textbf{deposition}
\end{itemize}

\subsection{Ways of Heat Transfer}
There are three ways for heat transfer:
\begin{itemize}
	\item \textbf{conduction}: by contact
	\item \textbf{convection}: by motion of fluid
	\item \textbf{radiation}: by EM waves
\end{itemize}
When crystal absorbs of gives off heat, either temperature of phase changes.

\section{Thermal Expansion}
Change in temperature may lead to change in size, usually size expanded when temperature increases and shrinks as temperature goes lower. Relationship in change of length is:
\[
L_f - L_i = \alpha L_i(T_f- T_i)
\]
or in simpler form:
\[
\Delta L = \alpha L_i \Delta T
\]
where
\begin{itemize}
	\item $L$: length, sub $i$ represents initial and sub $f$ represents final
	\item $\alpha$: coefficient of linear expansion
\end{itemize}
Similarly, relationship in change of volume is:
\[
\Delta V = \beta V_i \Delta T
\]
where $\beta$ is the \textbf{coefficient of volume expansion}. For most solid $\beta \approxeq 3\alpha$. Most substance have positive $\beta$, one exception is water from $0^\circ C$ to $4^\circ C$(maximum density).

\section{Kinetic Theory of Gases}
Pressure($P$) is defined as:
\[
P = \frac{F}{A}
\]
Avogadro's constant(number of atoms or molecules in $1$ \textbf{mole}) is:
\[
N_A = 6.02 \times 10^{23}
\]
The molar mass($M$) of an object is:
\[
M = m N_A \quad \text{where $m$ is th \textbf{unit mass}}
\]
\textbf{Idea gas law} can be expressed as equation:
\[
PV = nRT
\]
where
\begin{itemize}
	\item $n$: number of moles of gas
	\item $R$: universal gas constant which is $8.314$ in SI
	\item $T$: temperature in kelvins
\end{itemize}
Average translational kinetic energy of gas molecules is directly proportional to the absolute temperature:
\[
K_{avg} = \frac{3}{2} k_B T
\]
where 
\[
k_B = \frac{R}{N_A}
\]
By root-mean-square speed:
\[
\frac{1}{2}mv^2 = \frac{3}{2} k_B T
\]
RMS speed is:
\[
v_{rms} = \sqrt{\frac{3k_B T}{m}} = \sqrt{\frac{3RT}{M}}
\]

\section{Laws of Thermodynamics}

\subsection{Zeroth Law of Thermodynamics}
If object $1$ and $2$ are each in thermal equilibrium with object $3$, then object $1$ and $2$ are in thermal equilibrium with each other.

\subsection{First Law of Thermodynamics}
Energy is neither created nor destroyed in any thermodynamic system, mathematically:
\[
\Delta U = Q - W
\]
where
\begin{itemize}
	\item $\Delta U$ is the change in internal energy of the system
	\item $Q$ is heat added to the system
	\item $W$ is work done by the system
\end{itemize}
 Work done by gas(assume gas has constant pressure) is:
 \[
 W = P \Delta V \quad \text{or differntial form} \quad dW = PdV
 \]
 Two types of process:
 \begin{itemize}
 	\item isothermal process: temperature remains constant($P-V$ graph is reciprocal function)
 	\item adiabatic process: no heat exchange between system and surrounding($P-V$ graph is steeper)
 \end{itemize}

\subsection{Second Law of Thermodynamics}
The total \textbf{entropy} of a system plus its surrounding will never decrease. Which implies:
\begin{itemize}
	\item heat spontaneously go from hot to cold
	\item a heat engine can never operate at $100\%$ efficiency
\end{itemize}
A heat engine is a device that use heat to produce useful work. Its efficiency($e$) is:
\[
e = \frac{Q_H - |Q_C|}{Q_H} = 1 - \frac{|Q_C|}{Q_H}
\]
The most efficient heat engine follows \textbf{Carnot} cycle:
\begin{enumerate}
	\item isothermal expansion
	\item adiabatic expansion
	\item isothermal compression
	\item adiabatic compression
\end{enumerate}
it has efficiency:
\[
e = \frac{T_H - T_C}{T_H} = 1 - \frac{T_C}{T_H}
\]
\end{document}