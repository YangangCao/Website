\documentclass[10pt,a4paper,oneside]{article}
\usepackage[utf8]{inputenc}
\usepackage[T1]{fontenc}
\usepackage{amsmath}
\usepackage{amsfonts}
\usepackage{amssymb}
\usepackage{graphicx}

\usepackage{tikz}
\usepackage{xcolor}
\usepackage{eso-pic}
\newcommand{\watermark}[3]{\AddToShipoutPictureBG{
		\parbox[b][\paperheight]{\paperwidth}{
			\vfill%
			\centering%
			\tikz[remember picture, overlay]%
			\node [rotate = #1, scale = #2] at (current page.center)%
			{\textcolor{gray!70!cyan!40!red!20}{#3}};
			\vfill}}}

\author{Baboo J. Cui}
\title{Modern Physics}
\begin{document}
\watermark{50}{9}{www.baboocui.club}
\maketitle
\tableofcontents

\newpage

\begin{itemize}
	\item it's about physics in 20th century
	\item $10$ percent of the question on exam will cover this part
\end{itemize}

\section{Rutherford Model of the Atom}
Atoms was \textbf{wrongly} considered as \textit{raisin pudding model}. Rutherford fired \textbf{alpha particles}(helium nuclei) at extremely thin sheet of gold, if pudding model is correct, alpha particle should sail through with little deviation, however
\begin{itemize}
	\item most just deviated
	\item some deflected a very large angle: positive charge is concentrated in a very tiny volume(known as nucleus surround by electron)
\end{itemize}
this is known as Rutherford nuclear model, which is \textit{accurate enough}.

\section{Photoelectric Effect}
Duality of light:
\begin{itemize}
	\item wave
	\item particle(reveled in 20th century)
\end{itemize}
When metal is illuminated by EM radiation, the energy from photon can liberate electron from their bound state to fly off, these electrons are known as \textbf{photoelectrons}. And the phenomena is called \textbf{photoelectric effect}. And it has properties:
\begin{itemize}
	\item the effect has no time delay
	\item increasing the intensity of light won't cause the electrons to leave with greater kinetic energy
	\item each metal has a threshold frequency $f_0$, if less than it, no photo-electron were ejected regardless of the intensity of light
	\item more electron are ejected by increasing intensity of light, there is a maximum photo-electron kinetic energy
	\it
\end{itemize}
these can only be explained by particle property of light. The energy of photon $E$ is:
\[
E = hf
\]
where
\begin{itemize}
	\item $h$ is Planck's constant, it's about $6.63\times10^{-34}J \cdot s$, VIP value for future study
	\item $f$ is the frequency of EM
\end{itemize}
\textbf{Work function} $\Phi$ is a certain amount of energy had to imparted to an electron to liberate it with a maximum kinetic energy $K$ with relationship:
\[
K = hf - \Phi
\]
Clearly, photon with energy less that $\Phi$ won't cause photo-electron effect(threshold frequency explanation). It is required to understand $2$ graphs:
\begin{itemize}
	\item $K-f$ graph: linear function with $x$ intercept $\Phi/h$
	\item $K-\text{Intensity}$ graph: a constant function
\end{itemize}
\textbf{Electronvolt}($eV$) is a small energy unit:
\[
1 eV = 1.6 \times 10^{-19} J
\]

\section{Bohr Model of the Atom}
From experimentation, \textbf{atomic spectra} only contains patterns of sharp lines, that means atoms emit radiation only at certain wavelengths:
\begin{itemize}
	\item energy levels are \textbf{quantized}
	\item electron can be in different energy level
	\item electron can be \textbf{excited} to higher energy level by absorbing energy
	\item a photon is emitted when electron returns to a lower energy level(orbit)
\end{itemize}
For \textbf{one}-electron atoms, the energy level is
\[
E_n = \frac{Z^2}{n^2}(-13.6eV)
\]
\begin{itemize}
	\item $n$ is the level index, goes from $1$ to $\infty$(infinitely far away from nucleus)
	\item $Z$ is the number of protons in the atom's nucleus
\end{itemize}
The energy $E$ of photon emitted when electron goes from high level $h$ to low level $l$ is:
\[
E = E_h - E_l
\]

\section{Wave-Particle Duality}
Wave-partial duality says that electromagnetic radiation propagates like a wave but exchanges energy like a particle.
\begin{itemize}
	\item wave can behave like particle
	\item particles can behave like a wave too
\end{itemize}
Particles in motion can display wave characteristics:
\[
\lambda = \frac{h}{p} = \frac{h}{mv}
\]
\begin{itemize}
	\item $\lambda$ is the wave length of moving object, known as \textbf{de Broglie wavelength}
	\item $p$ is the momentum of the object
\end{itemize}

\section{Nuclear Physics}
Some basic definition:
\begin{itemize}
	\item nucleus is composed of \textbf{nucleons}, which contains \textbf{neutron} and \textbf{proton}
	\item \textbf{atomic number} is the number of proton, denoted by $Z$, number of neutron is denoted by $N$
	\item \textbf{mass number} is also called nucleons number, it is $A=Z+N$
	\item \textbf{isotope} is nuclei that contains the same number of protons but different number of neutrons
	\item the notation for a nuclide is
	\[
	\sideset{^A_Z}{}{\mathop{\mathrm{X}}}
	\]
\end{itemize}

\subsection{Nuclear Force}
The \textbf{strong nuclear force}, which binds together neutrons and protons to form nuclei, though there is a strong repulsive Coulomb force.

\subsection{Binding Energy}
When a proton and neutron is bound to form a deuteron, the nucleus of deuterium, there is a \textbf{mass defect}, and binging energy is computed by mass-energy equation:
\[
E_B = (\Delta m) c^2
\]
Binding energy per nucleon is $E_B/A$. deuteron has lowest BEPN $1.12 MeV$, and highest is $\sideset{^{62}}{}{\mathop{\mathrm{Ni}}}$, which is $8.8MeV$.
\begin{itemize}
	\item \textbf{fusion} of elements to form new element before nickel release energy
	\item \textbf{fission} of elements after nickel release energy
\end{itemize}

\section{Radioactivity}
An unstable nucleus that will spontaneously change into a lower-energy configuration is said to be \textbf{radioactive}. Usually happened for cases:
\begin{itemize}
	\item nuclei is too large
	\item bad neutron-to-proton ratio
\end{itemize}

\subsection{Alpha Decay}
Alpha decay emits an alpha particle $\sideset{^4_2}{}{\mathop{\mathrm{He}}}$, e.g
\[
\sideset{^{222}_{86}}{}{\mathop{\mathrm{Rn}}} \rightarrow \sideset{^{218}_{84}}{}{\mathop{\mathrm{Po}}}+
\sideset{^4_2}{}{\mathop{\mathrm{He}}}
\]

\subsection{Beta Decay}
There are $3$ types of $\beta$decay:
\begin{itemize}
	\item $\beta^-$ decay happens when neutron to proton ratio is too large, it is the most common $\beta$ decay. In this case, neutron transforms into a proton, electron and electron-antineutrino(caused by \textbf{weak nuclear force}):
	\[
	\sideset{^{14}_{6}}{}{\mathop{\mathrm{C}}} \rightarrow \sideset{^{14}_{7}}{}{\mathop{\mathrm{N}}}+
	\sideset{^0_{-1}}{}{\mathop{\mathrm{e}}}+\bar{\nu}_e
	\]
	\item $\beta^+$ decay happens when neutron to proton ratio is too small. In this case, proton transforms into a neutron, positron and electron-neutrino:
	\[
	\sideset{^{17}_{9}}{}{\mathop{\mathrm{F}}} \rightarrow \sideset{^{17}_{8}}{}{\mathop{\mathrm{O}}}+
	\sideset{^0_{+1}}{}{\mathop{\mathrm{e}}}+\nu_e
	\]
	\item electron capture happens when nucleus capture an orbiting electron:
	\[
	\sideset{^{7}_{4}}{}{\mathop{\mathrm{Be}}}+ \sideset{^0_{-1}}{}{\mathop{\mathrm{e}}} \rightarrow \sideset{^{7}_{3}}{}{\mathop{\mathrm{Li}}}+\nu_e
	\]
\end{itemize}

\subsection{Gamma Decay}
Gamma decay does not alter the identity of the nucleus, it just allows the nucleus to relax and shed energy when it is in excited status:
\[
\sideset{^{42}_{20}}{}{\mathop{\mathrm{Ca^*}}} \rightarrow \sideset{^{42}_{20}}{}{\mathop{\mathrm{Ca}}} + \gamma
\]

\subsection{Decay Rate}
The fraction(or proportion) of nuclei that decay per second, the \textbf{decay constant} $\lambda$(not wavelength), does not change during the process. The higher the constant, the faster the decay. And decay follow the equation:
\[
A = A_0 e^{\lambda t}
\]
which is exponential decay.
\begin{itemize}
	\item one disintegration per second is one \textbf{Becquerel}(Bq);
	\item  \textbf{half-life} is defined as the time period when $A = 0.5A$
\end{itemize}

\section{Nuclear Reaction}
It happens when:
\begin{itemize}
	\item radioactive decay happens
	\item bombardment of target nuclei with subatomic particles to artificially induce radioactivity(nuclear fission)
\end{itemize}
Here is one example:
\[
\sideset{^{1}_{0}}{}{\mathop{\mathrm{n}}}
\sideset{^{198}_{80}}{}{\mathop{\mathrm{Hg}}} \rightarrow \sideset{^{197}_{79}}{}{\mathop{\mathrm{Au}}}+\sideset{^{2}_{1}}{}{\mathop{\mathrm{H}}}
\]
Nuclear reaction involves absorption(\textbf{endothermic}) or emission(\textbf{exothermic}) of energy, which can be calculated by mass-energy equation with defect mass known.

\section{Special Relativity}
Two postulates:
\begin{itemize}
	\item all the laws of physics are the same in all inertial reference frames.
	\item the speed of light in vacuum always has the same value, regardless of the motion of the source or observer
\end{itemize}
An inertial reference frame is one in which Newton’s first law holds. Here is a list of formula:
\begin{itemize}
	\item relative velocity:
	\[
	V = \frac{u+v}{1+uv/c^2}
	\]
	\item relative time:
	\[
	T = \frac{1}{\sqrt{1 - (v/c)^2}} t = \gamma t
	\]
	define $\gamma$ as relativistic factor:
	\[
	\gamma = \frac{1}{\sqrt{1 - (v/c)^2}}
	\]
	\item relative length:
	\[
	L =\frac{l}{\gamma}
	\]
	\item relative kinetic energy:
	\[
	K = (\gamma-1) mc^2
	\]
	note that rest energy $E_R$ is:
	\[
	E_R = mc^2
	\]
	total energy is:
	\[
	E = E_R + K = \gamma m c^2
	\]
\end{itemize}
Summary: mass increased, length is compressed, time is slowed.
\end{document}