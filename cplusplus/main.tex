\documentclass[10pt,a4paper,oneside]{article}
\usepackage[utf8]{inputenc}
\usepackage[T1]{fontenc}
\usepackage{amsmath}
\usepackage{amsfonts}
\usepackage{amssymb}
\usepackage{graphicx}
\usepackage{enumerate}
\usepackage{listings}
\newcommand{\tabincell}[2]{\begin{tabular}{@{}#1@{}}#2\end{tabular}}
\usepackage{xcolor}
\definecolor{CPPLight}  {HTML} {686868}
\definecolor{CPPSteel}  {HTML} {888888}
\definecolor{CPPDark}   {HTML} {262626}
\definecolor{CPPBlue}   {HTML} {4172A3}
\definecolor{CPPGreen}  {HTML} {487818}
\definecolor{CPPBrown}  {HTML} {A07040}
\definecolor{CPPRed}    {HTML} {AD4D3A}
\definecolor{CPPViolet} {HTML} {7040A0}
\definecolor{CPPGray}  {HTML} {B8B8B8}
\lstset{
	basicstyle=\ttfamily,
	columns=fullflexible,
	breakatwhitespace=false,      
	breaklines=true,
	captionpos=b,
	columns=fixed,       
	%numbers=left,                                        % 在左侧显示行号
	frame=single,                                          % 不显示背景边框
	%backgroundcolor=\color[RGB]{245,245,244},            % 设定背景颜色
	keywordstyle=\color{blue},                % 设定关键字颜色
	numberstyle=\footnotesize\color{darkgray},           % 设定行号格式
	commentstyle=\color[RGB]{0,96,96},                % 设置代码注释的格式
	stringstyle=\color[RGB]{128,0,0},   % 设置字符串格式
	showstringspaces=false,                              % 不显示字符串中的空格
	language=c++,                                       % 设置语言
	morekeywords={alignas,continute,friend,register,true,alignof,decltype,goto,
		reinterpret_cast,try,asm,defult,if,return,typedef,auto,delete,inline,short,
		typeid,bool,do,int,signed,typename,break,double,long,sizeof,union,case,
		dynamic_cast,mutable,static,unsigned,catch,else,namespace,static_assert,using,
		char,enum,new,static_cast,virtual,char16_t,char32_t,explict,noexcept,struct,
		void,export,nullptr,switch,volatile,class,extern,operator,template,wchar_t,
		const,false,private,this,while,constexpr,float,protected,thread_local,
		const_cast,for,public,throw,std},
	emph={map,set,multimap,multiset,unordered_map,unordered_set,
		unordered_multiset,unordered_multimap,vector,string,list,deque,
		array,stack,forwared_list,iostream,memory,shared_ptr,unique_ptr,
		random,bitset,ostream,istream,cout,cin,endl,move,default_random_engine,
		uniform_int_distribution,iterator,algorithm,functional,bing,numeric,},
	emphstyle=\color{CPPViolet}, 
}
    
\date{June 25, 2019}
\author{Yangang Cao}
\title{C{}\texttt{++} Q\&A}
\begin{document}
\maketitle
\tableofcontents

\newpage
\section{Basic Type}
\begin{center}
\begin{tabular}{cc}
	\hline
	Type& Size in Xcode\\
	\hline
	bool& 1\\
	char& 1\\
	wchar\_t&4\\
	char16\_t&2\\
	char32\_t&4\\
	short&2\\
	int&4\\
	long&8\\
	long long&8\\
	float&4\\
	double&8\\
	long double&16\\
	\hline
\end{tabular}
\end{center}
and the code
\begin{lstlisting}
#include <iostream>
using namespace std;

int main()
{
cout << "bool:" << sizeof(bool) << endl;
cout << "char:" << sizeof(char) << endl;
cout << "wchar_t:" << sizeof(wchar_t) << endl;
cout << "char16_t:" << sizeof(char16_t) << endl;
cout << "char32_t:" << sizeof(char32_t) << endl;
cout << "short:" << sizeof(short) << endl;
cout << "int:" << sizeof(int) << endl;
cout << "long:" << sizeof(long) <<  endl;
cout << "long long:" << sizeof(long long) << endl;
cout << "float:" << sizeof(float) << endl;
cout << "double:" << sizeof(double) << endl;
cout << "long double:" << sizeof(long double) << endl;

return 0;
}
\end{lstlisting}
\section{Pointer}
\begin{enumerate}[1.]
\item Why do we need pointer?
\begin{itemize}
\item 8 byte for any type in 64-bit system, 4 byte in 32-bit system
\item CallByPtr can change the parameters or not
\item Using ``new'' to allocate memory for dynamic memory
\item Faster than callByVal for comlpex type
\end{itemize}
\item How to allocate and free memory using ``new'' and ``delete''?\\
\begin{lstlisting}
typeName * pointerName = new typeName;

delete pointerName;
\end{lstlisting}
``delete'' will free the memory, but not delete the pointer.
\begin{lstlisting}
int *p = new int; // return memory address
*p = 10; // assignment directly
delete p; // free memory
int a =3;
*p = &a; // OK
\end{lstlisting}
\end{enumerate}
\section{Overload}
\begin{enumerate}[1.]
\item When do we use overload?\\
Funtion with different parameters types
\item What's the key of overload?\\
Function signature, contains three points: number, type and order of parameters, complier realize overload by name mangling
\item How about reference in overload?\\
Type and type reference share one function signature in complier, error will be raised if both of them exist
\item What happen if no funtion signature matches?\\
Compiler attempt to cast it to funtion signature already exists, if more than one ways it can cast, error will be raised
\item How about keyword ``const'' in function signature ?\\
Matching the corresponding function signature, if no match,  cast will happen, ``const'' signature can match modifiable and const parameter, however, signature without ``const'' only match modifiable parameter.
\end{enumerate}
\section{Storage and Static}
How many porperties used to desrcibe storage modes in C\texttt{++}?
\begin{enumerate}[1.]
\item Storage persistence:
\begin{itemize}
\item Automatic storage persistence: variables defined in functions
\item Static storage persistence: variables defined ouside functions, and defined by ``static'' 
\item Thread storage persistence: parallel programming
\item Dynamic storage persistence: the address of variables are allocated by ``new''
\end{itemize}
\item Scope
\begin{itemize}
	\item Global: normal functions
	\item Local: auto variable
	\item function prototype scope: function prototype
	\item Namespace
	\item Class
\end{itemize}
\item Linkage
\begin{itemize}
	\item External: shared between files, need ``extern'' in other files
	\item Internal: shared in functions in one file
	\item None: just in one function in ine file, such as: auto variables
\end{itemize}
\end{enumerate}
A example code:
\begin{lstlisting}
...
int global = 1000; // static dutation, global, external linkage
static int one_file = 50; // static duration, global, internal linkage
const int fingers = 10; // static duration, global, internal linkage
extern const int states = 50; // static duration, global, external linkage
int main()
{
...
}

void funct1(int n) // static duration, global, external linkage
{
    static int count = 0; // static dutation, local,no linkage
    int llama = 0; // automatic dutation, local, no linkage, 
...
}

static void funct2(int n) // static duration, global, internal linkage
{
...
}
\end{lstlisting}

\begin{table}

\centering
\begin{tabular}{ccc}
	\hline
	Porperties &\tabincell{c}{Static\\member\\functions}&\tabincell{c}{Normal\\member\\functions}\\
	\hline
    Count& Only one& One per object\\
    Depend on object? & No & Yes\\
    Have ``this'' pointer?&No&Yes\\
	Call by ``class::function''?&Yes&No\\
	Call by ``object.function''?&Yes&Yes\\
	Assess normal member variables?&No&Yes\\
	Assess static member variables?&Yes&Yes\\
	\hline
\end{tabular}
\caption{Static and normal member functions in class}
\end{table}
\section{Function Template}
\begin{enumerate}[1.]
\item How to understand function template?\\
Function with same algrithm for different type parameters
\begin{lstlisting}
// function template prototype
template <typename T>
void swap(T &a, T &b)

// function template definition
template <typename T>
void swap(T &a, T &b)
{
  T temp;
  temp = a;
  a = b;
  b = temp;
}
\end{lstlisting}
\item How to understand overloaded function template?\\
Function with different algrithms for different type parameters
\begin{lstlisting}
// original template prototype
template <typename T>
void swap(T &a, T &b);

// new template prototype
template <typename T>
void swap(T *a, T *b, int n);

\end{lstlisting}
\item How to understand explicit specialization?\\
Provides specific template definitions for specific types, prior to normal templates
\begin{lstlisting}
// explicit specialization for the job type
template <>
void swap(job &a, job &b);

\end{lstlisting}
\item How to understand explicit instantiation?\\
Generate a function from template directly, less detection, more efficient
\begin{lstlisting}
// explicit instantiation for the job type in main()
template 
void swap(char &a, char &b);

\end{lstlisting}
\end{enumerate}
\section{Copy Contructor}
\begin{enumerate}[1.]
\item How to use copy contructor?\\
If there is a pointer variable initialized by ``new'' in class, {\bfseries deep copy} should be set in copy constructor and ``='' operator overload.
\begin{lstlisting}
StringBad::StringBad(const StringBad & st)
{
num_string++;
len = st.len;
str = new char [len + 1];
std::strcpy(str, st.str);
}
\end{lstlisting}
\begin{lstlisting}
StringBad & StringBad::operator=(const StringBad & st) //  e.g. S0 = "fox", tip: = operator overload don't change num_string
{
if (this == &st) // in case S0 = S0
return *this; // directly return
delete [] str; // free S0 str
len = str.len;
str = new char [len + 1]; // get space for new string
std::strcpy(str, st.str);
return *this;
}
\end{lstlisting}
4 statements will call copy constructor.
\begin{lstlisting}
StringBad ditto(motto);
StringBad ditto = motto;
StringBad ditto = Stringbad(motto);
StringBad * ditto = new StringBad(motto);
\end{lstlisting}
Some tips:
\begin{itemize}
\item CallByVal and ReturnObj will call copy constructor(they both creates temporaty variable), we'd better to use reference.
\item If there is a static data member in class, its value will change when new object is created, a explicit copy constructor should be set.
\begin{lstlisting}
String::StringBad(const StringBad &s)
{
  num_strings++;
  ...// other code
}
\end{lstlisting}
\item Default constructor in dynamic memoty:
\begin{lstlisting}
String::String()
{
  len = 0;
  str = new char[1]; // Compatible with destructor "delete [] str"
  str[0] = `\0' ;
}
\end{lstlisting}
\item Static member functions can't use ``this'' pointer, non-static data member.
\begin{lstlisting}
static int HowMant() {return num_strings;} //declaration

int count = String::HowMany(); // how to use
\end{lstlisting}
\end{itemize}
\end{enumerate}
\section{Return Object}
Containing 4 ways of returning object.
\begin{enumerate}[1.]
\item when we return ``const ClassName \&''\\
Return the object in parameters list
\begin{lstlisting}
const Vector & Vector::Max(const Vector & v1, const Vector & v2)
{
  if (v1.magval() > v2.magval())
    return v1;
  else
    return v2;
}
\end{lstlisting}
Three points:
\begin{itemize}
	\item Returning object will call copy constructor, but object reference won't.
	\item The object should exists when we return the object reference.
	\item If all the parameters is ``const ClassName \&'', one of them should be returned, we must return ``const ClassName \&''.
\end{itemize}
\item when we return ``ClassName \&''?\\
2 conditions:
\begin{itemize}
\item overload assignment operator ``='', more efficient
\begin{lstlisting}
String & operator =(const String &);
String & operator =(const char *);

String s1("good stuff");
String s2, s3;
s1 = s2 = s3; // s2=s3 will change s2
\end{lstlisting}
\item overload output operator ``$\mathbf{<<}$'', must obey
\begin{lstlisting}
ostream & operator <<(ostream & os, const String & s1);
\end{lstlisting}
\end{itemize}
\item When we return ``ClassName''?\\
Return local variables, such as overload ``+'', ``-'', ``*''
\begin{lstlisting}
Vector Vector::operator+(const Vector &b) const
{
  return Vector(x+b.x, y+b.y);
} 
\end{lstlisting}
\item When we return ``const ClassName''?\\
If we have overloaded operator ``+'' using 3, a strange property appears:
\begin{lstlisting}
net = force1 + force2; // OK
force1 + force2 = net; // also OK
\end{lstlisting}
When we want to forbid the second statement, ``const ClassName'' should be used
\begin{lstlisting}
const Vector Vector::operator+(const Vector &b) const
{
return Vector(x+b.x, y+b.y);
} 
\end{lstlisting}
\end{enumerate}
\section{Class Inheritance(is-a)}
This is a base class.
\begin{lstlisting}
#include <string> // tabtenn0.h
using std::string;

class TableTennisPlayer
{
private:
    string firstname;
    string lastname;
    bool hasTable;
public:
    TableTennisPlayer(const string &fn = "none", const string &ln = "none", bool ht = false);
    void Name() const;
    bool HasTable() const {return hasTable;};
    void ResetTable(bool v) {hasTable = v;};
}
\end{lstlisting}
\begin{lstlisting}
#include "tabtenn0.h" // cpp file
#include <iostream>

class TableTennisPlayer
{
    TableTennisPlayer::TableTennisPlayer(const string &fn, const string &ln, bool ht): firstname(fn), lastname(ln), hasTable(ht) {} // efficient
    
    void TableTennisPlayer::Name() const
    {
        std::cout << lastname << "," << firstname;
    }
}
\end{lstlisting}
RatedPalyer class derives from the TableTennisPlayer base class.
\begin{lstlisting}
class RatedPlayer: public TableTennisPlayer // tabtenn1.h
{
private:
    unsigned int rating; // add a new data member
public:
    RatedPlayer(unsigned int r = 0, const string &fn = firstname, const string &ln = lastname, bool ht = false); // new constructor
    RatedPlayer(unsigned int r, const TableTennisPlayer & tp); // new constructor
    unsigned int Rating() const {return rating;} // new member function
    void ResetRating (unsigned int r) {rating = r;} //new member function
};
\end{lstlisting}
The first constructor:
\begin{lstlisting}
RatedPlayer::RatedPlayer(unsigned int r, const string &fn, const string &ln, bool ht): TableTennisPlayer(fn, ln, ht)
{
    rating = r;
}
\end{lstlisting}
or
\begin{lstlisting}
RatedPlayer::RatedPlayer(unsigned int r, const string &fn, const string &ln, bool ht): TableTennisPlayer(fn, ln, ht), rating(r)
{
}
\end{lstlisting}
The second constructor:
\begin{lstlisting}
RatedPlayer::RatedPlayer(unsigned int r, const TableTennisPlayer &tp): TableTennisPlayer(tp)
{
    rating = r;
}
\end{lstlisting}
 or
 \begin{lstlisting}
 RatedPlayer::RatedPlayer(unsigned int r, const TableTennisPlayer &tp): TableTennisPlayer(tp),  rating(r)
 {
 }
 \end{lstlisting}
 Above all, 3 points can be obtained about constructor of derived class:
 \begin{itemize}
 \item Complier creates base class object before creating derived class object, the constructor of base class will called firstly as well
 \item Initialization list should be used to deliver the parameters from derived class to base in derived class construtors
 \item Derived class constructors should initial new data member
 \end{itemize}
{\bfseries Special relationship between base and drived class:}
\begin{itemize}
\item Derived class objects can call public base class methods
 \begin{lstlisting}
RatedPlayer rplayer(1140,"Mallory", "Duck", true);
rplayer.Name(); // OK
\end{lstlisting}
\item Base class pointer and reference can point to derived class object, but not vice versa
\begin{lstlisting}
RatedPlayer rplayer(1140,"Mallory", "Duck", true);
TableTennisPlayer &rt = rplayer; // OK
TableTennisPlayer *pt = &rplayer; //OK

TableTennisPlay player("Besty", "Bloop", true);
RatedPlayer &rr = player; // not OK
RatedPlayer *pr = &player; // not OK
\end{lstlisting}
\item The base class pointer and reference which point to derived class can't invoke the derived class methods
\begin{lstlisting}
RatedPlayer rplayer(1140,"Mallory", "Duck", true);
TableTennisPlayer &rt = rplayer;
TableTennisPlayer *pt = &rplayer;
rt.Name(); // OK
pt->Name(); // OK
rt.Rating(); // not OK
pt->Rating(); // not OK
\end{lstlisting}
\item If we set a base class pointer or reference on parameter, both base and derived class object pointer or reference can be appected as argument
\begin{lstlisting}
void Show(const TableTennisPlayer &rt)

TableTennisPlayer Player("Tara","Boomdea", flase);
RatedPlayer rplayer(1140,"Mallory", "Duck", true);
Show(player); // OK
Show(rplayer); // OK
\end{lstlisting}
\item We can initial base class by derive class and assign a deirve class to base class
\begin{lstlisting}
RatedPlayer olaf1(1140,"Mallory", "Duck", true);
TableTennisPlayer olaf2(olaf1); // initialization
\end{lstlisting}
\begin{lstlisting}
RatedPlayer olaf1(1140,"Mallory", "Duck", true);
TableTennisPlayer winner
winner = olaf1; // assignment
\end{lstlisting}
\end{itemize}
\section{Public Inheritance Polymorphism}
Two kinds of polymorphism: redefining and virtual method
\begin{lstlisting}
class Brass // base class
{
...
public:
    void ViewAcct() const;
    virtual void Withdraw(double amt);
    virtual ~Brass() {};
...
};

class BrassPlus: public Brass // derived class
{
...
public:
    void ViewAcct() const; // redefining
    virtual void Withdraw(double amt); // vitual method
...
}
\end{lstlisting}
Polymorphism is associated with reference and pointer. The redefined member function is chosen according to the type of reference and pointer, the virtual member function is chosen according to the type of reference and pointer point to. Both of them is chosen according to the object type when object calls them directly.
\begin{lstlisting}
Brass dom ("Dominic Banker", 11224, 4183.45);
BrassPlus dot("Dorothy Banker", 12118, 2592.00);

// Objects call redefining and virtual method directly
dom.ViewAcct(); // use Brass::ViewAcct()
dot.ViewAcct(); // use BrassPlus::ViewAcct()

dom.Withdraw(); // use Brass::Withdraw()
dot.Withdraw(); // use BrassPlus::Withdraw()

// References call redefining method, as well pointer
Brass & b = dom;
Brass & bp = dot;
b.ViewAcct(); // use Brass::ViewAcct()
bp.ViewAcct(); // use Brass::ViewAcct()

// Reference call virtual method, as well pointer
Brass & b = dom;
Brass & bp = dot;
b.Withdraw(); // use Brass::Withdraw()
bp.Withdraw(); // use BrassPlus::Withdraw()
\end{lstlisting}
\section{Protected}
\begin{lstlisting}
class Brass
{
...
protected:
    double balance
...
}
\end{lstlisting}
The ``protected'' members in base class are same as ``public'' members to derived class, ``private'' members to external class, however, ``protected'' data members are useless, only ``protected'' member functions are useful.
\section{Containment and Private Inheritance(has-a)}
Two kinds of ``has-a'' relationship: containment and private inheritance.\\
Containment (composition or layering) means that data members in class are objects of anothor class. Public iheritance can get API, containment can't get API, it's a part of ``has-a''.
\begin{lstlisting}
class Student
{
private:
    string name; // use a string object for name
    valarray<double> scores; // use a valarray<double> object for scores
...
}
\end{lstlisting}
Private inheritance means than all members in base class become private in derived class.
\begin{lstlisting}
class Student: private std::string, private std::valarray<double>
{
public:
...
}
\end{lstlisting}
\section{Mutiple Inheritance(is-a)}
Mutiple inheritance means that derived from more than one base class, naturally, some problems appear.
\begin{lstlisting}
class SingingWaiter: public Waiter, public Singer {...};
class SingingWaiter: public Waiter, Singer {...}; // Singer is private base
\end{lstlisting}
To solve some probems in MI, virtual base class is introduced, it makes a class which is derived from multiple classes (they share the same base class) inherite only one base class object.
\begin{lstlisting}
class Singer: virtual public Worker {...};
class Waiter: public virtual Worker {...}; // order changing is OK

class SingingWaiter: public Singer, public Waiter {...};
\end{lstlisting}
To no-virtual base class, the construtors maybe like this:
\begin{lstlisting}
class A
{
    int a;
public:
    A(int n = 0): a(n) {}
    ...
}

class B: public A
{
    int b;
public:
    B(int m = 0, int n = 0): A(n), b(m) {}
    ...
}

class C: public B
{
    int c;
public:
    C(int q = 0, int m = 0, int n = 0): B(m, n), c(q) {}
    ...
};
\end{lstlisting}
To virtual base class, however, it not works:
\begin{lstlisting}
SingingWaiter(const Worker &wr, int p = 0, int v = Singer::other): Waiter(wk, p), Singer(wk, v) {} // not OK
\end{lstlisting}
It must be like this: (it' not ok for no-virtual base class)
\begin{lstlisting}
SingingWaiter(const Worker &wr, int p = 0, int v = Singer::other): Waiter(wk), Waiter(wk, p), Singer(wk, v) {} // OK
\end{lstlisting}
Which method will be used?\\
It's ambiguous:
\begin{lstlisting}
SingingWaiter newhire("Elise Hawks",2005, 6, soprano);
newhire.Show(); // ambiguous
\end{lstlisting}
We can use this:
\begin{lstlisting}
SingingWaiter newhire("Elise Hawks",2005, 6, soprano);
newhire.Signer::Show();
\end{lstlisting}
A better way:
\begin{lstlisting}
void SingingWaiter::Show()
{
    Singer::Show();
}
\end{lstlisting}
\section{Class Template}
Class templates are useful to containers, the grammer is shown in following code:
\begin{lstlisting}
template <typename T>
class Stack
{
private:
    T items[10];
    ...
public:
    Stack();
    bool pop(T &item);
    ...
};

template <typename T>
Stack<T>::Stack()
{
...
}

template <typename T>
bool Stack<T>::pop(T & item)
{
...
}
...

Stack<int> kernels;
Stack<string> colonels;
\end{lstlisting}
Specialization of class template (similar as function template):
\begin{lstlisting}
ArrayTP<int, 100> stuff; // a example class template
\end{lstlisting}
\begin{itemize}
\item Inplicit instantiation: generate a class declaration only when an object is needed.
\begin{lstlisting}
ArrayTP<double, 30> *pt; // a pointer, no object needed yet
pt = new ArrayTP<double, 30>; // now an object is needed
\end{lstlisting}
\item Explicit instantiation: generate a class declaration directly
\begin{lstlisting}
template
class ArrayTP<string, 100>;
\end{lstlisting}
\item Explicit specialization: declaration for specialized type
\begin{lstlisting}
template <>
class ArrayTP<specialized type>
{
...
};
\end{lstlisting}
\item Partial specializaiton: limitting partial generality of template
\begin{lstlisting}
template <typename T1, typename T2> class Pair {...}; // general template
template<typename T1> class Pair<T1, int> {...}; // partial specialization, set T2 to int

// if we do this, it becomes  explicit specialization
template<> class Pair<int, int> {...};
\end{lstlisting}
\end{itemize}
\section{Friend  in Class}
\begin{itemize}
\item {\bfseries Friend class}: All the member functions in friend class can assess private and protect members in original class.
\begin{lstlisting}
class TV
{
public:
    friend class Remote; // can also put elsewhere
    ...
};

class Remote
{
// something about TV, so the class Remote declaration should behind of class TV, unless forward declaration
};
\end{lstlisting}
\item {\bfseries Friend member function}: Sometime, we want to only some member functions rather than all of them become the friend functions of original class, friend member function is introduced.
\begin{lstlisting}
class TV
{
friend void Remote::setChan(Tv &t, int c); // friend member function
}
\end{lstlisting}
{\bfseries VIP}: If we use friend member function, class $Remote$ should be defined before class TV, otherwise the complier doesn't know $Remote$ is a class. In the other hand, the parameters of $Remote::setChan$ contain reference of class $Tv$, so the class $Tv$ should be defined first. To solve this problem, {\bfseries forward declaration} is introduced.
\begin{lstlisting}
// Must obey the following order !!!
class Tv; // forward declaration
class Remote {...}; // just declaration, no definition
class Tv {...}; // declaration 
... // class Remote definition, inline or something
\end{lstlisting}
\item {\bfseries Friend mutually}:
\begin{lstlisting}
class Tv
{
friend class Remote;
....
};

class Remote
{
friend class Tv;
....
};
\end{lstlisting}
\item {\bfseries Common friend}:
\begin{lstlisting}
class Analyzer; // forward declaration

class Probe
{
    friend void sync(Analyzer &a, const Probe &p); // sync a to p
    friend void sync(Probe &p, const Analyzer &a); // sync p to a
    ...
};

class Analyzer
{
    friend void sync(Analyzer &a, const Probe &p); // sync a to p
    friend void sync(Probe &p, const Analyzer &a); // sync p to a
    ...
};

// define the friend functions
inline void sync(Analyzer &a, const Probe &p)
{
...
}
inline void sync(Probe &p, const Analyzer &a)
{
...
}
\end{lstlisting}
\end{itemize}
\end{document}