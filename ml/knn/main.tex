\documentclass[10pt,a4paper,oneside]{article}
\usepackage[utf8]{inputenc}
\usepackage{amsmath}
\usepackage{amsfonts}
\usepackage{amssymb}
\usepackage{graphicx}
\usepackage{breqn}
\usepackage{pythonhighlight}
\author{Baboo J. Cui}

\begin{document}

\title{KNN}
\maketitle
\tableofcontents

\newpage
\section{Introduction}
KNN is short for K-nearest-neighbor:
\begin{itemize}
	\item supervised learning
	\item very easy
	\item classification is based on measuring distance between characteristics
\end{itemize}
Its advantages:
\begin{itemize}
	\item accurate(in terms of algorithm not result)
	\item insensitive to extreme values
	\item no training process
\end{itemize}
Its disadvantages:
\begin{itemize}
	\item computational costly(traverse the whole data set)
	\item time consuming
	\item high requirement for storage
	\item data must be in the same dimension
\end{itemize}

\section{Ideology}
Given a data set with label on each element, when a new sample arrives, find \textbf{top} $k$ elements in set that are \textbf{closest }to it. The label for new sample is simply set to be the mass label of the $k$ elements.
\begin{itemize}
	\item $k$ usually is less than $20$
	\item $k$ should be odd number
\end{itemize}

\section{Algorithm}
\textbf{KNN algorithm} is as follow:
\begin{enumerate}
	\item calculate \textbf{distance}(usually $\mathcal{L}_2$, may be others) between coming sample and all elements in data set
	\item sorting data set by distance from closest to farthest
	\item pick up the first $k$ elements in data set
	\item get the label frequency in $k$ elements
	\item the prediction is the label with highest frequency
\end{enumerate}
Here is a list of tricky situation:
\begin{itemize}
	\item when label is not number, assign each label with a number for convenience
	\item when dynamic range of each characters differs too much, normalization is necessary, it can be done by:
	\[
	\text{newVal} = \frac{\text{oldVal} - \min}{\max - \min}
	\]  
\end{itemize}

\section{Example}
Here is a quick example of KNN for finding the type of point:
\inputpython{bc_knn.py}{1}{32}
This will print "B" as expected.
\end{document}
