\documentclass[10pt,a4paper,oneside]{article}
\usepackage[utf8]{inputenc}
\usepackage{amsmath}
\usepackage{amsfonts}
\usepackage{amssymb}
\usepackage{graphicx}
\usepackage{breqn}
\author{Baboo J. Cui}

\begin{document}

\title{Introduction}
\maketitle

Welcome to machine learning course from Baboo J. Cui. This course is based on many materials. So just enjoy it!

\section{Types of Learning}
Based on the amount and type of supervision they get during training
\begin{itemize}
	\item Supervised learning: when data for training contains expected outcomes(known as labels)
	\item Unsupervised learning: when data for training don't contain labels
	\item Semi-supervised learning
	\item Reinforcement learning
\end{itemize}

\section{Format of ML}
\begin{itemize}
	\item Input(features): denoted by $x^{(i)}$, where $i$ is index, not exponent
	\item Output(target): denoted by $y^{(i)}$, this is what we need to predict
	\item Training example: a pair of input and output $(x^{(i)}, y^{(i)})$, note that $x$, $y$  and $\theta$ can all be vectors
	\item Training set: a list of training example, let's say it has length of $m$: $\{(x^{(i)}, y^{(i)}); i \in [1,m]\}$
	\item Space notation: use $\mathcal{X}$ and $\mathcal{Y}$ to represent space of input and output
	\item Hypothesis: a function $h:\mathcal{X} \mapsto \mathcal{Y}$, so that $h(\cdot)$ can predict $y$ given $x$ Function $h$ can be obtained by putting training set into learning algorithm
	\item Regression: when target variable is continuous
	\item Classification: when target variable is discrete
\end{itemize}

\end{document}
