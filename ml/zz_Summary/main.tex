\documentclass[10pt,a4paper,oneside]{article}
\usepackage[utf8]{inputenc}
\usepackage{amsmath}
\usepackage{amsfonts}
\usepackage{amssymb}
\usepackage{graphicx}
\usepackage{breqn}
\author{Baboo J. Cui}

\begin{document}
\title{Machine Learning Summary}
\maketitle
\newpage
\tableofcontents

\newpage
\section{Introduction}
In reality, is it possible to predict house price based on features of the house? Suppose size of the house, location are known.

\subsection{Types of Learning}
Based on the amount and type of supervision they get during training
\begin{itemize}
	\item Supervised learning: when data for training contains expected outcomes(known as labels)
	\item Unsupervised learning: when data for training don't contain labels
	\item Semi-supervised learning
	\item Reinforcement learning
\end{itemize}

\subsection{Format of ML}
\begin{itemize}
	\item Input(features): denoted by $x^{(i)}$, where $i$ is index, not exponent
	\item Output(target): denoted by $y^{(i)}$, this is what we need to predict
	\item Training example: a pair of input and output $(x^{(i)}, y^{(i)})$, note that $x$, $y$  and $\theta$ can all be vectors
	\item Training set: a list of training example, let's say it has length of $m$: $\{(x^{(i)}, y^{(i)}); i \in [1,m]\}$
	\item Space notation: use $\mathcal{X}$ and $\mathcal{Y}$ to represent space of input and output
	\item Hypothesis: a function $h:\mathcal{X} \mapsto \mathcal{Y}$, so that $h(\cdot)$ can predict $y$ given $x$ Function $h$ can be obtained by putting training set into learning algorithm
	\item Regression: when target variable is continuous
	\item Classification: when target variable is discrete
\end{itemize}

\newpage
\section{Linear Regression}
Approximate output $y$ as a function of inputs(features) $x$, note that both $x$ and $y$ can be vectors.
\[
\hat{y} = h_\theta(x) = \theta_0 + \theta_1 x_1 + \theta_2 x_2 + \cdots
\]
\begin{itemize}
	\item parameters(weights): $\theta_i$, which control the mapping from $\mathcal{X}$ to $\mathcal{Y}$.
	\item subscript notation: $h_\theta$ means function $h$ is parametrized by $\theta$
	\item hat notation: $\hat{\cdot}$ represent the estimation
\end{itemize}
By convention, set $x_0 = 1$, so that
\[
\hat{y} = h_\theta (x) = \sum_{i=0}^{n} \theta_i x_i = \theta^T x = x^T \theta
\] 
\begin{itemize}
	\item intercept term: $x_0$ here is known as intercept term
	\item vector dimension: by default, all vectors are column vectors
	\item the number of input is $n$ instead of $n+1$, the number of parameter is $n+1$
\end{itemize}
Cost function is defined as
\[
J(\theta) = \frac{1}{2} \sum_{i=1}^{m} (h_\theta(x^{(i)}) - y^{(i)})^2
\]
\begin{itemize}
	\item $J(\theta)$ is known as least-square cost function, lead to ordinary least square regression model
	\item  $m$ is the length of training list
\end{itemize}
We need to optimize $\theta$ to minimize $J(\theta)$.

\subsection{Least Mean Square(LMS) Algorithm}
Gradient descent algorithm can find a value in domain to minimize a function, for general scalar function $y = f(x)$, by setting
\[
x := x - \alpha \frac{d}{dx}f(x)
\]
a value of $x$ can be found to minimize $f(x)$, note that:
\begin{itemize}
	\item $:=$ represent assignment
	\item $x$ can be a vector, if so, take partial derivative to all variables(next part)
	\item $\alpha$ is called rate(or learning rate in ML), it is positive
	\item gradient is the deepest increase direction, so its negative is the deepest decrease direction, in this case, the gradient is a scalar
\end{itemize}
For vector case, let's say $y = f(\textbf{x})$, where $x \in \mathbb{R}^n$, the gradient of function $f$ is denoted by $\nabla f$, and is defined as
\[
\nabla f = \nabla f(\textbf{x}) = \begin{bmatrix}
\frac{\partial f}{\partial x_1} & \frac{\partial f}{\partial x_2} & \cdots &  \frac{\partial f}{\partial x_n}
\end{bmatrix}^T
\]
the minimization process can be compactly written as
\[
\textbf{x} := \textbf{x} - \alpha \nabla f(\textbf{x})
\]
To apply gradient descent method to $J(\theta)$
\[
\theta := \theta - \alpha \nabla J(\theta)
\]
to get $\nabla J$, use chain rule for each element
\begin{align*}
\frac{\partial J(\theta)}{\partial \theta_j} &= \frac{\partial \quad  \frac{1}{2} \sum_{i=1}^{m} (h_\theta(x^{(i)}) - y^{(i)})^2}{\partial \theta_j}\\
&= 2 \times \frac{1}{2}  \sum_{i=1}^{m} \left((h_\theta(x^{(i)}) - y^{(i)}) \frac{\partial (h_\theta(x^{(i)}) - y^{(i)})}{\partial \theta_j}  \right)\\
&= \sum_{i=1}^{m} \left((h_\theta(x^{(i)}) - y^{(i)}) x_j^{(i)}  \right)
\end{align*}
\begin{itemize}
	\item this is the LMS update rule
	\item this is called batch gradient descent since it looks every example, bad computational cost
	\item this algorithm is sensitive to initial condition, may lead to different local minimum. it always converge to global minimum if function is convex
	\item $J$ is a convex quadratic function, which is good
\end{itemize}
Instead of looking for all element, if the parameter is updated just based on new coming example, the cost function can be simplified as
\[
J(\theta) = h_\theta((x^{(i)}) - y^{(i)})^2
\]
and each update will be simpler. This is known as stochastic gradient descent method.
\begin{itemize}
	\item stochastic algorithm is computationally faster than batch gradient
	\item stochastic algorithm coverages faster than batch gradient
	\item stochastic algorithm may oscillate and never reach the minimum
	\item stochastic algorithm is preferred when training set is large 
\end{itemize}

\subsection{Normal Equation}
Normal equation can explicitly find the solution to LMS problem by setting derivative of cost function to zero. Define design matrix 
\[
X \in \mathbb{R}^{m \times (n+1)}
\]
\begin{itemize}
	\item each row of $X$ represents $n$ features of one training example with intercept $1$, all together $n+1$ element
	\item the number of row of $X$ represent $m$ training examples
\end{itemize}
Let $y$ be the target value set so that 
\[
y \in \mathbb{R}^{m \times 1}
\]
Let $\theta$ be the parameter set so that
\[
\theta \in \mathbb{R}^{(n+1) \times 1}
\]
It should be clear that
\[
J(\theta) = \frac{1}{2}(X\theta-y)^T (X\theta-y)
\]
By properties of matrix we have
\[
\nabla _\theta J(\theta) = X^T X\theta - X^T y
\]
Set it to be zero and we can get normal equation
\[
X^T X\theta = X^T y
\]
So the closed form of $\theta$ can be found as
\[
\theta = (X^T X)^{-1} X^T y
\]
This is also known as pseudo inverse in linear algebra.

\subsection{Probabilistic Interpretation}
Suppose target variable and inputs are related via
\[
y^{(i)} = \theta^T x^{(i)} + \epsilon^{(i)}
\] 
where $\epsilon^{(i)}$ are the errors and are independently and identically distributed(IID) as
\[
\epsilon^{(i)} \sim \mathcal{N}(0, \sigma^2)
\]
Recall multivariate normal distribution is
\[
x \sim \mathcal{N} (\mu, \Sigma)= \frac{1}{\sqrt{(2\pi)^k|\Sigma|}} \exp \left( -\frac{1}{2}(x-\mu)^T \Sigma ^{-1} (x-\mu)\right)
\]
where $\Sigma$ is the covariance matrix, which is symmetric and positive definite. So the density function of $\epsilon$ is given by
\[
p(\epsilon^{(i)}) = \frac{1}{\sqrt{2\pi} \sigma} \exp \left(-\frac{(\epsilon^{(i)})^2}{2\sigma^2} \right)
\]
This implies that
\[
p(y^{(i)}|x^{(i)}; \theta) = \frac{1}{\sqrt{2\pi} \sigma} \exp \left(-\frac{( y^{(i)} - \theta^T x^{(i)})^2}{2\sigma^2} \right)
\]
which is equivalent to
\[
\left( y^{(i)} | x^{(i)}; \theta \right) \sim \mathcal{N}(\theta^Tx^{(i)}, \sigma^2)
\]
Bringing back design matrix $X$, and the parameter $\theta$, the probability of $y$ is denoted as
\[
p(y | X; \theta)
\]
Note that in this case, a bad $\theta$ will cause horrible deviation from $y$ to $\hat{y}$. When we wish to explicitly view this as a function of $\theta$, it is called likelihood function:
\[
L(\theta) = L(\theta; X, y) = p(y | X; \theta)
\]
Based on the assumption on $\epsilon^{(i)}$, $L$ can be written as:
\begin{align*}
	L(\theta) &= \prod_{i=1}^{m}  p(y^{(i)} | x^{(i)}; \theta)\\
	&= \prod_{i=1}^{m}   \frac{1}{\sqrt{2\pi} \sigma} \exp \left(-\frac{( y^{(i)} - \theta^T x^{(i)})^2}{2\sigma^2} \right)
\end{align*}
Clearly, from the principle of maximum likelihood, $\theta$ should be chosen to maximize $L(\theta)$. Maximizing a function can be achieved by maximizing any strictly increasing function of that function, and we choose to take log likelihood $l(\theta)$:
\begin{align*}
	l(\theta) &= \ln L(\theta)\\
	&= \ln \prod_{i=1}^{m}   \frac{1}{\sqrt{2\pi} \sigma} \exp \left(-\frac{( y^{(i)} - \theta^T x^{(i)})^2}{2\sigma^2} \right)\\
	&= \sum_{i=1}^{m} \ln  \frac{1}{\sqrt{2\pi} \sigma} \exp \left(-\frac{( y^{(i)} - \theta^T x^{(i)})^2}{2\sigma^2} \right)\\
	&= m \ln \frac{1}{\sqrt{2\pi} \sigma} - \frac{1}{\sigma^2} \frac{1}{2} \sum_{i=1}^{m} ( y^{(i)} - \theta^T x^{(i)})^2
\end{align*}
to maximize this equation is to minimizing
\[
\frac{1}{2} \sum_{i=1}^{m} ( y^{(i)} - \theta^T x^{(i)})^2
\]
which is the cost function $J(\theta)$, the origin least square cost function. And the we can summarize that
\begin{itemize}
	\item least square is equivalent to maximizing likelihood estimation(why it is natural)
	\item the probabilistic assumption is not necessary
	\item the variance $\sigma^2$ doesn't matter in this case  
\end{itemize}

\end{document}





