\documentclass[10pt,a4paper,oneside]{article}
\usepackage[utf8]{inputenc}
\usepackage[T1]{fontenc}
\usepackage{amsmath}
\usepackage{amsfonts}
\usepackage{amssymb}
\usepackage{graphicx}
\usepackage{enumerate}
\usepackage{multirow}

\date{July 9, 2019}
\author{Baboo J. Cui, Yangang Cao}
\title{Functions of Square Matrices}

\begin{document}
\maketitle
\tableofcontents

\newpage
\section{Functions and Polynomial of Matrix}
We can define functions of square matrices $A \in \mathbb{R}^{n \times n}$ like:
\[
f(A)=A^{k}, A^{\frac{1}{2}}, e^{A}, \ln (A),\left(I_{n}-A\right)^{-1}
\]
Real analytic functions has the properties:
\begin{itemize}
\item $f(\lambda)$ is infinitely differentiable
\item At any $\lambda_{0}$ where $f(\cdot)$ is defined, its Taylor expansion converges locally:
\[
f(\lambda)=f\left(\lambda_{0}\right)+f^{\prime}\left(\lambda_{0}\right)\left(\lambda-\lambda_{0}\right)+\frac{1}{2 !} f^{\prime \prime}\left(\lambda_{0}\right)\left(\lambda-\lambda_{0}\right)^{2}+\cdots
\]
\end{itemize}
For polynomials $f(\lambda)=a_{k} \lambda^{k}+a_{k-1} \lambda^{k-1}+\cdots+a_{0},$ the extension is easy:
\[
f(A)=a_{k} A^{k}+a_{k-1} A^{k-1}+\cdots+a_{0} I_{n} \in \mathbb{R}^{n \times n}, \quad \forall A \in \mathbb{R}^{n \times n}
\]
\begin{itemize}
\item replace every occurrence of $\lambda$ by $A$
\item replace constant term $a_0$ by $a_0 I_n$
\end{itemize}

\section{Polynomials of Matrices via JCF}
Suppose $A \in \mathbb{R}^{n\times n}$ has Jordan Canonical Form $(\mathrm{JCF})$:
\[
A=T J T^{-1}=T\left[\begin{array}{lll}{J_{1}} & {} \\ {} & {\ddots} & {} \\ {} & {} & {J_{q}}\end{array}\right] T^{-1}
\]
Clearly, $A^k = T J^k T^{-1}$, and for a polynomial function $f(\lambda), f(A)$ can be computed via JCF. Each Jordan block $J_{i}$ of size $n_{i}$ can be decomposed as $J_{i}=\lambda_{i} I+N_{i}$
\begin{itemize}
\item $\lambda_{i} I$ and $N_{i}$ commute
\item $N_{i}$ is nilpotent: $\left(N_{i}\right)^{n_{i}}=0$
\end{itemize}
Polynomial function of $J_{i}$ can be computed using these properties.
Let $\chi_{A}(\lambda)=\operatorname{det}(\lambda l-A)$ be the \textbf{characteristic polynomial} of $A$, for any Jordan block $J_{i}$ of $A$,
\[
\chi_{A}\left(J_{i}\right)=0
\]
\textbf{Cayley-Hamilton Theorem}: for any matrix $A\in \mathbb{R}^{n\times n}$, $\chi_{A}(A)=0$, since:
\[
\chi_{A}(A)=T\left[\begin{array}{ccc}{\chi_{A}\left(J_{1}\right)} & {} & {} \\ {} & {\ddots} & {} \\ {} & {} & {\chi_{A}\left(J_{q}\right)}\end{array}\right] T^{-1}=0
\]
\section{Minimal Polynomials(Very Useful)}
The minimal polynomial of $A \in \mathbb{R}^{n \times n}$ is the polynomial $\mu_{A}(\lambda)$ with the \textbf{minimum degree} satisfying $\mu_{A}(A)=0$. Using the JCF, the minimal polynomial is
\[
\mu_{A}(\lambda)=\left(\lambda-\lambda_{1}\right)^{d_{1}} \cdots\left(\lambda-\lambda_{\ell}\right)^{d_{\ell}}
\]
\begin{itemize}
\item $\lambda_{1}, \ldots, \lambda_{\ell}$ are the distinct eigenvalues of $A$
\item $d_{i}$ is the largest size of Jordan blocks associated with $\lambda_{i}$
\end{itemize}

\section{Implication of C-H Theorem}

\subsection{Part 1}
Given a square matrix $A\in \mathbb{R}^{n\times n}$
\begin{itemize}
\item $A^{k} \in \operatorname{span}\left\{I_{n}, A, A^{2}, \ldots, A^{n-1}\right\},$ for $k=0,1,2, \ldots$
\item For any polynomial function $f(\lambda)$
\[
f(A) \in \operatorname{span}\left\{I_{n}, A, A^{2}, \ldots, A^{n-1}\right\}
\]
which indicate that
\[
f(A)=h(A) \text { for some polynomial } h(\lambda) \text { of } \text {degree} \leq n-1
\]
\end{itemize}

\subsection{Part 2}
Given a square matrix $A\in\mathbb{R}^{n\times n}$ and a polynomial function $f(\lambda)$ so that
\[
f(A)=h(A)
\]
for a polynomial $h(\lambda)$ of degree at most $n-1$ satisfying
\[
\left\{\begin{array}{ll}{f^{(j)}\left(\lambda_{1}\right)=h^{(j)}\left(\lambda_{1}\right),} & {j=0,1, \ldots, m_{1}-1} \\ {f^{(j)}\left(\lambda_{\ell}\right)=h^{(j)}\left(\lambda_{\ell}\right),} & {j=0,1, \ldots, m_{\ell}-1}\end{array}\right.\quad\quad(1)
\]
\begin{itemize}
\item $\lambda_{1}, \ldots, \lambda_{\ell}$ are the distinct eigenvalues of $A$
\item $m_{1}, \ldots, m_{\ell}$ are their \textbf{algebraic} multiplicities
\item $f(\lambda)$ and $h(\lambda)$ agree on the spectrum $\sigma(A)$ of matrix $A$
\end{itemize}

\section{General Functions of Square Matrices}
Given an analytical function $f(\lambda)$ that is defined at $\lambda=0$
\[
\begin{aligned} f(\lambda) &=f(0)+f^{\prime}(0) \lambda+\frac{1}{2 !} f^{\prime \prime}(0) \lambda^{2}+\cdots \\ \Rightarrow \quad f(A) &=f(0) I+f^{\prime}(0) A+\frac{1}{2 !} f^{\prime \prime}(0) A^{2}+\cdots \end{aligned}
\]
The general function of the square matrix $A \in \mathbb{R}^{n \times n}$ is defined as
\[
f(A)=h(A)
\]
where $h(\lambda)$ is a polynomial of degree at most $n-1$ that agrees with $f(\lambda)$
on the spectrum $\sigma(A)$ of $A$.

\subsection*{Example}
Find $f(A)=e^{A}$ for $A_{1}=\left[\begin{array}{cc}{0.4} & {0.6} \\ {0.7} & {0.3}\end{array}\right], \quad A_{2}=\left[\begin{array}{ccc}{\lambda_{1}} & {1} & {} \\ {} & {\lambda_{1}} & {1} \\ {} & {} & {\lambda_{1}}\end{array}\right]$

\section{General Functions of Matrices via JCF}
Suppose $A \in \mathbb{R}^{n \times n}$ has Jordan Canonical Form (JCF):
\[
A=T\left[\begin{array}{lll}{J_{1}} & {} & {} \\ {} & {\ddots} & {} \\ {} & {} & {J_{q}}\end{array}\right] T^{-1} \Rightarrow f(A)=T\left[\begin{array}{lll}{f\left(J_{1}\right)} & {} & {} \\ {} & {\ddots} & {} \\ {} & {} & {f\left(J_{q}\right)}\end{array}\right] T^{-1}
\]
For each Jordan block $J_i$:
\[
f(J_i)=\left[\begin{array}{ccccc}{f\left(\lambda_{i}\right)} &{f^{\prime}\left(\lambda_{i}\right)}&{\frac{1}{2 !}f^{\prime\prime}\left(\lambda_{i}\right)}&\cdots&{\frac{1}{(n_i-1)!}f^{(n_i-1)}(\lambda_i)}\\
{} & {f\left(\lambda_{i}\right)} & {f^{\prime}\left(\lambda_{i}\right)} & {\ddots}  & {\vdots} \\ 
{} & {} & {f\left(\lambda_{i}\right)} & {\ddots} & {\frac{1}{2 !} f^{\prime \prime}\left(\lambda_{i}\right)} \\ 
{} & {} & {} & {\ddots} & {f^{\prime}\left(\lambda_{i}\right)} \\
 {} & {} & {} &{}& {f\left(\lambda_{i}\right)}\end{array}\right]
\]
\subsection*{Example}
Consider $A=\left[\begin{array}{lll}{\lambda_{1}} & {1} \\ {} & {\lambda_{1}} & {1} \\ {} & {} & {\lambda_{1}}\end{array}\right]$
\begin{itemize}
\item $f_{1}(\lambda)=\lambda^{1000}$, $f_{1}(A)=A^{1000}$
\item $f_{2}(\lambda)=\frac{1}{s-\lambda}$ for some constant scalar $s$, $f_{2}(A)=(sI-A)^{-1}$

\end{itemize}
\end{document}