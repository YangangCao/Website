\documentclass[10pt,a4paper,oneside]{article}
\usepackage[utf8]{inputenc}
\usepackage[T1]{fontenc}
\usepackage{amsmath}
\usepackage{amsfonts}
\usepackage{amssymb}
\usepackage{graphicx}
\usepackage{enumerate}
\usepackage{multirow}
\usepackage[framed,numbered,autolinebreaks,useliterate]{mcode} 
\date{July 11, 2019}
\author{Baboo J. Cui, Yangang Cao}
\title{Lecture 8: Stability of Linear Systems}
\newcommand{\tabincell}[2]{\begin{tabular}{@{}#1@{}}#2\end{tabular}}
\begin{document}
\maketitle
\tableofcontents
\newpage
\section{Stability of C-T Autonomous Linear Systems}
A continuous-time autononous linear system
\begin{equation}
\dot{x}(t)=A(t) x(t)
\end{equation}
Definition (Asymptotic Stability):\\
The linear system $(1)$ is called asymptotically stable at $x_{e}=0$ if its solution $x(t)$ starting from any initial condition $x(0) \in \mathbb{R}^{n}$ satisfies
\[
x(t) \rightarrow 0 \text { as } t \rightarrow \infty
\]
Definition (Exponential Stability):\\
The linear system $(1)$ is called exponentially stable at $x_{e}=0$ if its solution
$x(t)$ starting from any initial condition $x(0) \in \mathbb{R}^{n}$ satisfies
\[
\|x(t)\| \leq K e^{-r}\|x(0)\|, \quad \forall t \geq 0,
\]
for some constants $K, r>0$.
\section{Stability of C-T LTI Systems}
Theorem:\\
For a C-T LTI system $\dot{x}=A x,$ the following statements are equivalent
\begin{enumerate}
\item System is asymptotically stable
\item System is exponentially stable
\item All eigenvalues of $A$ are in the open left half of the complex plane $\mathbb{C}$
\end{enumerate}
\section{Phase Portraits of Stable 2D Systems}
\section{Unstable Systems}
Definition:\\
The LTI system $\dot{x}=Ax$ is unstable if, starting from some $x(0)$, the solution $x(t)$ will deverge to infinity.\\
\\
Theorem:\\
The LTI system is unstable if either of the following is true:
\begin{enumerate}
\item $A$ has eigenvalues on the open right half plane of $\mathbb{C}$
\item $A$ has a defective eigenvalues on the $j\omega$-axis
\end{enumerate}
$\bullet$ An eigenvalue is defective if at least one of its Jordan blocks has size greater than one
\section{Phase Portraits of Unstable 2D Systems}
\section{Marginally Stable Systems}
Definition:\\
The LTI system $\dot{x}=Ax$ is marginally stable if, starting from some $x(0)$, the solution $x(t)$ will neither converge to zero nor diverge to infinity.\\
\\
Theorem:\\
The LTI system is marginally stable if both of the following hold:
\begin{enumerate}
\item All eigen of $A$ are in the closed left half of the complex plane $\mathbb{C}$
\item There are some eigenvalues of $A$ on the $j\omega$-axis, and all the Jordan blocks associated with such eigenvalues have size one
\end{enumerate}
\section{Phase Portraits of Marginally Stable 2D Systems}
\section{Phase Portraits of 3D Systems}
\section{Stability of C-T LTV Systems}
For LTV system $\dot{x}(t)=A(t)x(t)$, its solution is $x(t) = \Phi(t)x(0),t\geq0$\\
\\
Theorem:
\begin{itemize}
\item LTV system is asymptotically stable if $\Phi(t)\rightarrow0$ as $t\rightarrow\infty$
\item LTV system is exponentially stable if there exist $C$, $r>0$ such that
\[
||\Phi(t)||\leq Ce^{-rt}, \forall t\geq 0
\]
\end{itemize}
\section{Stability of D-T Autonomous Linear Systems}
A discrete-time LTV system
\[
x[k+1]=A[k] x[k], \quad k=0,1, \ldots
\]
Definition (Asymptotic Stability):\\
LTV system is {\bfseries asymptotically stable at time} $k_{0}$ if its solution $x[k]$
starting from any initial condition $x\left[k_{0}\right]$ at time $k_{0}$ satisfies
\[
x[k] \rightarrow 0 \text { as } k \rightarrow \infty
\]
Definition(Exponential Stability):\\
LTV system is {\bfseries exponentially stable at time} $k_{0}$ if its solution $x[k]$
starting from any initial condition $x\left[k_{0}\right]$ at time $k_{0}$ satisfies
\[
\|x[k]\| \leq K r^{k-k_{0}}\left\|x\left[k_{0}\right]\right\|, \quad \forall k=k_{0}, k_{0}+1, \ldots
\]
for some constants $K>0, 0\leq r<1$
\section{Stability of D-T LTV Systems}
For LTV system $x[k+1]=A[k] x[k], \quad k=0,1, \ldots$
\begin{itemize}
\item LTV system is asymptotically stable at time $k_{0}$ if
\[
\Phi\left[k, k_{0}\right] \rightarrow 0 \text { as } k \rightarrow \infty
\]
\item LTV system is exponentially stable at time $k_0$ if there exist $C\geq0$,
\[
\left\|\Phi\left[k, k_{0}\right]\right\| \leq C r^{k-k_{0}}, \quad \forall k \geq k_{0}
\]
\item Asymptotic stability is {\bfseries not equivalent} to exponential stability
\item The starting time $k_0$ {\bfseries does matter}
\end{itemize}
\section{Stability of D-T LTI Systems}
Consider the discrete-time LTI system
\[
x[k+1] = Ax[k], k=0,1,...
\]
Theorem:\\
The following statements are equivalent
\begin{enumerate}
\item The LTI system is asymptetically stable
\item The LTI system is exponentially stable
\item All the eigenvalues of $A$ are inside the open unit disk of the complex plane $\mathbb{C}$
\end{enumerate}
$\bullet$ For LTI systems, the starting time $k_0$ does not matter
\section{Marginal Stability of D-T LTI Systems}
Given a LTI system $x[k+1]=Ax[k]$
The LTI system is {\bfseries marginally stable} if both of the following hold:
\begin{enumerate}
\item All the eigenvalues of $A$ are inside the closed unit disk of $\mathbb{C}$
\item There are some non-defective eigenvalues of $A$ on the unit circle
\end{enumerate}
The LTI system is {\bfseries unstable} if either of the following is true:
\begin{enumerate}
\item $A$ has eigenvalues outside the closed unit disk of $\mathbb{C}$
\item $A$ has a defective eigenvalues on the unit circle
\end{enumerate}
\end{document}