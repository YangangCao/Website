\documentclass[10pt,a4paper,oneside]{article}
\usepackage[utf8]{inputenc}
\usepackage[T1]{fontenc}
\usepackage{amsmath}
\usepackage{amsfonts}
\usepackage{amssymb}
\usepackage{graphicx}
\usepackage{enumerate}
\usepackage{multirow}
\usepackage[framed,numbered,autolinebreaks,useliterate]{mcode} 
\date{July 18, 2019}
\author{Baboo J. Cui, Yangang Cao}
\title{Lecture 10: Lumped Nonlinear Systems}
\newcommand{\tabincell}[2]{\begin{tabular}{@{}#1@{}}#2\end{tabular}}
\begin{document}
\maketitle
\tableofcontents
\newpage
\section{Lumped Nonlinear Systems}
Lumped continuous-time nonlinear system:
\[
\frac{d}{d t} x(t)=f(x, u, t), \quad y(t)=g(x, u, t)
\]
Lumped discrete-time nonlinear system:
\[
x[k+1]=f(x[k], u[k], k), \quad y[k]=g(x[k], u[k], k)
\]
$\bullet$ For $m$-input, $n$-state, $p$-output systems,
\[
f : \mathbb{R}^{n} \times \mathbb{R}^{m} \times \mathbb{R} \rightarrow \mathbb{R}^{n}, \quad g : \mathbb{R}^{n} \times \mathbb{R}^{m} \times \mathbb{R} \rightarrow \mathbb{R}^{p}
\]
\section{Autonomous Time-lnvariant Nonlinear System}
\[
\frac{d}{d t} x(t)=f(x(t)), \quad \text { where } f : \mathbb{R}^{n} \rightarrow \mathbb{R}^{n}
\]
\begin{itemize}
\item {\bfseries Equilibrium points}: solutions $x_{e}$ to $f(x)=0$
\[
f\left(x_{\mathrm{e}}\right)=0
\]
\item Let $\delta x(t)=x(t)-x_{\mathrm{e}} .$ The dynamics of $\delta x(t)$ is approximated by
\[
\frac{d}{d t} \delta x(t) \approx \underbrace{  \operatorname{Df}\left(x_{\mathrm{e}}\right)}_{\text { Jacobian }} \delta x(t)
\]
\end{itemize}
\section{Example: Simple Pendulum}
Dynamics: $\ddot{\theta}=-m g \ell \sin \theta-\eta \dot{\theta}$\\
Define state as $\left[\begin{array}{l}{x_{1}} \\ {x_{2}}\end{array}\right]=\left[\begin{array}{l}{\theta} \\ {\dot{\theta}}\end{array}\right]$\\
State equation:
$\frac{d}{d t}\left[\begin{array}{l}{x_{1}} \\ {x_{2}}\end{array}\right]=\left[\begin{array}{c}{x_{2}} \\ {-m g \ell \sin x_{1}-\eta x_{2}}\end{array}\right]$
\section{Stability of Autonomous Nonlinear Systems}
Definition:\\
$\frac{d}{d t} x(t)=f(x(t))$ is locally asymptotically stable near the equilibrium point $x_e$ if there exists some $r>0$ such that
\[
\left\|x(0)-x_{e}\right\|<r \Rightarrow \lim _{t \rightarrow \infty}\left\|x(t)-x_{e}\right\| \rightarrow 0
\]
\begin{itemize}
\item All solutions starting in a ball of radius $r$ around $x_e$ converge to it
\item If $r$ can be chosen to be $\infty,$ we get global asymptotic stability
\end{itemize}
\section{Sufficient Condition for (In)stability}
Theorem (Hartman-Gorbman Theorem):\\
For a nonlinear system $\dot{x}=f(x)$ with an equilibrium point $x_{\mathrm{e}}$, let $\frac{d}{d t} \delta x=D f\left(x_{e}\right) \delta x$ be its linearization around $x_{e}$
\begin{itemize}
\item If $D f\left(x_{e}\right)$ has all eigenvalues with negative real part, then nonlinear system $\dot{x}=f(x)$ is asy. stable around the equilibrium point $x_e$
\item If $D f\left(x_{e}\right)$ has all eigenvalues with positive real part, then nonlinear system $\dot{x}=f(x)$ is unstable around the equilibrium point $x_e$
\end{itemize}
\section{Inconclusive Results from Linearization}
\begin{itemize}
\item When sufficient stability/instability conditions fail, can have either
stability or instability
\item Example: $\dot{x}=-x^{3}$
\item Example: $\dot{x}=x^{3}$
\item Simple pendulum at $x_{\mathrm{e}}=\left[\begin{array}{l}{0} \\ {0}\end{array}\right]$
\end{itemize}
\section{Example: Lotka-Volterra Model}
Population model of two species:
\begin{itemize}
\item $x_1$: population of prey
\item $x_2$: population of predator
\end{itemize}
Assumptions:
\begin{itemize}
\item Unlimited food supply
\item Predator total dependence on prey
\[
\left\{\begin{array}{l}{\frac{d x_{1}}{d t}=4 x_{1}-2 x_{1} x_{2}} \\ {\frac{d x_{2}}{d t}=-x_{2}+x_{1} x_{2}}\end{array}\right.
\]
\end{itemize}
Equilibrium points: $x_{e, 1}=\left[\begin{array}{l}{0} \\ {0}\end{array}\right], x_{e, 2}=\left[\begin{array}{l}{1} \\ {2}\end{array}\right]$
\section{Phase Plot of Lotka-Volterra Model}
\section{Linearization of Controlled NLTI Systems}
A controlled nonlinear time-invaraint system
\[
\dot{x}(t)=f(x(t), u(t))
\]
\begin{itemize}
\item $x_e$ is an equilibrium if it satisfies
\[
f(x_e,0)=0
\]
\item For small initial deviation $\delta x=x-x_e$, and small input $u$:
\[
\frac{d}{d t} \delta x(t) \simeq \underbrace{D_{x} f\left(x_{e}, 0\right)}_{A} \delta x(t)+\underbrace{D_{u} f\left(x_{e}, 0\right)}_{B} u(t)
\]
\end{itemize}
\section{Linearization of NLTV Systems around a Trajectory}
Suppose the controlled nonlinar time-varying system
\[
\frac{d}{d t} x(t)=f(x, u, t), \quad x(0)=x_{0}, \quad y(t)=g(x, u, t)
\]
has a solution $x^{*}(t)$ and $y^{*}(t)$ under the input $u^{*}(t)$\\
Suppose input is perturbed by a small perturbation: $u(t)=u^{*}(t)+\delta u(t)$\\
The resulting $x(t)=x^{*}(t)+\delta x(t)$ and $y(t)=y^{*}(t)+\delta y(t)$ satisfy
\end{document}