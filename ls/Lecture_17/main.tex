\documentclass[10pt,a4paper,oneside]{article}
\usepackage[utf8]{inputenc}
\usepackage[T1]{fontenc}
\usepackage{amsmath}
\usepackage{amsfonts}
\usepackage{amssymb}
\usepackage{graphicx}
\usepackage{enumerate}
\usepackage{multirow}
\date{July 31, 2019}
\author{Baboo J. Cui, Yangang Cao}
\title{Lecture 17: State Feedback Control}

\begin{document}
\maketitle
\tableofcontents
\newpage
\section{Controller Design}
A continuous-time (for discrete-time) LTI system
\[
\dot{x}=A x+B u \quad \text { or } \quad x[k+1]=A x[k]+B u[k]
\]
Problem: Design input $u$ so that behavior of system is ``better''
\begin{itemize}
\item Stabilize the system (if $A$ is unstable)
\item Make state trajectories decay faster
\item Track a reference state trajectory
\item Minimize control cost and tracking error
\end{itemize}
Strategies:
\begin{itemize}
\item Open-loop control: Design $u(t)$ directly
\item Closed-loop control: Design $u=f_{sfb}(x,t)$ as a function of state and time (state-feedback control)
\end{itemize}
\section{State Feedback Control}
LTI system $\dot{x}=A x+B u,$ with $A \in \mathbb{R}^{n \times n}$ and $B \in \mathbb{R}^{n \times m}$\\
State feedback control:\\
\[
u=-Kx+v, \quad \text { for some constant }(\text {gain}) \text { matrix } K \in \mathbb{R}^{m \times n}
\]
Problem: Design gain matrix $K$ to achieve desirable closed-loop behaviors
\begin{itemize}
\item Many behaviors are characterized by eigenvalues of $A_{cl}$
\item How much can we re-allocate the eigenvalues of $A$ to those of $A_{cl} ?$
\end{itemize}
\section{State Feedback and Controllability}
Fact:\\
For any $K\in\mathbb{R}^{m\times n}$, system $(A-BK,B)$ is controllable if and only if system $(A,B)$ is controllable.\\
\\$\bullet$ State feedback cannot make an uncontrollable system controllable
\section{Example I}
\[
\dot{x}=A x+B u=\left[\begin{array}{ll}{1} & {3} \\ {3} & {1}\end{array}\right] x+\left[\begin{array}{l}{1} \\ {0}\end{array}\right] u
\]
\section{Example II}
\[
\dot{x}=A x+B u=\left[\begin{array}{ll}{1} & {0} \\ {0} & {1}\end{array}\right] x+\left[\begin{array}{l}{1} \\ {1}\end{array}\right] u
\]
\section{Pole-Placement of Uncontrollable Systems}
Fact:\\
Eigenvalues of $A-BK$ cannot be arbitrarily re-assigned if $(A,B)$ is not controllable.
\section{Example}
\[
\dot{x}=\left[\begin{array}{ll}{1} & {0} \\ {0} & {1}\end{array}\right] x+\left[\begin{array}{l}{1} \\ {1}\end{array}\right] u
\]
With coordinate transformation $x=\left[\begin{array}{cc}{1} & {1} \\ {1} & {-1}\end{array}\right] \tilde{x},$ we have
\[
\dot{x}=\left[\begin{array}{ll}{1} & {0} \\ {0} & {1}\end{array}\right] \tilde{x}+\left[\begin{array}{l}{1} \\ {0}\end{array}\right] u
\]
\section{Pole-Placement of Controllable Systems}
Fact:\\
Eigenvalues of $A-BK$ can be arbitrarily re-assigned for controllable $(A,B)$\\
\\
Proof: Single-input case: $(A,b)$\\
\\Claim: Suppose $\chi_{A}(\lambda)=\lambda^{n}+a_{1} s^{n-1}+\cdots+a_{n}$. A coordinate transform
\\
$x=Tx_c$ exists that transforms the system to its controller canonical form:
\[
A_{c}=T^{-1} A T=\left[\begin{array}{cccc}{-a_{1}} & {\cdots} & {-a_{n-1}} & {-a_{n}} \\ {1} & {} & {} &{}\\ {} & {\ddots} & {} &{}\\{} & {} & {1} & {}\end{array}\right], \quad B_{c}=T^{-1} B=\left[\begin{array}{c}{1} \\ {0} \\ {\vdots} \\ {0}\end{array}\right]
\]
(Proof in Textbook, pp. 234-235)\\
Choose $K_{c}=\left[\begin{array}{llll}{k_{1}} & {k_{2}} & {\cdots} & {k_{n}}\end{array}\right]$. Then
\[
\chi_{A_{c}-B_{c} K_{c}}(\lambda)=\lambda^{n}+\left(a_{1}+k_{1}\right) \lambda^{n-1}+\cdots+\left(a_{n}+k_{n}\right)
\]
whose roots can be arbitrary (subject to complex conjugate constraint)
\section{Example}
\[
\dot{x}=\left[\begin{array}{ll}{1} & {1} \\ {1} & {1}\end{array}\right] x+\left[\begin{array}{l}{1} \\ {0}\end{array}\right] u
\]
With coordinate transformation $x=\left[\begin{array}{cc}{1} & {-1} \\ {0} & {1}\end{array}\right] x_{c},$ we have
\[
\dot{x}_{c}=\left[\begin{array}{ll}{2} & {0} \\ {1} & {0}\end{array}\right] x_{c}+\left[\begin{array}{l}{1} \\ {0}\end{array}\right] u
\]
Suppose we want poles at $-1,-1$.
\section{Proof of Multi-Input Case}
Proof: Let $B=\left[\begin{array}{llll}{b_{1}} & {b_{2}} & {\cdots} & {b_{m}}\end{array}\right]$. If $\left(A, b_{i}\right)$ is controllable for any $i$, then previous proof works for single-input system with input $u_{i} .$ Otherwise, assume without loss of generality that $b_{1} \neq 0$.\\
\\Define matrix $V=\left[\begin{array}{lll}{v_{1}} & {\cdots} & {v_{n}}\end{array}\right] \in \mathbb{R}^{n \times n}$ by recursion:
\begin{itemize}
\item $v_{1}=b_{1}, v_{2}=A v_{1}, \ldots, v_{n_{1}}=A b_{n_{1}-1} .$ Stop when no linearly independent vector is generated
\item $v_{n_{1}+1}=A^{n_{1}} b_{1}+b_{2},$ where $b_{2}$ is any column of $B$ that is linearly
independent of previous $v_{i} ; v_{n+2}=A v_{n_{1}+1}, \ldots, v_{n+n_{2}}=A v_{n_{1}+n_{2}-1}$
Stop when no linearly independent vector is generated
\item Continue until $n$ linearly independent $v_{i}^{\prime}$ 's are generated
\end{itemize}
Define $F_{0}=\left[\underbrace{0 \cdots 0\quad \mathrm{e}_{2}}_{n_{1}} \quad\underbrace{0 \cdots 0 \quad\mathrm{e}_{3}}_{n_{2}}\quad\cdots\right] \cdot V^{-1}$\\
\\
Claim: $\left(A+B F_{0}, b_{1}\right)$ is controllable
\begin{itemize}
\item Check that its controllability matrix is exactly $V$
\item After feedback, closed-loop system controllable from single input $b_{1}$
\end{itemize}
\section{Effect on Transfer Function}
Fact:\\
For a contrattable SISO system $\dot{x}=A x+B u, y=c x,$ state feedback can
change the poles. but not the zeros, of its transfer function.
\section{Stabilizability}
Definition (Stabilizability):\\
System $(A,B)$ is called stabilizable if there exist a state feedback matrix $K$ such that the closed-loop system $A-BK$ is stable
\begin{itemize}
\item If $A$ is stable itself, $(A, B)$ is stabilizable
\item If $(A, B)$ is controllable, it is stabilizable as well
\item If $(A, B)$ is not controllable, it could still be stabilizable
\end{itemize}
Theorem:\\
System $(A, B)$ is stabilizable if all its uncontrolable modes correspond to stable eigenvalues of $A$
\section{Examples}
\end{document}
