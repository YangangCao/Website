\documentclass[10pt,a4paper,oneside]{article}
\usepackage[utf8]{inputenc}
\usepackage[T1]{fontenc}
\usepackage{amsmath}
\usepackage{amsfonts}
\usepackage{amssymb}
\usepackage{graphicx}
\usepackage{enumerate}
\usepackage{multirow}
\usepackage[framed,numbered,autolinebreaks,useliterate]{mcode} 
\date{July 10, 2019}
\author{Baboo J. Cui, Yangang Cao}
\title{Lecture 7: Autonomous LTV Systems}
\newcommand{\tabincell}[2]{\begin{tabular}{@{}#1@{}}#2\end{tabular}}
\begin{document}
\maketitle
\tableofcontents
\newpage
\section{Continuous-Time Autonomous LTV Systems}
Consider an autonomous linear time-varying (LTV) system
\[
\dot{x}(t)=A(t)x(t),\quad t\geq0,\quad \text{with initial condition $x(0)\in\mathbb{R}^n$}
\]
$\bullet$ $A(t) \in \mathbb{R}^{n \times n}$ is a function $\operatorname{of} t  \geq 0$\\
\\
Theorem (Existence and Uniqueness of Solutions):\\
If $A(t)$ is a piecewise continuous function of $t,$ then starting from any $x(0)$, the LTV system has a unique solution $x(t)$ for all $t \geq 0$.
\section{Scalar Autonomous LTV Systems}
Consider the scalar case with $x(t), a(t)\in\mathbb{R}$:
\[
\dot{x}(t)=a(t) x(t), \quad x(0) \in \mathbb{R}
\]
\section{Solution Space}
The solution space $\mathbb{X}$ of the LTV system is the set of all its solutions:
\[
\mathbb{X} :=\{x(t), t \geq 0 | \dot{x}(t)=A(t) x(t)\}
\]
$\bullet$ The solution space $\mathbb{X}$ is a vector space of dimension $n$
\section{Fundamental Matrix}
Define the fundamental matrix $\Phi(t)$, $t\geq0$, as
\[
\Phi(t) :=\left[\phi_{1}(t) \ldots \phi_{n}(t)\right] \in \mathbb{R}^{n \times n}, \quad t \geq 0.
\]
\begin{itemize}
\item $\phi_{i}(t)$ is the solution with intial condition $x(0)=e_{i}, i=1, \ldots, n$
\item Solution from any initial condition $x(0)$ can be written as
\[
x(t)=\Phi(t) x(0), \quad t \geq 0
\]
\end{itemize}
\section{Properties of Fundamental Matrix}
Fundamental matrix $\Phi(t) \in \mathbb{R}^{n \times n}$ solves the matrix differential equation:
\[
\dot{\Phi}(t)=A(t) \Phi(t), \quad \Phi(0)=I_{n}
\]
$\bullet$ $\Phi(t)$ is nonsingular at all time $t\geq0$ due to uniqueness of solutions
\section{State Transition Matrix}
Definition (State Transition Matrix):\\
The state transition matrix $\Phi(t, s) \in \mathbb{R}^{n \times n}, s.t \geq 0,$ for the LTV system is defined from the fundamental matrix $\Phi(t)$ by
\[
\Phi(t, s)=\Phi(t) \Phi(s)^{-1}
\]
\begin{itemize}
\item It relates the state solution at different times: for any solution $x(t)$,
\[
x\left(t_{2}\right)=\Phi\left(t_{2}, t_{1}\right) x\left(t_{1}\right), \quad \forall t_{1}, t_{2} \geq 0.
\]
\item For any $t_1,t_2,t_3\geq0$,
\[
\Phi\left(t_{3}, t_{2}\right) \Phi\left(t_{2}, t_{1}\right)=\Phi\left(t_{3}, t_{1}\right)
\]
\end{itemize}
\section{Example I}
Consider the LTV system: $\dot{x}(t)=\left[\begin{array}{ll}{0} & {1} \\ {0} & {t}\end{array}\right] x(t)$
\section{Example II}
Consider the LTV system: $\dot{x}(t)=A(t) x(t)=\left[\begin{array}{cc}{0} & {\omega(t)} \\ {-\omega(t)} & {0}\end{array}\right] x(t)$, Note that $A(t), t \geq 0,$ can be diagonalized by the same transformation:
\[
A(t)=\left[\begin{array}{cc}{0} & {\omega(t)} \\ {-\omega(t)} & {0}\end{array}\right]=\left[\begin{array}{cc}{j} & {-j} \\ {-1} & {1}\end{array}\right]\left[\begin{array}{cc}{-j \omega(t)} & {} \\ {} & {j\omega(t)}\end{array}\right]\left[\begin{array}{cc}{j} & {-j} \\ {-1} & {1}\end{array}\right]^{-1}
\]

\section{A Useful Special Case}
Proposition:\\
If $A(s)$ and $A(t)$ commute for all $s.t.\geq0$, then
\[
\Phi(t)=e^{\int_{0}^{t} A(\tau) d \tau}, \quad t \geq 0
\]
\begin{itemize}
\item State transition matrix becomes
\[
\Phi(t, s)=e^{\int_{s}^{t} A(\tau) d \tau}, \quad \forall s, t \geq 0
\]
\item Solution to the LTV system is
\[
x(t)=e^{\int_{0}^{t} A(\tau) d \tau} x(0), \quad t \geq 0
\]
\end{itemize}
\section{Example III}
Consider the LTV system: $\dot{x}(t)=A(t) x(t)=\left[\begin{array}{cc}{-e^{-t}} & {\frac{1}{t+1}} \\ {0} & {-e^{-t}}\end{array}\right] x(t)$
\section{Time-Reversed System}
Given a LTV system $\dot{x}(t)=A(t) x(t),$ its {\bfseries time-reversed system} is
\[
\dot{\hat{x}}(t)=\hat{A}(t) \hat{x}(t)=-A(-t) \hat{x}(t)
\]
\section{Discrete-Time Autonomous LTV Systems}
Discrete-Time LTV system
\[
x[k+1]=A[k] x[k], \quad k=0,1, \ldots
\]
with the initial condition $x[0] \in \mathbb{R}^{n}$\\
Solution $x[k]$ is
\[
x[k]=\underbrace{A[k-1] A[k-2] \cdots A[0]}_{\phi[k]} \cdot x[0], \quad k=0,1, \ldots
\]
where $\Phi[k] \in \mathbb{R}^{n \times n}$ is the fundamental matrix
\begin{itemize}
\item Unlike C-T case, {\bfseries fundamental matrix} $\Phi[k]$ {\bfseries may be singular}.
\item No time reversal in general, unless every $A[k]$ is nonsingular
\item For D-T LTI system, $\Phi[k]=A^{k}$
\end{itemize}
\section{Discrete-Time State Transition Matrix}
Definition (State Transition Matrix):\\
The state transition matrix $\Phi[k, \ell] \in \mathbb{R}^{n \times n}, k \geq \ell,$ for the discrete-time LTV system is defined as
\[
\Phi[k, \ell]=A[k-1] \cdots A[\ell]| \quad k \geq \ell
\]
\begin{itemize}
\item $\Phi[k, \ell]$ is defined only for $k \geq \ell$
\item $\Phi[k, k]=I_{n}$ for all $k=0,1, \ldots$
\item $\Phi\left[k_{3}, k_{2}\right] \Phi\left[k_{2}, k_{1}\right]=\Phi\left[k_{3}, k_{1}\right]$ for all $k_{3}>k_{2} \geq k_{1}$
\end{itemize}
\end{document}