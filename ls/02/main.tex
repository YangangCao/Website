%\documentclass[10pt,a4paper,oneside]{article}
%\usepackage[utf8]{inputenc}
%\usepackage[T1]{fontenc}
%\usepackage{amsmath}
%\usepackage{amsfonts}
%\usepackage{amssymb}
%\usepackage{graphicx}
%\usepackage{enumerate}
%\usepackage{multirow}
%\usepackage[framed,numbered,autolinebreaks,useliterate]{mcode} 



\documentclass[10pt,a4paper,oneside]{article}
\usepackage[utf8]{inputenc}
\usepackage[T1]{fontenc}
\usepackage{amsmath}
\usepackage{amsfonts}
\usepackage{amssymb}
\usepackage{graphicx}

\date{July 8, 2019}
\author{Baboo J. Cui, Yangang Cao}

\title{State-Space Model vs. Input-Output Model}
\newcommand{\tabincell}[2]{\begin{tabular}{@{}#1@{}}#2\end{tabular}}
\begin{document}
\maketitle
\tableofcontents
\newpage

\section{Internal vs. External Models}
\begin{itemize}
	\item Transfer function models are external (or I/O) models
	\begin{itemize}
		\item Describe only how the input affects the output
		\item System viewed as a black box
	\end{itemize}
	\item State-space models are internal models
	\begin{itemize}
		\item Describe how the input affects not only the output, but also all the internal state variables
		\item More complete models suitable for complicated system
	\end{itemize}
\end{itemize}
Many state-space model may correspond to one transfer function.

\section{IO Model}
\subsection{IO Models of CT LTI System}
Consider a continuous-time LTI system with \textbf{zero initial state}, let $h(t)$, $t>0$ be the system's \textbf{impulse response}, then, under any $u(t) , t\geq0$, system has the output
\[
y(t)=\int_{0}^{t}h(t-\tau)u(\tau)d\tau=h(t)*u(t), t\geq0
\]
Taking Laplace transform, we obtain the\textbf{ transfer function model}:
\[
\frac{Y(s)}{U(s)}=H(s)=\mathcal{L}[h(t)]
\]

\subsection{IO Models of DT LTI System}
For a discrete-time LTI system with zero initial state, \textbf{transfer function model} can be obtained by taking z-transform:
\[
\frac{Y(z)}{U(z)}=H(z)
\]
recall that $ay[k-n]$ has z-transform $az^{-n}Y(z)$

\subsection{IO Models from CT State-Space Models}
A continuous LTI system with $x\in\mathbb{R}^n$, $u\in\mathbb{R}^m$, $y\in\mathbb{R}^p$, and zero initial condition $x(0)=0$. Its transfer function (or matrix) can be deduced by taking Laplace transformation:
\begin{align*}
	sX(s) &= AX(s)+BU(s)\\
	Y(s) &= CX(s)+DU(s)
\end{align*}
which lead to
\[
Y(s) = \underbrace{[C(sI - A)^{-1}B + D]}_{H(s)}U(s)
\]

\subsection{IO Models from DT State-Space Models}
A discrete-time LTI system with $x\in\mathbb{R}^n$, $u\in\mathbb{R}^m$, $y\in\mathbb{R}^p$, and zero initial condition $x[0]=0$. Its transfer function (or matrix) can be deduced by taking z-transformation:
\begin{align*}
zX(z) &= AX(z)+BU(z)\\
Y(z) &= CX(z)+DU(z)
\end{align*}
which lead to
\[
Y(z) = \underbrace{[C(sI - A)^{-1}B + D]}_{H(z)}U(z)
\]
Example:
\begin{align*}
	\left[\begin{array}{l}{\dot{x}_{1}} \\ {\dot{x}_{2}}\end{array}\right] &= \left[\begin{array}{cc}{-1} & {0} \\ {0} & {-2}\end{array}\right]\left[\begin{array}{l}{x_{1}} \\ {x_{2}}\end{array}\right]+\left[\begin{array}{l}{1} \\ {1}\end{array}\right] u\\
	y &= \left[\begin{array}{ll}{-2} & {1}\end{array}\right] x
\end{align*}
has transfer function
\[
H(s) = -\frac{2}{s+1} + \frac{1}{s+2}
\]

\section{State-Space Realization}
\begin{itemize}
\item A continuous-time state-space model $(A,B,C,D)$ is called a realization of the transfer function $H(s)$ if $C(sI-A)^{-1}B+D=H(s)$
\item A discrete-time state-space model $(A,B,C,D)$ is called a realization of the transfer function $H(z)$ if $C(zI-A)^{-1}B+D=H(z)$
\end{itemize}
There are \textbf{infinitely many} realizations of a transfer function.

\subsection{Obtaining State-Space Realizations from IO Model}
IO model of a single-input single-output (SISO) system can be written in difference equation form:
\[
\sum_{i=0}^{n} a_i y^{(i)}(t) = \sum_{i=0}^{m} b_i u^{(i)}(t)
\]
Transfer function can be directly obtained by Laplace transformation.

\subsection{Controller Canonical Form}
Should be able to draw block diagram and write system dynamics.

\subsection{Controllability Canonical Form}
Should be able to draw block diagram and write system dynamics.

\subsection{Observer Canonical Form}
Should be able to draw block diagram and write system dynamics.

\subsection{Observability Canonical Form}
Should be able to draw block diagram and write system dynamics.

\subsection{Diagonal Realization}
Suppose $H(s)$ has distinct poles:
\[
\begin{aligned} H(s) &=\frac{b_{1} s^{2}+b_{2} s+b_{3}}{\left(s-\lambda_{1}\right)\left(s-\lambda_{2}\right)\left(s-\lambda_{3}\right)} \\ &=\frac{\gamma_{1}}{s-\lambda_{1}}+\frac{\gamma_{2}}{s-\lambda_{2}}+\frac{\gamma_{3}}{s-\lambda_{3}} \end{aligned}
\]
Diagonal realization will be:
\begin{align*}
	\left[\begin{array}{c}{\dot{x}_{1}} \\ {\dot{x}_{2}} \\ {\dot{x}_{3}}\end{array}\right] &= \left[\begin{array}{ccc}{\lambda_{1}} & {0} & {0} \\ {0} & {\lambda_{2}} & {0} \\ {0} & {0} & {\lambda_{3}}\end{array}\right]\left[\begin{array}{c}{x_{1}} \\ {x_{2}} \\ {x_{3}}\end{array}\right]+\left[\begin{array}{c}{1} \\ {1} \\ {1}\end{array}\right] u\\
	y &= \left[\begin{array}{lll}{\gamma_{1}} & {\gamma_{2}} & {\gamma_{3}}\end{array}\right]\left[\begin{array}{l}{x_{1}} \\ {x_{2}} \\ {x_{3}}\end{array}\right]
\end{align*}
\end{document}