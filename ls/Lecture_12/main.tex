\documentclass[10pt,a4paper,oneside]{article}
\usepackage[utf8]{inputenc}
\usepackage[T1]{fontenc}
\usepackage{amsmath}
\usepackage{amsfonts}
\usepackage{amssymb}
\usepackage{graphicx}
\usepackage{enumerate}
\usepackage{multirow}
\usepackage[framed,numbered,autolinebreaks,useliterate]{mcode} 
\date{July 22, 2019}
\author{Baboo J. Cui, Yangang Cao}
\title{Lecture 12: Controllability I}
\newcommand{\tabincell}[2]{\begin{tabular}{@{}#1@{}}#2\end{tabular}}
\begin{document}
\maketitle
\tableofcontents
\newpage
\section{Controllability of C-T LTI Systems}
A continuous-time $n$-state $m$-input LTI system
\begin{equation}
\dot{x}=Ax+Bu,\quad x(0)=x_0
\end{equation}

Given a terminal time $t_{f}>0$. Over the time interval $\left[0, t_{f}\right]$, the control
input $u(t), 0 \leq t \leq t_{f}$, steers (or transfers) the state from $x_{0}$ to
\[
x_{f} :=x\left(t_{f}\right)=e^{A t_{f}} x_{0}+\int_{0}^{t_{f}} e^{A\left(t_{f}-\tau\right)} B u(\tau) d \tau
\]
Definition:
\begin{itemize}
\item The LTI system is called {\bfseries controllable at time} $t_f>0$ if for any initial state $x_0\in\mathbb{R}^n$ and any target state $x_f\in\mathbb{R}^n$, a control $u(t)$ exists that can steer the system from $x_0$ to $x_f$ over the time interval $[0, t_f]$.
\item It is called {\bfseries controllable} if it is controllable at a large enough $t_f$
\end{itemize}
\section{A Example}
State equation:
\[
\frac{d}{d t}\left[\begin{array}{l}{i_{1}} \\ {i_{2}}\end{array}\right]=\left[\begin{array}{ll}{0} & {0} \\ {0} & {0}\end{array}\right]\left[\begin{array}{l}{i_{1}} \\ {i_{2}}\end{array}\right]+\left[\begin{array}{l}{1} \\ {1}\end{array}\right] u
\]
Assume unit inductances, and zero initial current $x(0)=\left[\begin{array}{l}{i_{1}(0)} \\ {i_{2}(0)}\end{array}\right]=\left[\begin{array}{l}{0} \\ {0}\end{array}\right]$
\section{Question Related to Controllability}
\begin{itemize}
\item Where can $x_{0}$ be transferred to over the time period $\left[0, t_{f}\right] ?$
\item If doable, how do we find $u$ that transfers $x_{0}$ to $x_{f} ?$
\item How quickly can $x_{0}$ be transferred to $x_{f} ?$
\item How do we find ``efficient'' $u$ that transfers $x_{0}$ to $x_{f} ?$
\end{itemize}
\section{Reachable Set}
To study the controllability of the LTI system $\dot{x}=A x+B u,$ we first consider the special case when $x_{0}=0 .$ Then at a terminal time $t_{f}>0$
\[
x_{f} :=x\left(t_{f}\right)=\int_{0}^{t_{f}} e^{A\left(t_{f}-\tau\right)} B u(\tau) d \tau
\]
Definition (Reachable Set):\\
The reachable set at time $t_{f}>0$ of the LTI system is the set of states
the system can be steered to using arbitrary control inputs over $\left[0, t_{f}\right]$:
\[
\mathcal{R}_{t_{f}} :=\left\{\int_{0}^{t_{f}} e^{A\left(t_{f}-\tau\right)} B u(\tau) d \tau | u(t), 0 \leq t \leq t_{f}\right\}
\]
\begin{itemize}
\item $\mathcal{R}_{t_f}$ is a subspace of $\mathbb{R}^{n}$ since it is the image of the linear map
\[
u(t), 0 \leq t \leq t_{f} \mapsto x_{f}=\int_{0}^{t_{f}} e^{A\left(t_{f}-\tau\right)} B u(\tau) d \tau
\]
\item $\mathcal{R}_{t_{f}} \subset \mathcal{R}_{\tilde{t}_{f}}$ whenever $t_{f}<\tilde{t}_{f}(\text { can reach more states given more time })$
\end{itemize}
\section{Reachability of C-T LTI Systems}
Definition (Reachability):\\
System is called {\bfseries reachable} at time $t_f>0$ if $\mathcal{R}_{t_{f}}=\mathbb{R}^{n}$, i.e., if it is can be steer from $x_0=0$ to any $x_f\in\mathbb{R}^n$ over the time interval $[0, t_f]$.\\
\\
Proposition (Reachability = Controllability):\\
At any $t_{f}>0,$ the LTI system is controllable if and only if is reachable.
\section{Controllability of D-T LTI Systems}
A discrete-time $n$-state $m$-input LTI system
\begin{equation}
x[k+1]=A x[k]+B u[k], \quad x[0]=x_{0}
\end{equation}
Definition:
\begin{itemize}
\item The LTI system (2) is {\bfseries controllable at time} $k_f>0$ if for any $x_0$, $x_f\in\mathbb{R}^n$, a control $u[k]$, $k=0,...,k_f-1$, exists that can steer the system from $x_0$ at time 0 to $x_f$ at time $k_f$
\item It is {\bfseries controllable} if it is controllable at a large enough time $k_f$
\end{itemize}
\section{Reachability of D-T LTI System}
Definition (Reachable Set):\\
The {\bfseries reachable set} at time $k_{f}>0$ is the set of states the system (2)
starting from $x_{0}=0$ can be steered to at time $k_{f}$:
\[
\mathcal{R}_{k_{f}} :=\left\{\sum_{i=0}^{k_{f}-1} A^{k_{f}-1-i} B u[i] | u(k), k=0,1, \ldots, k_{f}-1\right\}
\]
$\bullet$ System is {\bfseries reachable at time} $k_f>0$ if $\mathcal{R}_{k_{f}}=\mathbb{R}^{n}$\\
\\Proposition (Reachability = Controllability):\\
The D-T LTI system is controllable if and only if it is reachable (at any $k_f$)
\section{Controllability Matrix}
For the D-T system $x[k+1]=A x[k]+B u[k]$ with $x[0]=0$,
\[
x[k]=\underbrace{\left[B \quad A B \cdots A^{k-1} B\right]}_{C_{k}}\left[\begin{array}{c}{u[k-1]} \\ {\vdots} \\ {u[0]}\end{array}\right]
\]
\begin{itemize}
\item Reachable set at time $k$ is the range of the matrix $\mathcal{C}_{k}$ i.e., $\mathcal{R}\left(\mathcal{C}_{k}\right)$
\item {\bfseries Observation}: $\mathcal{R}\left(\mathcal{C}_{k}\right)=\mathcal{R}\left(\mathcal{C}_{n}\right)$ for $k \geq n$
\end{itemize}
Definition (Controllability Matrix):\\
The controllability matrix of the system is
\[
\mathcal{C} :=\mathcal{C}_{n}=\left[\begin{array}{llll}{B} & {A B} & {\cdots} & {A^{n-1} B}\end{array}\right]
\]
\section{Characterizing Controllability}
Proposition:\\
The reachable subspace of the D-T LTI system $(A,B)$ is the range of its controllability matrix  $\mathcal{C}=\left[\begin{array}{llll}{B} & {A B} & {\cdots} & {A^{n-1} B}\end{array}\right]$:
\[
\mathcal{R}=\mathcal{R}(\mathcal{C})
\]
Theorem:\\
The D-T LTI system $(A,B)$ is controllable (reachable) if and only if its controllability matrix $\mathcal{C}$ is onto, or equivalently, full (row) rank
\section{Example}
\begin{enumerate}
\item $x[k+1]=\left[\begin{array}{ll}{0} & {0} \\ {0} & {0}\end{array}\right] x[k]+\left[\begin{array}{l}{1} \\ {1}\end{array}\right] u[k]$
\item $x[k+1]=\left[\begin{array}{ll}{2} & {0} \\ {0} & {2}\end{array}\right] x[k]+\left[\begin{array}{l}{1} \\ {1}\end{array}\right] u[k]$
\item $x[k+1]=\left[\begin{array}{ll}{2} & {0} \\ {0} & {1}\end{array}\right] x[k]+\left[\begin{array}{l}{1} \\ {1}\end{array}\right] u[k]$
\end{enumerate}
\section{Equivalent Condition for Controllability}
Recall that $\mathcal{C}$ is onto if and only if $\mathcal{CC}^T$ is nonsingular\\
\\
Theorem:\\
The D-T LTI system $(A,B)$ is controllable if and only if 
\[
\mathcal{C C}^{T}=\sum_{k=0}^{n-1} A^{k} B B^{T}\left(A^{T}\right)^{k}
\]
is nonsingular
\section{PHB Tests of Controllability}
Theorem (Popov-Belevitch-Hautus):\\
The D-T LTI system $(A, B)$ is controllable if and only if
\begin{enumerate}
\item {\bfseries Rank Test}: for any $\lambda\in\mathbb{C}$
\[
\operatorname{rank}[\lambda I-A \quad B]=n
\]
\item {\bfseries Eigenvector Test}: for any left eigenvector $w\in\mathbb{C}^n$ of $A$,
\[
w^{T} B \neq 0
\]
\end{enumerate}
\section{Proof of PBH Tests}
\section{Example}
\begin{enumerate}
\item $x[k+1]=\left[\begin{array}{ll}{2} & {0} \\ {0} & {2}\end{array}\right] \times[k]+\left[\begin{array}{l}{1} \\ {1}\end{array}\right] u[k]$
\item $x[k+1]=\left[\begin{array}{ll}{2} & {0} \\ {0} & {1}\end{array}\right] \times[k]+\left[\begin{array}{l}{1} \\ {1}\end{array}\right] u[k]$
\item $A = \left[\begin{array}{ccccccc}{-1} & {1} & {0} & {0} & {0} & {0} & {0} \\ {0} & {-1} & 0 & {0} & {0} & {0} & {0} \\ {0} & {0} & {-1} & {1} & {0} & {0} & {0} \\ {0} & {0} & {0} & {-1} & {0} & {0} & {0} \\ {0} & {0} & {0} & {0} & {0} & {1} & {0} \\ {0} & {0} & {0} & {0} & {0} & {0} & {1} \\ {0} & {0} & {0} & {0} & {0} & {0} & {0}\end{array}\right]$, $B = \left[\begin{array}{cc}0&0\\-2&2\\2&0\\1&-1\\0&0\\0&0\\0&3\end{array}\right]$
\end{enumerate}
\section{Example}
Consider the composite system by the negative feedback connection of two matching linear systems $\left(A_{1}, B_{1}, C_{1}\right)$ and $\left(A_{2}, B_{2}, C_{2}\right)$:
\[
\tilde{A}=\left[\begin{array}{cc}{A_{1}} & {-B_{1} C_{2}} \\ {-B_{2} C_{1}} & {A_{2}}\end{array}\right], \quad \tilde{B}=\left[\begin{array}{cc}{B_{1}} & {0} \\ {0} & {B_{2}}\end{array}\right]
\]
Fact:\\
The composite system $(\tilde{A}, \tilde{B})$ is controllable if and only if both subsystems $\left(A_{1}, B_{1}\right)$ and $\left(A_{2}, B_{2}\right)$ are controllable.
\section{Minimum Energy Input for Reachability}
Suppose system $(A,B)$ is controllable, and $k\geq n$
{\bfseries Minimum energy input} is the input $u^{*}[0], \ldots, u^{*}[k-1]$ that can steer the system from $x[0]=0$ to $x_d$ at time $k$ with the least energy $\sum_{i=0}^{k-1}\|u[i]\|^{2}$
\begin{itemize}
\item Minimum energy input is
\[
u^{*}=\left[\begin{array}{c}{u^{*}[k-1]} \\ {\vdots} \\ {u^{*}[0]}\end{array}\right]=\mathcal{C}_{k}^{T}\left(\mathcal{C}_{k} \mathcal{C}_{k}^{T}\right)^{-1} x_{d}
\]
\item Minimun energy required to reach $x_d$ is
\[
\mathcal{E}_{\min }=\left\|u^{*}\right\|^{2}=x_{d}^{T}\left(\mathcal{C}_{k} \mathcal{C}_{k}^{T}\right)^{-1} x_{d}=x_{d}^{T}\left(\sum_{i=0}^{k-1} A^{i} B B^{T}\left(A^{T}\right)^{i}\right)^{-1} x_{d}
\]
\end{itemize}
\section{Example}
\[
x[k+1]=\left[\begin{array}{cc}{1.75} & {0.8} \\ {-0.95} & {0}\end{array}\right] x[k]+\left[\begin{array}{l}{1} \\ {0}\end{array}\right] u[k], \text { with } x[0]=0, x_{d}=\left[\begin{array}{l}{1} \\ {1}\end{array}\right]
\]
\section{Minimum Energy Over Infinite Horizon}
If $(A, B)$ is controllable, the following matrix always exists:
\[
P=\lim _{k \rightarrow \infty}\left(\sum_{i=0}^{k-1} A^{i} B B^{T}\left(A^{T}\right)^{i}\right)^{-1}
\]
The minimun energy required to reach a point $x_d$ with no limit on $k$ is
\[
\mathcal{E}_{\min }=x_{d}^{T} P x_{d}
\]
\end{document}