\documentclass[10pt,a4paper,oneside]{article}
\usepackage[utf8]{inputenc}
\usepackage[T1]{fontenc}
\usepackage{amsmath}
\usepackage{amsfonts}
\usepackage{amssymb}
\usepackage{graphicx}


\author{Baboo J. Cui}
\title{Systems and State Variables}
\date{Fall, 2014}
\begin{document}
\maketitle
\tableofcontents

\newpage

\section{System}

\subsection{Examples of Systems}
There are many systems around us, here are some examples:
\begin{itemize}
	\item electrical system
	\item mechanical system
	\item transportation system
	\item biological system
	\item ecological system
	\item stock market
\end{itemize}

\subsection{Structure of Studying System}
\begin{itemize}
	\item \textbf{modeling}: difference equation(direct way), transfer functions(undergraduate), state-space models(graduate), etc
	\item \textbf{analysis}(about the properties): solution, stability, controllability, observability, stabilizability, detectability, etc
	\item \textbf{design}: feedback control, optimal control, robust control, etc
\end{itemize}

\subsection{System Models}
System essentially is a signal \textbf{processor}:
\[
y = \mathcal{T}(u)
\]
\begin{itemize}
	\item $u$: \textbf{input}
	\item $y$: \textbf{output}
	\item $\mathcal{T}$: IO \textbf{mapping}, could be described by \textbf{ODE}(ordinary), \textbf{PDE}(partial), \textbf{SDE}(stochastic) or difference equations.
\end{itemize}

\subsection{Signals}
Signals are discussed in details in DSP, here is a brief summary. Mathematically, signal is functions over time index $\mathcal{I}$:
\begin{itemize}
	\item $u: \mathcal{I} \rightarrow \mathbb{R}^m$, a $m$-dimension input signal
	\item $y: \mathcal{I} \rightarrow \mathbb{R}^p$, a $p$-dimension output signal
\end{itemize}
Signals are be classified based on characteristics of $\mathcal{I}$:
\begin{itemize}
	\item $\mathcal{I} = \mathbb{R}$ leads to \textbf{continuous}-time signal, denote as $u(t)$
	\item $\mathcal{I} = \mathbb{Z}$ leads to \textbf{discrete}-time signal, denoted as $u[k]$
	\item $\mathcal{I}=[0, +\infty)$ or $\mathcal{I}=\{0, 1, \dots,\}$ leads to \textbf{causal} signal
\end{itemize}
Mathematically a mapping can be written in $2$ ways:
\begin{itemize}
	\item set mapping: $u: \mathcal{I} \rightarrow \mathbb{R}^m$
	\item set mapping with detailed info: $u: i \in \mathcal{I} \mapsto u(i) \in \mathbb{R}^m$
\end{itemize}
\textbf{Admissible signals} are allowed to put into systems, the admissible input set $\mathcal{U}$ must have the following properties for space state systems:
\begin{itemize}
	\item it is a vector space
	\item closed to time right shift operation(delay)
\end{itemize}
For transfer function approach, the system must be
\begin{itemize}
	\item casual
	\item exponentially bounded
\end{itemize} 

\subsection{Classification of Systems}
This part is explained in details in DSP class, here is a quick summary:
\begin{itemize}
	\item \textbf{discrete} vs \textbf{continuous}-time system: 
	\begin{itemize}
		\item continuous if both IO are continuous signal
		\item discrete if both IO are discrete signal
		\item \textbf{hybrid} if IO have both types of signals
	\end{itemize}
	\item \textbf{linear} vs \textbf{nonlinear} system: 
	\begin{itemize}
		\item linear if \textbf{superposition principle} is satisfied
		\[
		\mathcal{T}(\lambda_1 u_1 + \lambda_2 u_2) = \lambda_1 \mathcal{T} (u_1) +  \lambda_2 \mathcal{T}(u_2)
		\]
		\item nonlinear otherwise
	\end{itemize}
	\item \textbf{discrete} vs \textbf{continuous}-time system: 
	\begin{itemize}
		\item continuous if both IO are continuous signal
		\item discrete if both IO are discrete signal
		\item \textbf{hybrid} if IO have both types of signals
	\end{itemize}
	\item \textbf{time varying} vs \textbf{time invariant} system: 
	\begin{itemize}
		\item invariant: if $\forall u \in \mathcal{U}$ and $\tau \in \mathcal{I}$:
		\[
		y(\cdot) = \mathcal{T}(u(\cdot)) \Longrightarrow y(\cdot - \tau)  \quad = \quad \mathcal{T}(u(\cdot - \tau))
		\]
		\item varying: otherwise
	\end{itemize}
	\item \textbf{causal} vs \textbf{non-causal} system: 
	\begin{itemize}
		\item causal: if  output only depends on past input
		\item non-causal:  otherwise
	\end{itemize}
	\item \textbf{lump} vs \textbf{distributed} system: 
	\begin{itemize}
		\item lump: if the system has finite number of state variables
		\item distributed: otherwise, namely infinite number state variable
	\end{itemize}
	Usually, lumped system are typically modeled by ODE, while distributed systems arise due to PDE or the presence of delay(require infinite memory).
\end{itemize}

\section{State Variables}
The \textbf{state variables} of a system is a set of internal variables whose values at any moment $t_0$ together with future input $u(t)$, $t>t_0$, are sufficient to determine the system output $y(t)$, $t>t_0$.
\begin{itemize}
	\item it summarize the past input history
	\item also called initial condition in mathematical perspective
\end{itemize}

\subsection{Implication of System with State Variables}
A new IO relation can be obtained by introducing state variables:
\[
y(t) | _{t \geq t_0} = \mathcal{T} \left(u(t) | _{t\geq t_0} , x(t_0)\right)
\]

\subsection{Decomposition of Response}
The response of a linear system can be decomposed as:
\begin{align*}
y(n) &= \mathcal{T} \left(u(t) | _{t\geq t_0} , x(t_0)\right)\\
&=\mathcal{T} \left(u(t) | _{t\geq t_0} , 0\right) +\mathcal{T} \left(0, x(t_0)\right)
\end{align*}
\begin{itemize}
	\item the first term is called \textbf{zero-state} response
	\item the second term is called \textbf{zero-input} response
\end{itemize}

\subsection{General State-Space Model of Lumped System}
Suppose the system has state variable $x \in \mathbb{R}^n$, input $u \in \mathbb{R}^m$ and output $\mathbb{R}^p$:
\begin{itemize}
	\item a continuous system:
	\[
	\left\{
	\begin{array}{ll}
	\dot{x}= f(x(t), u(t), t)\\
	y(t) = g(x(t), u(t), t)\
	\end{array}\right.
	\]
	\item a discrete system:
	\[
	\left\{
	\begin{array}{ll}
	x[k+1] = f(x[k], u[k], k)\\
	y(k) = g(x[k], u[k], k)\
	\end{array}\right.
	\]
\end{itemize}

\subsection{State-Space Model of Linear Lumped System}
Suppose the system has state variable $x \in \mathbb{R}^n$, input $u \in \mathbb{R}^m$ and output $\mathbb{R}^p$:
\begin{itemize}
	\item a continuous system:
	\[
	\left\{
	\begin{array}{ll}
	\frac{dx(t)}{dt} = A(t)x(t) + B(t)u(t)\\
	y(t) = C(t)x(t) + D(t)u(t)\\
	\end{array}\right.
	\]
	\item a discrete system:
	\[
	\left\{
	\begin{array}{ll}
	x[k+1] = A[k]x[k] + B[k]u[k]\\
	y(k) = C[k]x[k] + D[k]u[k])\\
	\end{array}\right.
	\]
\end{itemize}
Note that the general $f$ and $g$ now is a function linear to $x$ and $u$.

\subsection{State-Space Model of Lumped Linear Time-Invariant(LTI) System}
Suppose the system has state variable $x \in \mathbb{R}^n$, input $u \in \mathbb{R}^m$ and output $\mathbb{R}^p$:
\begin{itemize}
	\item a continuous system:
	\[
	\left\{
	\begin{array}{ll}
	\frac{dx(t)}{dt} = Ax(t) + Bu(t)\\
	y(t) = Cx(t) + Du(t)\\
	\end{array}\right.
	\]
	\item a discrete system:
	\[
	\left\{
	\begin{array}{ll}
	x[k+1] = Ax[k] + Bu[k]\\
	y(k) = Cx[k] + Du[k])\\
	\end{array}\right.
	\]
\end{itemize}
Note that the $(A, B, C, D)$ are constants. Here are $2$ important processor in graphical representation:
\begin{itemize}
	\item integration for CT system:
	\[
	\frac{1}{s}
	\]
	usually, the input is state variable
	\item delay for DT system
	\[
	z^{-1}
	\]
	usually, the input is term whose index is  $k+1$  and output with index $k$
\end{itemize}

\section{Extension}
\subsection{Equation for Circuits}
\begin{itemize}
	\item capacitor has characteristic equation:
	\[
	i(t) = C \frac{dV}{dt}
	\]
	choose current $i$ as state variable
	\item inducer has characteristic equation:
	\[
	V(t) = L \frac{di}{dt}
	\]
	choose voltage $V$ as state variable
\end{itemize}
\subsection{General Linear Mechanical System}
A mechanical system with $n$ DOF (degree of freedom)can be written as:
\[
M\ddot{q} + D\dot{q} + Kq = F
\]
\begin{itemize}
	\item $q$: general displacement vector
	\item $M$: mass
	\item $D$: damping
	\item $K$: stiffness
	\item $F$: external force
\end{itemize}
choose $x$ to be:
\[
x = \begin{bmatrix}
q\\
\dot{q}
\end{bmatrix}
\]
\end{document}



