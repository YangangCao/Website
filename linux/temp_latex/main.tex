\documentclass[10pt,a4paper,oneside]{article}
\usepackage[utf8]{inputenc}
\usepackage[T1]{fontenc}
\usepackage{amsmath}
\usepackage{amsfonts}
\usepackage{amssymb}
\usepackage{graphicx}
\usepackage{enumerate}
\date{June 20, 2019}
\author{Baboo J. Cui, Yangang Cao}
\title{Linux Q\&A (CentOS Based)}
\begin{document}
\maketitle
\tableofcontents

\newpage

\section{Django Related Installation}
\begin{enumerate}[1.]
\item How to install python?\\
Use conda directly
\item How to install Django?\\
Use command \$ conda install django, Tip: pip may cause version problem
\item How to check version of Django?\\
Use command \$ python -m django - -version, this can be used for other package
\item What is the default database for Django?\\
It’s SQLite
\item What 2 parts does Django project have?\\
One is the production website another is the admin system
\item How to create a Django project?\\
Use command \$ django-admin startproject PRO\_NAME, this will create a Django project
\item Where should project code be?\\
Any path but not static path(code exposure may cause security problem)
\item Any trick on project name?\\
Do not use any keyword!
\item How to start a Django service?\\
Use command \$ python manage.py runserver, and it is started! (may be migration problems)
\item How to quickly avoid migrate problem?\\
Use command \$ python manage.py migrate(will be detailed latter)
\end{enumerate}
\section{Structure of Django Project}
\begin{enumerate}[1.]
\item Does outmost folder name matter?\\
No, it’s just a container
\item What is manage.py file for?\\
For python interaction in command line
\item What does the package for the project contain?\\
4 parts: \_\_init\_\_.py(empty), setting.py(config file), url.py(URL dispatcher, table of content), wsgi.py(WSGI service)
\item Do we prefer WSGI?\\
No, because of poor performance
\item How to start a Django service with certain port?\\
Use command \$ python manage.py runserver 8080, by default 8000 will be used
\item What is automatic reloading?\\
You don't need to restart the server to make the code to take effects
\end{enumerate}
\section{View}
\begin{enumerate}[1.]
\item How to create an app in Django?\\
Use command \$ python manage.py startapp APP\_NAME, this will create a package
\item What does the package for the app contain?\\
\_\_init\_\_.py, admin.py, apps.py, models.py, test.py, urls.py, view.py and migrations folder
\item What is view.py for?\\
For controlling how and what the webpage is to display, a view is a ``type'' of webpage, represented by a function
\item What does view do?\\
Return a http response or raise an exception like Http404
\item How to write a simple view?\\
1. from django.http import HttpResponse\\
2. def index(request): \#write index function\\
\indent\quad \quad \quad return HttpResponse(``Hi, this is the poll index'')
\item Comment on simple view?\\
Function name can be anything, but must be registered in url(next one)
\item How to register a view in urls.py?(Recommanded)\\
1. create urls.py in the app package, by default, there is no url.py
\\
2. from django.urls import path\\
3. from . import views \# import the corresponding view\\
4. urlpatterns = [path(`', views.index, name=`index')]
\# add url pattern list\\
5. from django.urls import path, include \# register in project url by include() \indent\quad  function
\\
\indent\quad urlpatterns = [path(`admin/', admin.site.urls),\\
\indent\quad path(`polls/', include(`polls.urls')),]
\item Comments on registering in urls.py\\
View needs to be registered to app url, app url needs to be registered to project urls, both by path function
\item Why path for admin is different from the others?\\
Admin case is the only one exception
\item API of path() function?\\
Route: regex contain url pattern(will be append!), view: a function, name: refer to it unambiguously in template
\item How to use URL to pass arg to function?\\
Use angle brackets <>, with format <TYPE: ARG>, e.g <int:question\_id>, url info will be used as arg to pass into func
\end{enumerate}
\section{Settings}
\begin{enumerate}[1.]
\item Where is database config info?\\
In project setting.py, DATABASES dict
\item Why Django use SQLite by default?\\
SQLite is included in python, easier to use
\item How to change database type?\\
In ENGINE entry, start with django.db.backends.ENG\_TYPE, it can be sqlite3(default), mysql, oracle, postgresql
\item How to setup NAME for SQLite3?\\
Just use default
\item What if RuntimeError: cryptography is required?\\
Use command \$ pip install cryptography
\item Preparation in MySQL?\\
You must create a database schema with the target name before use it!
\item How to setup NAME for MySQL?\\
Set it to be the same as the database name in MySQL, e.g ``mysite''
\item Other setup info for MySQL?\\
\$ pip install pymysql is required for connection, (mysqlclient may have problem)
\item How to init pymysql?\\
 Add import pymysql and pymysql.install\_as\_MySQLdb() in project \_\_init\_\_.py
\item All config for MySQL?\\
ENGINE, NAME, USER, PASSWORD, HOST, PORT, all var and value should be in string
\item How to check if MySQL connection works?\\
Use command \$ python manage.py migrate to see if data could be injected into the database
\item How to make log display SQL command(Debug)?\\
Add the following code in project setting.py:
\\
LOGGING = \{\\
\indent\quad\quad	`version': 1,\\
\indent\quad\quad	`disable\_existing\_loggers': False,\\
\indent\quad\quad	`handlers': \{\\
\indent\quad\quad\quad		`console': \{\\
\indent\quad\quad\quad\quad		`level': `DEBUG',\\
\indent\quad\quad\quad\quad		`class': `logging.StreamHandler',\\
\indent\quad\quad\quad		\},\\
\indent\quad\quad	\},\\
\indent\quad\quad	`loggers': \{\\
\indent\quad\quad\quad		`django.db.backends': \{\\
\indent\quad\quad\quad\quad		`handlers': [`console'],\\
\indent\quad\quad\quad\quad		`propagate': True,\\
\indent\quad\quad\quad\quad		`level': `DEBUG',\\
\indent\quad\quad\quad		\},\\
\indent\quad\quad	\}\\
\indent\quad\}
\item What TIME\_ZONE setting should use?\\
Always use TIME\_ZONE = `UTC', other time should be converted
\item Where does all Apps registered?\\
In INSTALLED\_APPS list in project settings, there are several apps by default for connivence 
\item What does migrate command do?\\
Look INSALLED\_APPS settings and create necessary database table according to project settings
\end{enumerate}
\section{Models}
\begin{enumerate}[1.]
\item What is model?\\
Database layout and additional metadata, highly related to database
\item How to import model class?\\
In app models.py, add from django.db import models, and create class inherit from models.Model
\item Mechanism of model class?\\
Each var has a field type which is directly linked to field in database
\item API for models.ForeignKey?\\
Set a field as foreign key, has 2 args, first one is the class name as foreign key, second on\_delete=models.CASCADE
\item How on\_delete works?\\
Control if foreign key needs to be delete if primary key is delete, CASCADE will, and DO\_NOTHING will not
\item API for models.CharField()?\\
Create a string, have a arg max\_length=200, which set the max length
\item API for models.DateTimeField()?\\
Create a time field, arg is a string to give it name
\item API for models.IntegerField()?\\
Create a integer field, arg is default=0, default value should be set
\item How to activate a model?\\
In project setting, add `APP\_NAME.app.APP\_NAMEConfig' in INSTALLED\_APPS
\item How to apply changes in models to database?\\
Use command \$ python manage.py makemigrations [poll], app name is optional
\item How to see corresponding SQL to migrations?\\
Use command \$ python manage.py sqlmigrate APP\_NAME 000x, this will print SQL
\item How to bring migrations into effects?\\
Again, use command \$ python manage.py migrate, this will create corresponding tables in DB
\item Migrate vs makemigrations?\\
That createmigrate creates the changes in model and migrate apply changes to database
\item How to change the model class representation?\\
Use \_\_str\_\_() function, make the return value easy to read
\item How to add method to a model?\\
Directly add a method as general method
\item How to get a model latest list?\\
Use syntax list = CLASS\_NAME.object.order\_by(`-ATTR\_NAME')[num1:num2]
\item How to get a model total list?\\
Use syntax list = CLASS\_NAME.object.all()
\end{enumerate}
\section{Debug and Shell}
\begin{enumerate}[1.]
\item How to use shell to debug?\\
Use command \$ python manage.py shell, which is not recommend
\item How to get current time?\\
Use from django.utils import timezone and use timezone.now() to get current time
\item How to get current year?\\
Use time.now().year
\item More advanced way to debug?\\
Use debug script with format:
\\
import os\\
import django\\
if \_\_name\_\_ == ``\_\_main\_\_'':\\
\indent\quad\quad\#copy from manage.py\\
\indent\quad\quad os.environ.setdefault(`DJANGO\_SETTINGS\_MODULE', `mysite.settings') \\
\indent\quad\quad django.setup()
\item How to see table details?\\
Import corresponding class and use CLASS\_NAME.object.all() to get list
\item How to see record count?\\
Import corresponding class and use CLASS\_NAME.object.count() to num of record
\item How to create a record?\\
VAR = CLASS\_NAME(var1=val1\dots), then VAR.save()
\item How to get field value from a record?\\
VAR.FIELD\_NAME
\item How to set field value for a record?\\
Use VAR.FIELD\_NAME = new\_val, after this, must call VAR.save()
\item How to use filter API?\\
With syntax CLASS\_NAME.object.filter(condition)
\item What condition could be implemented to filter?\\
Value equality condition(id=1), relationship className\_property\_\_filterMethod=value, all should in low case
\end{enumerate}
\section{Django Admin}
\begin{enumerate}[1.]
\item What is Django admin?\\
A site for stuff and management, site for manager
\item How to create Django admin?\\
Use command \$ python manage.py createsuperuser, and then follow instruction
\item How to go to admin site?\\
Use command \$ python manage.py runserver and then go to /admin to login
\item Which app offer service to group and users?\\
By app: django.contrib.auth
\item How to make app modifiable in admin?\\
In app admin.py, import needed class and use admin.site.register(CLASS\_NAME) to do,
Tip: no indication on register() function and don’t register a class more than once!
\end{enumerate}
\section{Template}
\begin{enumerate}[1.]
\item What is template?\\
Mechanism to separate design from python
\item How to create template?\\
Create a folder named ``template'' in app package and then a folder with the same name as app(namespacing issue)
\item Where the template setting is?\\
In project setting.py, TEMPLATES list, by default, `APP\_DIRS': True
\item How to write a template?\\
A Lot\dots coming soon, for now just copy and paste
\end{enumerate}
\end{document}